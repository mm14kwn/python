
% Default to the notebook output style

    


% Inherit from the specified cell style.




    
\documentclass{article}

    
    
    \usepackage{graphicx} % Used to insert images
    \usepackage{adjustbox} % Used to constrain images to a maximum size 
    \usepackage{color} % Allow colors to be defined
    \usepackage{enumerate} % Needed for markdown enumerations to work
    \usepackage{geometry} % Used to adjust the document margins
    \usepackage{amsmath} % Equations
    \usepackage{amssymb} % Equations
    \usepackage{eurosym} % defines \euro
    \usepackage[mathletters]{ucs} % Extended unicode (utf-8) support
    \usepackage[utf8x]{inputenc} % Allow utf-8 characters in the tex document
    \usepackage{fancyvrb} % verbatim replacement that allows latex
    \usepackage{grffile} % extends the file name processing of package graphics 
                         % to support a larger range 
    % The hyperref package gives us a pdf with properly built
    % internal navigation ('pdf bookmarks' for the table of contents,
    % internal cross-reference links, web links for URLs, etc.)
    \usepackage{hyperref}
    \usepackage{longtable} % longtable support required by pandoc >1.10
    \usepackage{booktabs}  % table support for pandoc > 1.12.2
    

    
    
    \definecolor{orange}{cmyk}{0,0.4,0.8,0.2}
    \definecolor{darkorange}{rgb}{.71,0.21,0.01}
    \definecolor{darkgreen}{rgb}{.12,.54,.11}
    \definecolor{myteal}{rgb}{.26, .44, .56}
    \definecolor{gray}{gray}{0.45}
    \definecolor{lightgray}{gray}{.95}
    \definecolor{mediumgray}{gray}{.8}
    \definecolor{inputbackground}{rgb}{.95, .95, .85}
    \definecolor{outputbackground}{rgb}{.95, .95, .95}
    \definecolor{traceback}{rgb}{1, .95, .95}
    % ansi colors
    \definecolor{red}{rgb}{.6,0,0}
    \definecolor{green}{rgb}{0,.65,0}
    \definecolor{brown}{rgb}{0.6,0.6,0}
    \definecolor{blue}{rgb}{0,.145,.698}
    \definecolor{purple}{rgb}{.698,.145,.698}
    \definecolor{cyan}{rgb}{0,.698,.698}
    \definecolor{lightgray}{gray}{0.5}
    
    % bright ansi colors
    \definecolor{darkgray}{gray}{0.25}
    \definecolor{lightred}{rgb}{1.0,0.39,0.28}
    \definecolor{lightgreen}{rgb}{0.48,0.99,0.0}
    \definecolor{lightblue}{rgb}{0.53,0.81,0.92}
    \definecolor{lightpurple}{rgb}{0.87,0.63,0.87}
    \definecolor{lightcyan}{rgb}{0.5,1.0,0.83}
    
    % commands and environments needed by pandoc snippets
    % extracted from the output of `pandoc -s`
    \providecommand{\tightlist}{%
      \setlength{\itemsep}{0pt}\setlength{\parskip}{0pt}}
    \DefineVerbatimEnvironment{Highlighting}{Verbatim}{commandchars=\\\{\}}
    % Add ',fontsize=\small' for more characters per line
    \newenvironment{Shaded}{}{}
    \newcommand{\KeywordTok}[1]{\textcolor[rgb]{0.00,0.44,0.13}{\textbf{{#1}}}}
    \newcommand{\DataTypeTok}[1]{\textcolor[rgb]{0.56,0.13,0.00}{{#1}}}
    \newcommand{\DecValTok}[1]{\textcolor[rgb]{0.25,0.63,0.44}{{#1}}}
    \newcommand{\BaseNTok}[1]{\textcolor[rgb]{0.25,0.63,0.44}{{#1}}}
    \newcommand{\FloatTok}[1]{\textcolor[rgb]{0.25,0.63,0.44}{{#1}}}
    \newcommand{\CharTok}[1]{\textcolor[rgb]{0.25,0.44,0.63}{{#1}}}
    \newcommand{\StringTok}[1]{\textcolor[rgb]{0.25,0.44,0.63}{{#1}}}
    \newcommand{\CommentTok}[1]{\textcolor[rgb]{0.38,0.63,0.69}{\textit{{#1}}}}
    \newcommand{\OtherTok}[1]{\textcolor[rgb]{0.00,0.44,0.13}{{#1}}}
    \newcommand{\AlertTok}[1]{\textcolor[rgb]{1.00,0.00,0.00}{\textbf{{#1}}}}
    \newcommand{\FunctionTok}[1]{\textcolor[rgb]{0.02,0.16,0.49}{{#1}}}
    \newcommand{\RegionMarkerTok}[1]{{#1}}
    \newcommand{\ErrorTok}[1]{\textcolor[rgb]{1.00,0.00,0.00}{\textbf{{#1}}}}
    \newcommand{\NormalTok}[1]{{#1}}
    
    % Define a nice break command that doesn't care if a line doesn't already
    % exist.
    \def\br{\hspace*{\fill} \\* }
    % Math Jax compatability definitions
    \def\gt{>}
    \def\lt{<}
    % Document parameters
    \title{L01\_Introduction}
    
    
    

    % Pygments definitions
    
\makeatletter
\def\PY@reset{\let\PY@it=\relax \let\PY@bf=\relax%
    \let\PY@ul=\relax \let\PY@tc=\relax%
    \let\PY@bc=\relax \let\PY@ff=\relax}
\def\PY@tok#1{\csname PY@tok@#1\endcsname}
\def\PY@toks#1+{\ifx\relax#1\empty\else%
    \PY@tok{#1}\expandafter\PY@toks\fi}
\def\PY@do#1{\PY@bc{\PY@tc{\PY@ul{%
    \PY@it{\PY@bf{\PY@ff{#1}}}}}}}
\def\PY#1#2{\PY@reset\PY@toks#1+\relax+\PY@do{#2}}

\expandafter\def\csname PY@tok@gd\endcsname{\def\PY@tc##1{\textcolor[rgb]{0.63,0.00,0.00}{##1}}}
\expandafter\def\csname PY@tok@gu\endcsname{\let\PY@bf=\textbf\def\PY@tc##1{\textcolor[rgb]{0.50,0.00,0.50}{##1}}}
\expandafter\def\csname PY@tok@gt\endcsname{\def\PY@tc##1{\textcolor[rgb]{0.00,0.27,0.87}{##1}}}
\expandafter\def\csname PY@tok@gs\endcsname{\let\PY@bf=\textbf}
\expandafter\def\csname PY@tok@gr\endcsname{\def\PY@tc##1{\textcolor[rgb]{1.00,0.00,0.00}{##1}}}
\expandafter\def\csname PY@tok@cm\endcsname{\let\PY@it=\textit\def\PY@tc##1{\textcolor[rgb]{0.25,0.50,0.50}{##1}}}
\expandafter\def\csname PY@tok@vg\endcsname{\def\PY@tc##1{\textcolor[rgb]{0.10,0.09,0.49}{##1}}}
\expandafter\def\csname PY@tok@m\endcsname{\def\PY@tc##1{\textcolor[rgb]{0.40,0.40,0.40}{##1}}}
\expandafter\def\csname PY@tok@mh\endcsname{\def\PY@tc##1{\textcolor[rgb]{0.40,0.40,0.40}{##1}}}
\expandafter\def\csname PY@tok@go\endcsname{\def\PY@tc##1{\textcolor[rgb]{0.53,0.53,0.53}{##1}}}
\expandafter\def\csname PY@tok@ge\endcsname{\let\PY@it=\textit}
\expandafter\def\csname PY@tok@vc\endcsname{\def\PY@tc##1{\textcolor[rgb]{0.10,0.09,0.49}{##1}}}
\expandafter\def\csname PY@tok@il\endcsname{\def\PY@tc##1{\textcolor[rgb]{0.40,0.40,0.40}{##1}}}
\expandafter\def\csname PY@tok@cs\endcsname{\let\PY@it=\textit\def\PY@tc##1{\textcolor[rgb]{0.25,0.50,0.50}{##1}}}
\expandafter\def\csname PY@tok@cp\endcsname{\def\PY@tc##1{\textcolor[rgb]{0.74,0.48,0.00}{##1}}}
\expandafter\def\csname PY@tok@gi\endcsname{\def\PY@tc##1{\textcolor[rgb]{0.00,0.63,0.00}{##1}}}
\expandafter\def\csname PY@tok@gh\endcsname{\let\PY@bf=\textbf\def\PY@tc##1{\textcolor[rgb]{0.00,0.00,0.50}{##1}}}
\expandafter\def\csname PY@tok@ni\endcsname{\let\PY@bf=\textbf\def\PY@tc##1{\textcolor[rgb]{0.60,0.60,0.60}{##1}}}
\expandafter\def\csname PY@tok@nl\endcsname{\def\PY@tc##1{\textcolor[rgb]{0.63,0.63,0.00}{##1}}}
\expandafter\def\csname PY@tok@nn\endcsname{\let\PY@bf=\textbf\def\PY@tc##1{\textcolor[rgb]{0.00,0.00,1.00}{##1}}}
\expandafter\def\csname PY@tok@no\endcsname{\def\PY@tc##1{\textcolor[rgb]{0.53,0.00,0.00}{##1}}}
\expandafter\def\csname PY@tok@na\endcsname{\def\PY@tc##1{\textcolor[rgb]{0.49,0.56,0.16}{##1}}}
\expandafter\def\csname PY@tok@nb\endcsname{\def\PY@tc##1{\textcolor[rgb]{0.00,0.50,0.00}{##1}}}
\expandafter\def\csname PY@tok@nc\endcsname{\let\PY@bf=\textbf\def\PY@tc##1{\textcolor[rgb]{0.00,0.00,1.00}{##1}}}
\expandafter\def\csname PY@tok@nd\endcsname{\def\PY@tc##1{\textcolor[rgb]{0.67,0.13,1.00}{##1}}}
\expandafter\def\csname PY@tok@ne\endcsname{\let\PY@bf=\textbf\def\PY@tc##1{\textcolor[rgb]{0.82,0.25,0.23}{##1}}}
\expandafter\def\csname PY@tok@nf\endcsname{\def\PY@tc##1{\textcolor[rgb]{0.00,0.00,1.00}{##1}}}
\expandafter\def\csname PY@tok@si\endcsname{\let\PY@bf=\textbf\def\PY@tc##1{\textcolor[rgb]{0.73,0.40,0.53}{##1}}}
\expandafter\def\csname PY@tok@s2\endcsname{\def\PY@tc##1{\textcolor[rgb]{0.73,0.13,0.13}{##1}}}
\expandafter\def\csname PY@tok@vi\endcsname{\def\PY@tc##1{\textcolor[rgb]{0.10,0.09,0.49}{##1}}}
\expandafter\def\csname PY@tok@nt\endcsname{\let\PY@bf=\textbf\def\PY@tc##1{\textcolor[rgb]{0.00,0.50,0.00}{##1}}}
\expandafter\def\csname PY@tok@nv\endcsname{\def\PY@tc##1{\textcolor[rgb]{0.10,0.09,0.49}{##1}}}
\expandafter\def\csname PY@tok@s1\endcsname{\def\PY@tc##1{\textcolor[rgb]{0.73,0.13,0.13}{##1}}}
\expandafter\def\csname PY@tok@kd\endcsname{\let\PY@bf=\textbf\def\PY@tc##1{\textcolor[rgb]{0.00,0.50,0.00}{##1}}}
\expandafter\def\csname PY@tok@sh\endcsname{\def\PY@tc##1{\textcolor[rgb]{0.73,0.13,0.13}{##1}}}
\expandafter\def\csname PY@tok@sc\endcsname{\def\PY@tc##1{\textcolor[rgb]{0.73,0.13,0.13}{##1}}}
\expandafter\def\csname PY@tok@sx\endcsname{\def\PY@tc##1{\textcolor[rgb]{0.00,0.50,0.00}{##1}}}
\expandafter\def\csname PY@tok@bp\endcsname{\def\PY@tc##1{\textcolor[rgb]{0.00,0.50,0.00}{##1}}}
\expandafter\def\csname PY@tok@c1\endcsname{\let\PY@it=\textit\def\PY@tc##1{\textcolor[rgb]{0.25,0.50,0.50}{##1}}}
\expandafter\def\csname PY@tok@kc\endcsname{\let\PY@bf=\textbf\def\PY@tc##1{\textcolor[rgb]{0.00,0.50,0.00}{##1}}}
\expandafter\def\csname PY@tok@c\endcsname{\let\PY@it=\textit\def\PY@tc##1{\textcolor[rgb]{0.25,0.50,0.50}{##1}}}
\expandafter\def\csname PY@tok@mf\endcsname{\def\PY@tc##1{\textcolor[rgb]{0.40,0.40,0.40}{##1}}}
\expandafter\def\csname PY@tok@err\endcsname{\def\PY@bc##1{\setlength{\fboxsep}{0pt}\fcolorbox[rgb]{1.00,0.00,0.00}{1,1,1}{\strut ##1}}}
\expandafter\def\csname PY@tok@mb\endcsname{\def\PY@tc##1{\textcolor[rgb]{0.40,0.40,0.40}{##1}}}
\expandafter\def\csname PY@tok@ss\endcsname{\def\PY@tc##1{\textcolor[rgb]{0.10,0.09,0.49}{##1}}}
\expandafter\def\csname PY@tok@sr\endcsname{\def\PY@tc##1{\textcolor[rgb]{0.73,0.40,0.53}{##1}}}
\expandafter\def\csname PY@tok@mo\endcsname{\def\PY@tc##1{\textcolor[rgb]{0.40,0.40,0.40}{##1}}}
\expandafter\def\csname PY@tok@kn\endcsname{\let\PY@bf=\textbf\def\PY@tc##1{\textcolor[rgb]{0.00,0.50,0.00}{##1}}}
\expandafter\def\csname PY@tok@mi\endcsname{\def\PY@tc##1{\textcolor[rgb]{0.40,0.40,0.40}{##1}}}
\expandafter\def\csname PY@tok@gp\endcsname{\let\PY@bf=\textbf\def\PY@tc##1{\textcolor[rgb]{0.00,0.00,0.50}{##1}}}
\expandafter\def\csname PY@tok@o\endcsname{\def\PY@tc##1{\textcolor[rgb]{0.40,0.40,0.40}{##1}}}
\expandafter\def\csname PY@tok@kr\endcsname{\let\PY@bf=\textbf\def\PY@tc##1{\textcolor[rgb]{0.00,0.50,0.00}{##1}}}
\expandafter\def\csname PY@tok@s\endcsname{\def\PY@tc##1{\textcolor[rgb]{0.73,0.13,0.13}{##1}}}
\expandafter\def\csname PY@tok@kp\endcsname{\def\PY@tc##1{\textcolor[rgb]{0.00,0.50,0.00}{##1}}}
\expandafter\def\csname PY@tok@w\endcsname{\def\PY@tc##1{\textcolor[rgb]{0.73,0.73,0.73}{##1}}}
\expandafter\def\csname PY@tok@kt\endcsname{\def\PY@tc##1{\textcolor[rgb]{0.69,0.00,0.25}{##1}}}
\expandafter\def\csname PY@tok@ow\endcsname{\let\PY@bf=\textbf\def\PY@tc##1{\textcolor[rgb]{0.67,0.13,1.00}{##1}}}
\expandafter\def\csname PY@tok@sb\endcsname{\def\PY@tc##1{\textcolor[rgb]{0.73,0.13,0.13}{##1}}}
\expandafter\def\csname PY@tok@k\endcsname{\let\PY@bf=\textbf\def\PY@tc##1{\textcolor[rgb]{0.00,0.50,0.00}{##1}}}
\expandafter\def\csname PY@tok@se\endcsname{\let\PY@bf=\textbf\def\PY@tc##1{\textcolor[rgb]{0.73,0.40,0.13}{##1}}}
\expandafter\def\csname PY@tok@sd\endcsname{\let\PY@it=\textit\def\PY@tc##1{\textcolor[rgb]{0.73,0.13,0.13}{##1}}}

\def\PYZbs{\char`\\}
\def\PYZus{\char`\_}
\def\PYZob{\char`\{}
\def\PYZcb{\char`\}}
\def\PYZca{\char`\^}
\def\PYZam{\char`\&}
\def\PYZlt{\char`\<}
\def\PYZgt{\char`\>}
\def\PYZsh{\char`\#}
\def\PYZpc{\char`\%}
\def\PYZdl{\char`\$}
\def\PYZhy{\char`\-}
\def\PYZsq{\char`\'}
\def\PYZdq{\char`\"}
\def\PYZti{\char`\~}
% for compatibility with earlier versions
\def\PYZat{@}
\def\PYZlb{[}
\def\PYZrb{]}
\makeatother


    % Exact colors from NB
    \definecolor{incolor}{rgb}{0.0, 0.0, 0.5}
    \definecolor{outcolor}{rgb}{0.545, 0.0, 0.0}



    
    % Prevent overflowing lines due to hard-to-break entities
    \sloppy 
    % Setup hyperref package
    \hypersetup{
      breaklinks=true,  % so long urls are correctly broken across lines
      colorlinks=true,
      urlcolor=blue,
      linkcolor=darkorange,
      citecolor=darkgreen,
      }
    % Slightly bigger margins than the latex defaults
    
    \geometry{verbose,tmargin=1in,bmargin=1in,lmargin=1in,rmargin=1in}
    
    

    \begin{document}
    
    
    \maketitle
    
    

    
    \#

\section{Scientific Python }\label{scientific-python}

Kevin Stratford ~ ~ ~ ~ ~ ~ ~ ~ ~ ~kevin@epcc.ed.ac.uk Emmanouil
Farsarakis ~ ~ farsarakis@epcc.ed.ac.uk

Other course authors: Neelofer Banglawala Andy Turner Arno Proeme 

    ~

    ~

www.archer.ac.uk support@archer.ac.uk

 

    {[}Intro{]} ~ Course overview

Course website ~ ~ ~ https://hpcarcher.github.io/2015-12-14-Portsmouth/

Course material ~ ~ ~ as above

    {[}Intro{]} ~ Scientific computing

Typical workflow

\begin{itemize}
\item
  Generate data usually from simulation on HPC facilities (also from
  experiment!)
\item
  Process data to generate appropriate results
\item
  Visualise results to understand the significance of our work and gain
  scientific understanding
\item
  Communicate results through publications, presentations, web, etc.
\end{itemize}

 

    {[}Intro{]} ~ Why Python?

Python is a very flexible tool

\begin{itemize}
\itemsep1pt\parskip0pt\parsep0pt
\item
  Simple to learn
\item
  Can program ``procedurally'' or use full-blown OOP
\item
  High level: focus on what code does not how to write it
\item
  Interactive python shell aids rapid prototyping
\item
  Extensive standard library (and packages)
\end{itemize}

    {[}Intro{]} ~ For scientific computing?

Rich set of scientific computing functionality

\begin{itemize}
\itemsep1pt\parskip0pt\parsep0pt
\item
  Standard numerical and scientific libraries
\item
  Extensive graphical functionality
\item
  Can interface with existing C/C++/Fortran code
\item
  Good for control of complex ``workflow''
\end{itemize}

 

    {[}Intro{]} ~ And\ldots{}

\begin{itemize}
\item
  Free and Open Source
\item
  Interactive Python is an alternative to Matlab, R
\item
  Currently a ``growth area''
\item
  Useful links
\item
  https://www.python.org/
\item
  http://www.scipy.org/
\end{itemize}

 

    {[}Intro{]} ~ Core packages for scientific computing

\begin{itemize}
\item
  IPython - an advanced interactive shell
\item
  NumPy - tools for manipulating numerical arrays
\item
  Matplotlib - plotting in 2D and 3D
\item
  SciPy - High-level scientific routines for common algorithms
  e.g.~numerical integration, optimisation, Fourier transforms
\end{itemize}

 

    {[}Intro{]} ~ Other useful packages

\begin{itemize}
\item
  mpi4py : message passing parallel programming
\item
  pandas : data analysis library
\item
  scikit-learn : machine learning
\end{itemize}

\ldots{} and many more \ldots{}

    {[}Intro{]} ~ Some cautions

\begin{itemize}
\item
  Some backward compatability problems between versions 3.x and 2.x
\item
  Many packages can risk complex dependency ``sprawl''
\item
  It's interpreted
\item
  HPC tools (debuggers, profilers) lag behind compiled languages
\end{itemize}

    {[}Intro{]} ~ How to run Python?

Python code is executed by an interpreter: python

Python has two modes of operating:

\begin{itemize}
\itemsep1pt\parskip0pt\parsep0pt
\item
  non-interactive
\item
  interactive
\end{itemize}

 

    {[}Intro{]} ~ Non-interactive Python

In non-interactive mode, run the Python interpreter with a file

\begin{itemize}
\itemsep1pt\parskip0pt\parsep0pt
\item
   \textasciitilde{}\$ python myscript.py 
\item
  Python module files end in .py extension
\item
  ideal for persistent, resueable, large (complex) code
\end{itemize}

    {[}Intro{]} ~ Non-interactive Python : Hello world!

Try it! Run ``helloworld.py'' from the command-line (same directory as
notebook)

\textasciitilde{}\$ python helloworld.py Hello world!

    {[}Intro{]} ~ Interactive Python : the Python shell

Interactive mode : run Python interpreter without a file

\begin{itemize}
\itemsep1pt\parskip0pt\parsep0pt
\item
  Interpreter runs as a Python shell (interactive Python runtime
  environment)
\item
  Type Python commands directly into the Python shell (after the prompt
  \textgreater{}\textgreater{}\textgreater{})
\item
  Ideal for trying out commands, experimenting
\item
  Use Ctrl+d to exit
\item
  By default, lose session history when you exit shell
\end{itemize}

    {[}Intro{]} ~ Interactive Python : Hello world!

 Try it! Launch a Python shell and after the prompt type : ~ print
``Hello world!''

\textasciitilde{}\$ python Python 2.7.10 (default, Jul 01 2015,
09:00:00) {[}GCC 4.2.1 Compatible Apple LLVM 6.0 (clang-600.0.39){]} on
darwin Type ``help'', ``copyright'', ``credits'' or ``license'' for more
information. \textgreater{}\textgreater{}\textgreater{} print ``Hello
world!'' Hello world!

    {[}Intro{]} ~ IPython shell

\begin{itemize}
\item
  IPython is an enhanced Python shell
\item
  Ideal for interactive data manipulation and visualisation
\item
  Has features and built-in commands that make it easier to use than the
  standard Python shell
\item
  Launch an IPython shell like a Python shell (without a script):
  ipython
\item
  Useful links : http://ipython.org/
\end{itemize}

    {[}Intro{]} ~ IPython shell : useful commands

\begin{itemize}
\item
  TAB completion to list available functions in a module
  e.g.~module.a\textless{}TAB\textgreater{}
\item
  Query what function does with ? e.g.~function?
\item
  `Magic' commands with \%
\item
  \%hist to see commands issued so far
\item
  \%save to save commands issued so far
\item
  Paste code straight into IPython shell
\item
  Access system commands using !, e.g. !ls -l
\item
  Command quickref for a summary of capabilities
\end{itemize}

    {[}Intro{]} ~ IPython shell : Hello world!

Try it! Launch an IPython shell and print ``Hello world!''

IPython 4.0.0 -- An enhanced Interactive Python. ? -\textgreater{}
Introduction and overview of IPython's features. \%quickref
-\textgreater{} Quick reference. help -\textgreater{} Python's own help
system. object? -\textgreater{} Details about `object', use `object??'
for extra details. In {[}1{]}: print ``Hello world!'' Hello world!

    {[}Intro{]} ~ Python basics recap : data types

\begin{itemize}
\item
  Data types
\item
  integer e.g. -1, float e.g.~3.1412
\item
  string e.g. `a string' or e.g. ``a string also''
\item
  Dynamically typed, no explicit declaration

  \begin{itemize}
  \itemsep1pt\parskip0pt\parsep0pt
  \item
    e.g.~can have x=100, followed by x=``100'' without error
  \end{itemize}
\item
  Can cast one data type into another e.g.~x=int(``100'')
\end{itemize}

    {[}Intro{]} ~ Python basics recap : data types II

\begin{itemize}
\itemsep1pt\parskip0pt\parsep0pt
\item
  list : elements can be a mix of types e.g. {[}3, ``a'', False{]}
\item
  dictionary : like lists with generalised keys e.g.
  \{``key1'':``value1'', ``key2'':``value2''\}
\item
  tuple : an immutable sequence e.g. (1,2,3) or 1,2,3
\item
  empty tuple: (~)
\item
  ``one-tuple'' must have comma: e.g. (1,)
\end{itemize}

    {[}Intro{]} ~ Python basics recap : importing ~ I

A module is a file (e.g.~mymodule.py) containing Python definitions and
statements

To use functions defined within a module, need to import functionality

    {[}Intro{]} ~ Python basics recap : importing ~ II

There are several ways to import functionality

\begin{enumerate}
\def\labelenumi{\arabic{enumi}.}
\itemsep1pt\parskip0pt\parsep0pt
\item
  import mymodule, call function as mymodule.afunc
\item
  from mymodule import afunc, call function as afunc
\item
  To save typing, can use alias for imported module
\end{enumerate}

\begin{itemize}
\itemsep1pt\parskip0pt\parsep0pt
\item
  import mymodule as mod, then use mod.afunc
\end{itemize}

\begin{enumerate}
\def\labelenumi{\arabic{enumi}.}
\setcounter{enumi}{3}
\itemsep1pt\parskip0pt\parsep0pt
\item
  Avoid importing everything from a module
\end{enumerate}

\begin{itemize}
\itemsep1pt\parskip0pt\parsep0pt
\item
  from module import *
\end{itemize}

    {[}Intro{]} ~ Python basics recap : code structure

Whitespace matters, code blocks are indented with a tab or 4 spaces
(need colons!) * for item in list: ~ ~ ~ \#do some stuff * if condition:
~ ~ ~ \#do some stuff * def myfunc(arg1, arg2): ~ ~ ~ \#do some stuff 

    {[}Intro{]} ~ Python basics recap : functions I

\begin{itemize}
\itemsep1pt\parskip0pt\parsep0pt
\item
  A number of built-in functions

  \begin{itemize}
  \itemsep1pt\parskip0pt\parsep0pt
  \item
    Available from the interpreter, e.g., len(), int() 
  \end{itemize}
\item
  Methods related built-in datatypes (objects)

  \begin{itemize}
  \itemsep1pt\parskip0pt\parsep0pt
  \item
    Via dot notation, e.g., list.append(), list.sort()
  \end{itemize}
\end{itemize}

    {[}Intro{]} ~ Python basics recap : functions II

\begin{itemize}
\itemsep1pt\parskip0pt\parsep0pt
\item
  Very large selection of standard library packages:
\item
  Always available as part of python
\item
  import required package
\item
  invoke method as package.method()
\item
  Growing seleection of additional packages
\item
  May need to install appropriate version
\item
  May need addition to PYTHONPATH
\item
  import package and invoke as above
\end{itemize}

    {[}Intro{]} ~ Python basics recap : script files

\begin{itemize}
\item
  At the top of script files you may see:
\item
  \#!/usr/bin/env python 
\item
  Locate interpreter from your environment's \$PATH
\item
  At the bottom of script files you may see:
\item
  if \_\_name\_\_ == ``\_\emph{main\_}'': ~ ~ ~ ~ ~ import sys ~ ~ ~ ~ ~
  function(sys.argv)
\item
  Makes file usable as a script as well as an importable module
\end{itemize}

    {[}Intro{]} ~ Python basics recap : references

\begin{itemize}
\item
  Careful - variables are references
\item
  variables are references to objects: let a = 3, b = a; if b = 5 then a
  = 5 (try it!)
\item
  Getting help: make use of online documentation
\item
  Documentation
\item
  https://docs.python.org/
\item
  https://www.codecademy.com/learn/python
\end{itemize}

    {[}Intro{]} ~ Warm-up : exercise ~ I

Define a function age that takes a list of years between 1950 and 2015
and returns the median age, and the two ages closest to it, where age =
2015 - year. Assume the input list is randomly ordered and has an odd
number of elements $N$, where $N >= 3$. So for:

\begin{itemize}
\itemsep1pt\parskip0pt\parsep0pt
\item
  years = {[}1989, 1955, 2011, 1943, 1975{]}, age returns {[}26, 40,
  60{]}
\item
  Note: for a sorted list of numbers, the median is the number in the
  middle of the list.
\end{itemize}

    {[}Intro{]} ~ Warm-up : exercise ~ II

Steps to take:

\begin{enumerate}
\def\labelenumi{\arabic{enumi}.}
\itemsep1pt\parskip0pt\parsep0pt
\item
  You will need to create the input list. You can do this manually or
  use Python\ldots{}{[}Hint: you may want to use random.randint(start,
  stop){]}
\item
  You will need to sort the list
\item
  Use list indexing where possible e.g.~list{[}3:5{]}
\item
  Make sure to test your function.
\end{enumerate}

    {[}Intro{]} ~ Warm-up exercise : a solution

~

    \begin{Verbatim}[commandchars=\\\{\}]
{\color{incolor}In [{\color{incolor} }]:} \PY{c}{\PYZsh{} function to calculate the median age and }
        \PY{c}{\PYZsh{} its two neighbours from a list of years}
        
        \PY{k}{def} \PY{n+nf}{medianage}\PY{p}{(}\PY{n}{years}\PY{p}{)}\PY{p}{:}
            \PY{n}{ages}\PY{o}{=}\PY{p}{[}\PY{p}{]}\PY{p}{;}
            \PY{k}{for} \PY{n}{y} \PY{o+ow}{in} \PY{n}{years}\PY{p}{:}
                \PY{n}{ages}\PY{o}{.}\PY{n}{append}\PY{p}{(}\PY{l+m+mi}{2015}\PY{o}{\PYZhy{}}\PY{n}{y}\PY{p}{)}\PY{p}{;}
            \PY{n}{ages}\PY{o}{.}\PY{n}{sort}\PY{p}{(}\PY{p}{)}
            \PY{n}{n} \PY{o}{=} \PY{n+nb}{len}\PY{p}{(}\PY{n}{ages}\PY{p}{)}\PY{p}{;}
            \PY{n}{mid} \PY{o}{=} \PY{n}{n}\PY{o}{/}\PY{l+m+mi}{2}\PY{p}{;}
            \PY{k}{return} \PY{n}{ages}\PY{p}{[}\PY{n}{mid}\PY{o}{\PYZhy{}}\PY{l+m+mi}{1}\PY{p}{:}\PY{n}{mid}\PY{o}{+}\PY{l+m+mi}{2}\PY{p}{]}\PY{p}{;}
        
        \PY{n}{years} \PY{o}{=} \PY{p}{[}\PY{l+m+mi}{1989}\PY{p}{,} \PY{l+m+mi}{1955}\PY{p}{,} \PY{l+m+mi}{2011}\PY{p}{,} \PY{l+m+mi}{1943}\PY{p}{,} \PY{l+m+mi}{1975}\PY{p}{]}\PY{p}{;}
        \PY{n}{medianage}\PY{p}{(}\PY{n}{years}\PY{p}{)}
\end{Verbatim}

    {[}Intro{]} ~ Warm-up : optional ~ I

Read the list of years from a text file, `years.txt', which should have
the total number of years $N$ in the first line, followed by a numbered
list of years:

~ ~ ~ total number of years ~ ~ ~ 1 year ~ ~ ~ \ldots{}

    {[}Intro{]} ~ Warm-up : optional ~ II

To generate the input file `years.txt', you may need:

import sys, ~ yearsfn=open(filename, ``w''), ~ input.close(), ~
output.write(``\{0:2d\} \{1:2d\}\n''.format(i, j))

To read the input file `years.txt', you may need:

input=open(filename, ``w''),~ line = infile.readline(), ~ line.rstrip(),
~ line.split(), ~ int(``9''),

Could you use list comprehension (if you haven't already)? ~ ~ ~ ~ ~
E.g. ~ squared = {[}x*x for x in list{]}

    \begin{Verbatim}[commandchars=\\\{\}]
{\color{incolor}In [{\color{incolor} }]:} \PY{c}{\PYZsh{} extended exercise : create an input file containing}
        \PY{c}{\PYZsh{} years}
        
        \PY{k+kn}{from} \PY{n+nn}{random} \PY{k+kn}{import} \PY{n}{randint}\PY{p}{;}
        \PY{k+kn}{import} \PY{n+nn}{sys}\PY{p}{;}
        \PY{k+kn}{import} \PY{n+nn}{age}\PY{p}{;}
        
        \PY{n}{years} \PY{o}{=} \PY{p}{[}\PY{l+m+mi}{1989}\PY{p}{,} \PY{l+m+mi}{1955}\PY{p}{,} \PY{l+m+mi}{2011}\PY{p}{,} \PY{l+m+mi}{1943}\PY{p}{,} \PY{l+m+mi}{1975}\PY{p}{]}\PY{p}{;}
        \PY{n}{medianage}\PY{p}{(}\PY{n}{years}\PY{p}{)}\PY{p}{;}
        
        \PY{c}{\PYZsh{} using list comprehension}
        \PY{n}{years} \PY{o}{=} \PY{p}{[}\PY{n}{randint}\PY{p}{(}\PY{l+m+mi}{1950}\PY{p}{,}\PY{l+m+mi}{2015}\PY{p}{)} \PY{k}{for} \PY{n}{x} \PY{o+ow}{in} \PY{n+nb}{range}\PY{p}{(}\PY{l+m+mi}{9}\PY{p}{)}\PY{p}{]}\PY{p}{;} 
        
        \PY{c}{\PYZsh{} save years to file}
        \PY{n}{outyears} \PY{o}{=} \PY{n+nb}{open}\PY{p}{(}\PY{l+s}{\PYZdq{}}\PY{l+s}{years.txt}\PY{l+s}{\PYZdq{}}\PY{p}{,}\PY{l+s}{\PYZsq{}}\PY{l+s}{w}\PY{l+s}{\PYZsq{}}\PY{p}{)}\PY{p}{;}
        \PY{n}{i}\PY{o}{=}\PY{l+m+mi}{1}
        \PY{c}{\PYZsh{} output total number of years}
        \PY{n}{outyears}\PY{o}{.}\PY{n}{write}\PY{p}{(}\PY{l+s}{\PYZdq{}}\PY{l+s}{\PYZob{}0:2d\PYZcb{}}\PY{l+s+se}{\PYZbs{}n}\PY{l+s}{\PYZdq{}}\PY{o}{.}\PY{n}{format}\PY{p}{(}\PY{n+nb}{len}\PY{p}{(}\PY{n}{years2}\PY{p}{)}\PY{p}{)}\PY{p}{)}
                       
        \PY{k}{for} \PY{n}{y} \PY{o+ow}{in} \PY{n}{years2}\PY{p}{:}
            \PY{n}{outyears}\PY{o}{.}\PY{n}{write}\PY{p}{(}\PY{l+s}{\PYZdq{}}\PY{l+s}{\PYZob{}0:2d\PYZcb{} \PYZob{}1:2d\PYZcb{}}\PY{l+s+se}{\PYZbs{}n}\PY{l+s}{\PYZdq{}}\PY{o}{.}\PY{n}{format}\PY{p}{(}\PY{n}{i}\PY{p}{,} \PY{n}{y}\PY{p}{)}\PY{p}{)}\PY{p}{;}
            \PY{n}{i}\PY{o}{+}\PY{o}{=}\PY{l+m+mi}{1}\PY{p}{;}
            
        \PY{n}{outyears}\PY{o}{.}\PY{n}{close}\PY{p}{(}\PY{p}{)}\PY{p}{;}
        
        \PY{c}{\PYZsh{} read file}
        \PY{n}{inyears} \PY{o}{=} \PY{n+nb}{open}\PY{p}{(}\PY{l+s}{\PYZdq{}}\PY{l+s}{years.txt}\PY{l+s}{\PYZdq{}}\PY{p}{,}\PY{l+s}{\PYZdq{}}\PY{l+s}{r}\PY{l+s}{\PYZdq{}}\PY{p}{)}\PY{p}{;}
        \PY{n}{line} \PY{o}{=} \PY{n}{inyears}\PY{o}{.}\PY{n}{readline}\PY{p}{(}\PY{p}{)}
        \PY{c}{\PYZsh{} remove any newline characters}
        \PY{n}{N}\PY{o}{=}\PY{n+nb}{int}\PY{p}{(}\PY{n}{line}\PY{p}{)}\PY{p}{;}
        
        \PY{n}{ages}\PY{o}{=}\PY{p}{[}\PY{p}{]}\PY{p}{;}
        \PY{n}{yrs}\PY{o}{=}\PY{p}{[}\PY{p}{]}\PY{p}{;}
        \PY{k}{for} \PY{n}{i} \PY{o+ow}{in} \PY{n+nb}{range}\PY{p}{(}\PY{n}{N}\PY{p}{)}\PY{p}{:}
            \PY{n}{line} \PY{o}{=} \PY{n}{inyears}\PY{o}{.}\PY{n}{readline}\PY{p}{(}\PY{p}{)}
            \PY{n}{line} \PY{o}{=} \PY{n}{line}\PY{o}{.}\PY{n}{rstrip}\PY{p}{(}\PY{p}{)}\PY{p}{;}
            \PY{n}{tokens} \PY{o}{=} \PY{n}{line}\PY{o}{.}\PY{n}{split}\PY{p}{(}\PY{p}{)}\PY{p}{;}
            \PY{n}{yrs}\PY{o}{.}\PY{n}{append}\PY{p}{(}\PY{n+nb}{int}\PY{p}{(}\PY{n}{tokens}\PY{p}{[}\PY{l+m+mi}{1}\PY{p}{]}\PY{p}{)}\PY{p}{)}\PY{p}{;}
        
        \PY{n}{inyears}\PY{o}{.}\PY{n}{close}\PY{p}{(}\PY{p}{)}
        
        \PY{c}{\PYZsh{} check...}
        \PY{k}{print} \PY{n}{yrs}
        
        \PY{c}{\PYZsh{} get median age}
        \PY{n}{medianage}\PY{p}{(}\PY{n}{yrs}\PY{p}{)}
\end{Verbatim}

    {[}Intro{]} Summary

We have considered why Scientific Computing in Python is a worthwhile
pursuit

We have reviewed how to use Python together with some Python basics

We have also been introduced to the IPython shell

We are now ready to explore the packages that constitute the backbone of
scientific python

Next session : NumPy


    % Add a bibliography block to the postdoc
    
    
    
    \end{document}
