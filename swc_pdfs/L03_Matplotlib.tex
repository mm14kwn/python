
% Default to the notebook output style

    


% Inherit from the specified cell style.




    
\documentclass{article}

    
    
    \usepackage{graphicx} % Used to insert images
    \usepackage{adjustbox} % Used to constrain images to a maximum size 
    \usepackage{color} % Allow colors to be defined
    \usepackage{enumerate} % Needed for markdown enumerations to work
    \usepackage{geometry} % Used to adjust the document margins
    \usepackage{amsmath} % Equations
    \usepackage{amssymb} % Equations
    \usepackage{eurosym} % defines \euro
    \usepackage[mathletters]{ucs} % Extended unicode (utf-8) support
    \usepackage[utf8x]{inputenc} % Allow utf-8 characters in the tex document
    \usepackage{fancyvrb} % verbatim replacement that allows latex
    \usepackage{grffile} % extends the file name processing of package graphics 
                         % to support a larger range 
    % The hyperref package gives us a pdf with properly built
    % internal navigation ('pdf bookmarks' for the table of contents,
    % internal cross-reference links, web links for URLs, etc.)
    \usepackage{hyperref}
    \usepackage{longtable} % longtable support required by pandoc >1.10
    \usepackage{booktabs}  % table support for pandoc > 1.12.2
    

    
    
    \definecolor{orange}{cmyk}{0,0.4,0.8,0.2}
    \definecolor{darkorange}{rgb}{.71,0.21,0.01}
    \definecolor{darkgreen}{rgb}{.12,.54,.11}
    \definecolor{myteal}{rgb}{.26, .44, .56}
    \definecolor{gray}{gray}{0.45}
    \definecolor{lightgray}{gray}{.95}
    \definecolor{mediumgray}{gray}{.8}
    \definecolor{inputbackground}{rgb}{.95, .95, .85}
    \definecolor{outputbackground}{rgb}{.95, .95, .95}
    \definecolor{traceback}{rgb}{1, .95, .95}
    % ansi colors
    \definecolor{red}{rgb}{.6,0,0}
    \definecolor{green}{rgb}{0,.65,0}
    \definecolor{brown}{rgb}{0.6,0.6,0}
    \definecolor{blue}{rgb}{0,.145,.698}
    \definecolor{purple}{rgb}{.698,.145,.698}
    \definecolor{cyan}{rgb}{0,.698,.698}
    \definecolor{lightgray}{gray}{0.5}
    
    % bright ansi colors
    \definecolor{darkgray}{gray}{0.25}
    \definecolor{lightred}{rgb}{1.0,0.39,0.28}
    \definecolor{lightgreen}{rgb}{0.48,0.99,0.0}
    \definecolor{lightblue}{rgb}{0.53,0.81,0.92}
    \definecolor{lightpurple}{rgb}{0.87,0.63,0.87}
    \definecolor{lightcyan}{rgb}{0.5,1.0,0.83}
    
    % commands and environments needed by pandoc snippets
    % extracted from the output of `pandoc -s`
    \providecommand{\tightlist}{%
      \setlength{\itemsep}{0pt}\setlength{\parskip}{0pt}}
    \DefineVerbatimEnvironment{Highlighting}{Verbatim}{commandchars=\\\{\}}
    % Add ',fontsize=\small' for more characters per line
    \newenvironment{Shaded}{}{}
    \newcommand{\KeywordTok}[1]{\textcolor[rgb]{0.00,0.44,0.13}{\textbf{{#1}}}}
    \newcommand{\DataTypeTok}[1]{\textcolor[rgb]{0.56,0.13,0.00}{{#1}}}
    \newcommand{\DecValTok}[1]{\textcolor[rgb]{0.25,0.63,0.44}{{#1}}}
    \newcommand{\BaseNTok}[1]{\textcolor[rgb]{0.25,0.63,0.44}{{#1}}}
    \newcommand{\FloatTok}[1]{\textcolor[rgb]{0.25,0.63,0.44}{{#1}}}
    \newcommand{\CharTok}[1]{\textcolor[rgb]{0.25,0.44,0.63}{{#1}}}
    \newcommand{\StringTok}[1]{\textcolor[rgb]{0.25,0.44,0.63}{{#1}}}
    \newcommand{\CommentTok}[1]{\textcolor[rgb]{0.38,0.63,0.69}{\textit{{#1}}}}
    \newcommand{\OtherTok}[1]{\textcolor[rgb]{0.00,0.44,0.13}{{#1}}}
    \newcommand{\AlertTok}[1]{\textcolor[rgb]{1.00,0.00,0.00}{\textbf{{#1}}}}
    \newcommand{\FunctionTok}[1]{\textcolor[rgb]{0.02,0.16,0.49}{{#1}}}
    \newcommand{\RegionMarkerTok}[1]{{#1}}
    \newcommand{\ErrorTok}[1]{\textcolor[rgb]{1.00,0.00,0.00}{\textbf{{#1}}}}
    \newcommand{\NormalTok}[1]{{#1}}
    
    % Define a nice break command that doesn't care if a line doesn't already
    % exist.
    \def\br{\hspace*{\fill} \\* }
    % Math Jax compatability definitions
    \def\gt{>}
    \def\lt{<}
    % Document parameters
    \title{L03\_Matplotlib}
    
    
    

    % Pygments definitions
    
\makeatletter
\def\PY@reset{\let\PY@it=\relax \let\PY@bf=\relax%
    \let\PY@ul=\relax \let\PY@tc=\relax%
    \let\PY@bc=\relax \let\PY@ff=\relax}
\def\PY@tok#1{\csname PY@tok@#1\endcsname}
\def\PY@toks#1+{\ifx\relax#1\empty\else%
    \PY@tok{#1}\expandafter\PY@toks\fi}
\def\PY@do#1{\PY@bc{\PY@tc{\PY@ul{%
    \PY@it{\PY@bf{\PY@ff{#1}}}}}}}
\def\PY#1#2{\PY@reset\PY@toks#1+\relax+\PY@do{#2}}

\expandafter\def\csname PY@tok@gd\endcsname{\def\PY@tc##1{\textcolor[rgb]{0.63,0.00,0.00}{##1}}}
\expandafter\def\csname PY@tok@gu\endcsname{\let\PY@bf=\textbf\def\PY@tc##1{\textcolor[rgb]{0.50,0.00,0.50}{##1}}}
\expandafter\def\csname PY@tok@gt\endcsname{\def\PY@tc##1{\textcolor[rgb]{0.00,0.27,0.87}{##1}}}
\expandafter\def\csname PY@tok@gs\endcsname{\let\PY@bf=\textbf}
\expandafter\def\csname PY@tok@gr\endcsname{\def\PY@tc##1{\textcolor[rgb]{1.00,0.00,0.00}{##1}}}
\expandafter\def\csname PY@tok@cm\endcsname{\let\PY@it=\textit\def\PY@tc##1{\textcolor[rgb]{0.25,0.50,0.50}{##1}}}
\expandafter\def\csname PY@tok@vg\endcsname{\def\PY@tc##1{\textcolor[rgb]{0.10,0.09,0.49}{##1}}}
\expandafter\def\csname PY@tok@m\endcsname{\def\PY@tc##1{\textcolor[rgb]{0.40,0.40,0.40}{##1}}}
\expandafter\def\csname PY@tok@mh\endcsname{\def\PY@tc##1{\textcolor[rgb]{0.40,0.40,0.40}{##1}}}
\expandafter\def\csname PY@tok@go\endcsname{\def\PY@tc##1{\textcolor[rgb]{0.53,0.53,0.53}{##1}}}
\expandafter\def\csname PY@tok@ge\endcsname{\let\PY@it=\textit}
\expandafter\def\csname PY@tok@vc\endcsname{\def\PY@tc##1{\textcolor[rgb]{0.10,0.09,0.49}{##1}}}
\expandafter\def\csname PY@tok@il\endcsname{\def\PY@tc##1{\textcolor[rgb]{0.40,0.40,0.40}{##1}}}
\expandafter\def\csname PY@tok@cs\endcsname{\let\PY@it=\textit\def\PY@tc##1{\textcolor[rgb]{0.25,0.50,0.50}{##1}}}
\expandafter\def\csname PY@tok@cp\endcsname{\def\PY@tc##1{\textcolor[rgb]{0.74,0.48,0.00}{##1}}}
\expandafter\def\csname PY@tok@gi\endcsname{\def\PY@tc##1{\textcolor[rgb]{0.00,0.63,0.00}{##1}}}
\expandafter\def\csname PY@tok@gh\endcsname{\let\PY@bf=\textbf\def\PY@tc##1{\textcolor[rgb]{0.00,0.00,0.50}{##1}}}
\expandafter\def\csname PY@tok@ni\endcsname{\let\PY@bf=\textbf\def\PY@tc##1{\textcolor[rgb]{0.60,0.60,0.60}{##1}}}
\expandafter\def\csname PY@tok@nl\endcsname{\def\PY@tc##1{\textcolor[rgb]{0.63,0.63,0.00}{##1}}}
\expandafter\def\csname PY@tok@nn\endcsname{\let\PY@bf=\textbf\def\PY@tc##1{\textcolor[rgb]{0.00,0.00,1.00}{##1}}}
\expandafter\def\csname PY@tok@no\endcsname{\def\PY@tc##1{\textcolor[rgb]{0.53,0.00,0.00}{##1}}}
\expandafter\def\csname PY@tok@na\endcsname{\def\PY@tc##1{\textcolor[rgb]{0.49,0.56,0.16}{##1}}}
\expandafter\def\csname PY@tok@nb\endcsname{\def\PY@tc##1{\textcolor[rgb]{0.00,0.50,0.00}{##1}}}
\expandafter\def\csname PY@tok@nc\endcsname{\let\PY@bf=\textbf\def\PY@tc##1{\textcolor[rgb]{0.00,0.00,1.00}{##1}}}
\expandafter\def\csname PY@tok@nd\endcsname{\def\PY@tc##1{\textcolor[rgb]{0.67,0.13,1.00}{##1}}}
\expandafter\def\csname PY@tok@ne\endcsname{\let\PY@bf=\textbf\def\PY@tc##1{\textcolor[rgb]{0.82,0.25,0.23}{##1}}}
\expandafter\def\csname PY@tok@nf\endcsname{\def\PY@tc##1{\textcolor[rgb]{0.00,0.00,1.00}{##1}}}
\expandafter\def\csname PY@tok@si\endcsname{\let\PY@bf=\textbf\def\PY@tc##1{\textcolor[rgb]{0.73,0.40,0.53}{##1}}}
\expandafter\def\csname PY@tok@s2\endcsname{\def\PY@tc##1{\textcolor[rgb]{0.73,0.13,0.13}{##1}}}
\expandafter\def\csname PY@tok@vi\endcsname{\def\PY@tc##1{\textcolor[rgb]{0.10,0.09,0.49}{##1}}}
\expandafter\def\csname PY@tok@nt\endcsname{\let\PY@bf=\textbf\def\PY@tc##1{\textcolor[rgb]{0.00,0.50,0.00}{##1}}}
\expandafter\def\csname PY@tok@nv\endcsname{\def\PY@tc##1{\textcolor[rgb]{0.10,0.09,0.49}{##1}}}
\expandafter\def\csname PY@tok@s1\endcsname{\def\PY@tc##1{\textcolor[rgb]{0.73,0.13,0.13}{##1}}}
\expandafter\def\csname PY@tok@kd\endcsname{\let\PY@bf=\textbf\def\PY@tc##1{\textcolor[rgb]{0.00,0.50,0.00}{##1}}}
\expandafter\def\csname PY@tok@sh\endcsname{\def\PY@tc##1{\textcolor[rgb]{0.73,0.13,0.13}{##1}}}
\expandafter\def\csname PY@tok@sc\endcsname{\def\PY@tc##1{\textcolor[rgb]{0.73,0.13,0.13}{##1}}}
\expandafter\def\csname PY@tok@sx\endcsname{\def\PY@tc##1{\textcolor[rgb]{0.00,0.50,0.00}{##1}}}
\expandafter\def\csname PY@tok@bp\endcsname{\def\PY@tc##1{\textcolor[rgb]{0.00,0.50,0.00}{##1}}}
\expandafter\def\csname PY@tok@c1\endcsname{\let\PY@it=\textit\def\PY@tc##1{\textcolor[rgb]{0.25,0.50,0.50}{##1}}}
\expandafter\def\csname PY@tok@kc\endcsname{\let\PY@bf=\textbf\def\PY@tc##1{\textcolor[rgb]{0.00,0.50,0.00}{##1}}}
\expandafter\def\csname PY@tok@c\endcsname{\let\PY@it=\textit\def\PY@tc##1{\textcolor[rgb]{0.25,0.50,0.50}{##1}}}
\expandafter\def\csname PY@tok@mf\endcsname{\def\PY@tc##1{\textcolor[rgb]{0.40,0.40,0.40}{##1}}}
\expandafter\def\csname PY@tok@err\endcsname{\def\PY@bc##1{\setlength{\fboxsep}{0pt}\fcolorbox[rgb]{1.00,0.00,0.00}{1,1,1}{\strut ##1}}}
\expandafter\def\csname PY@tok@mb\endcsname{\def\PY@tc##1{\textcolor[rgb]{0.40,0.40,0.40}{##1}}}
\expandafter\def\csname PY@tok@ss\endcsname{\def\PY@tc##1{\textcolor[rgb]{0.10,0.09,0.49}{##1}}}
\expandafter\def\csname PY@tok@sr\endcsname{\def\PY@tc##1{\textcolor[rgb]{0.73,0.40,0.53}{##1}}}
\expandafter\def\csname PY@tok@mo\endcsname{\def\PY@tc##1{\textcolor[rgb]{0.40,0.40,0.40}{##1}}}
\expandafter\def\csname PY@tok@kn\endcsname{\let\PY@bf=\textbf\def\PY@tc##1{\textcolor[rgb]{0.00,0.50,0.00}{##1}}}
\expandafter\def\csname PY@tok@mi\endcsname{\def\PY@tc##1{\textcolor[rgb]{0.40,0.40,0.40}{##1}}}
\expandafter\def\csname PY@tok@gp\endcsname{\let\PY@bf=\textbf\def\PY@tc##1{\textcolor[rgb]{0.00,0.00,0.50}{##1}}}
\expandafter\def\csname PY@tok@o\endcsname{\def\PY@tc##1{\textcolor[rgb]{0.40,0.40,0.40}{##1}}}
\expandafter\def\csname PY@tok@kr\endcsname{\let\PY@bf=\textbf\def\PY@tc##1{\textcolor[rgb]{0.00,0.50,0.00}{##1}}}
\expandafter\def\csname PY@tok@s\endcsname{\def\PY@tc##1{\textcolor[rgb]{0.73,0.13,0.13}{##1}}}
\expandafter\def\csname PY@tok@kp\endcsname{\def\PY@tc##1{\textcolor[rgb]{0.00,0.50,0.00}{##1}}}
\expandafter\def\csname PY@tok@w\endcsname{\def\PY@tc##1{\textcolor[rgb]{0.73,0.73,0.73}{##1}}}
\expandafter\def\csname PY@tok@kt\endcsname{\def\PY@tc##1{\textcolor[rgb]{0.69,0.00,0.25}{##1}}}
\expandafter\def\csname PY@tok@ow\endcsname{\let\PY@bf=\textbf\def\PY@tc##1{\textcolor[rgb]{0.67,0.13,1.00}{##1}}}
\expandafter\def\csname PY@tok@sb\endcsname{\def\PY@tc##1{\textcolor[rgb]{0.73,0.13,0.13}{##1}}}
\expandafter\def\csname PY@tok@k\endcsname{\let\PY@bf=\textbf\def\PY@tc##1{\textcolor[rgb]{0.00,0.50,0.00}{##1}}}
\expandafter\def\csname PY@tok@se\endcsname{\let\PY@bf=\textbf\def\PY@tc##1{\textcolor[rgb]{0.73,0.40,0.13}{##1}}}
\expandafter\def\csname PY@tok@sd\endcsname{\let\PY@it=\textit\def\PY@tc##1{\textcolor[rgb]{0.73,0.13,0.13}{##1}}}

\def\PYZbs{\char`\\}
\def\PYZus{\char`\_}
\def\PYZob{\char`\{}
\def\PYZcb{\char`\}}
\def\PYZca{\char`\^}
\def\PYZam{\char`\&}
\def\PYZlt{\char`\<}
\def\PYZgt{\char`\>}
\def\PYZsh{\char`\#}
\def\PYZpc{\char`\%}
\def\PYZdl{\char`\$}
\def\PYZhy{\char`\-}
\def\PYZsq{\char`\'}
\def\PYZdq{\char`\"}
\def\PYZti{\char`\~}
% for compatibility with earlier versions
\def\PYZat{@}
\def\PYZlb{[}
\def\PYZrb{]}
\makeatother


    % Exact colors from NB
    \definecolor{incolor}{rgb}{0.0, 0.0, 0.5}
    \definecolor{outcolor}{rgb}{0.545, 0.0, 0.0}



    
    % Prevent overflowing lines due to hard-to-break entities
    \sloppy 
    % Setup hyperref package
    \hypersetup{
      breaklinks=true,  % so long urls are correctly broken across lines
      colorlinks=true,
      urlcolor=blue,
      linkcolor=darkorange,
      citecolor=darkgreen,
      }
    % Slightly bigger margins than the latex defaults
    
    \geometry{verbose,tmargin=1in,bmargin=1in,lmargin=1in,rmargin=1in}
    
    

    \begin{document}
    
    
    \maketitle
    
    

    
    \section{Matplotlib }\label{matplotlib}

Kevin Stratford ~ ~ ~ ~ ~ ~ ~ ~ ~ ~kevin@epcc.ed.ac.uk Emmanouil
Farsarakis ~ ~ farsarakis@epcc.ed.ac.uk

Other course authors: Neelofer Banglawala Andy Turner Arno Proeme 

    ~

    ~

www.archer.ac.uk support@archer.ac.uk

 

    {[}Matplotlib{]} ~ What is matplotlib? ~ I

Matplotlib is a plotting library for Python

~ ~ ``make the easy things easy and the hard things possible''

\begin{itemize}
\item
  Capable of:
\item
  interactive and non-interactive plotting
\item
  Producing publication-quality figures
\item
  Large amount of functionality:
\item
  scientific and statistical plots, heatmaps
\item
  surfaces, map-based plotting and more\ldots{}
\end{itemize}

    {[}Matplotlib{]} ~ What is matplotlib? ~ II

Matplotlib is Closely integrated with NumPy

\begin{itemize}
\item
  Use numpy functions for reading data
\item
  As data is in numpy, matplotlib can plot it easily
\item
  Documentation:
\item
  http://matplotlib.org/
\end{itemize}

    {[}Matplotlib{]} ~ What is matplolib? ~ III

\begin{itemize}
\item
  People often want to have a quick look at data in a plain text file
\item
  Gnuplot/Excel often used for this but matplotlib can provide a simple,
  feature-rich replacement
\item
  Manipulate data interactively and replot
\item
  Can save the session to keep record of what you did if required
\end{itemize}

Creating high-quality plots is easy in matplotlib!

    \begin{itemize}
\itemsep1pt\parskip0pt\parsep0pt
\item
  2D plotting only (VTK for 3D plots)
\item
  grew out of MATLAB
\item
  mostly pure Python but makes heavy use of numpy and is efficient with
  very large arrays\ldots{}
\item
  requirements for developer when looking for a plotting package (can I
  think of other requirements?)-- Plots should look great - publication
  quality. One important requirement for me is that the text looks good
  (antialiased, etc.)
\item
  Postscript output for inclusion with TeX documents
\item
  Embeddable in a graphical user interface for application development
\item
  Code should be easy enough that I can understand it and extend it
\item
  Making plots should be easy
\end{itemize}

    \begin{itemize}
\item
  fig.show() \textless{}-- when to use this? in scripts only?
\item
  launching an IPython shell with the \texttt{-{}-pylab} option tells
  matplotlib to plot figures to the screen i.e.~activates
  ``interactive'' plotting mode
\end{itemize}

matplotlib --\textgreater{} pyplot Explain pylab = pyplot + numpy, to
create convenient environment for interactive plotting BUT no longer
recommended as this is like : from pylab import \texttt{*} and universal
imports are discouraged. So either

ipython --matplotlib : who, \%who --\textgreater{} interactive namespace
empty for both ipython --pylab : \%who --\textgreater{} interactive
namespace empty, who --\textgreater{}

fig = figure() --\textgreater{} don't need this line in ipython
generally, but will if using ARCHER. figure will be generated by
default, this gives a handle to figure, and helps distinguish when you
have more than one figure plt.plot()

\begin{itemize}
\itemsep1pt\parskip0pt\parsep0pt
\item
  test plot\_this.py on ARCHER\ldots{} see with and without X server,
  does script try to produce a figure? Why would we need to use Agg?
  Perhaps we don't when running non-interactively? Check\ldots{} To
  support all of these use cases, matplotlib can target different
  outputs, and each of these capabilities is called a backend; the
  ``frontend'' is the user facing code, ie the plotting code, whereas
  the ``backend'' does all the hard work behind-the-scenes to make the
  figure. There are two types of backends: user interface backends (for
  use in pygtk, wxpython, tkinter, qt, macosx, or fltk; also referred to
  as ``interactive backends'') and hardcopy backends to make image files
  (PNG, SVG, PDF, PS; also referred to as ``non-interactive backends'')
\end{itemize}

MAKE CLEAR some backends suppress need for diplay (don't confuse with
`interactive')?

When in non-interactive mode, figures don't display anyway. Calling Agg
ensures nice rendering of figure, don't need to suppress X server
display? BE CLEAR why do you need ``use(''Agg``) in a script???

    {[}Matplotlib{]} ~ Basic concepts ~ I

\begin{itemize}
\item
  Everything is assembled by Python commands
\item
  Create a figure with an axes area (this is the plotting area)
\item
  Can create multiple plots in one figure
\end{itemize}

    {[}Matplotlib{]} ~ Basic concepts ~ II

\begin{itemize}
\item
  Only one figure (or axes) is active at a given time (i.e.~current
  figure, current axes)
\item
  In an IPython shell, can plot to the screen (interactive mode) or save
  to image (non-interactive mode)
\item
  Can use the show() command in, for example, a Python script to display
  the plot
\end{itemize}

matplotlib.pyplot contains the high-level functions we need to do all
the above and more

    Notes * Matplotlib is the whole package; matplotlib.pyplot is a module
in matplotlib; and pylab is a module that gets installed alongside
matplotlib.

\begin{itemize}
\item
  Pyplot provides the state-machine interface to the underlying
  object-oriented plotting library. The state-machine implicitly and
  automatically creates figures and axes to achieve the desired plot.
  For example
\item
  pylab is a convenience module that bulk imports matplotlib.pyplot (for
  plotting) and numpy (for mathematics and working with arrays) in a
  single name space. Although many examples use pylab, it is no longer
  recommended.
\end{itemize}

    {[}Matplotlib{]} ~ Basic plotting

Launch an IPython shell, import pyplot and numpy

    Notes 9. explain ``\%matplotlib'' and say that you will explain later
what this does 1. basic plot, line 2. use symbols (change pts),
different colour 3. add x again with different line colour 4. show fig =
plt.figure() --\textgreater{} plt.title(`A Title'); 5. then plt.plot()
again 6. add second plot to plt.plot()

    \begin{Verbatim}[commandchars=\\\{\}]
{\color{incolor}In [{\color{incolor}3}]:} \PY{c}{\PYZsh{} add \PYZsq{}inline\PYZsq{} option if using a notebook}
        \PY{o}{\PYZpc{}}\PY{k}{matplotlib} inline 
        \PY{k+kn}{import} \PY{n+nn}{matplotlib.pyplot} \PY{k+kn}{as} \PY{n+nn}{plt}\PY{p}{;} \PY{k+kn}{import} \PY{n+nn}{numpy} \PY{k+kn}{as} \PY{n+nn}{np}
\end{Verbatim}

    \begin{Verbatim}[commandchars=\\\{\}]
{\color{incolor}In [{\color{incolor}4}]:} \PY{n}{xmin}\PY{o}{=}\PY{l+m+mi}{0}\PY{p}{;} \PY{n}{xmax}\PY{o}{=}\PY{l+m+mi}{10}\PY{p}{;} \PY{n}{pts} \PY{o}{=} \PY{l+m+mi}{50}\PY{p}{;}
        \PY{n}{x} \PY{o}{=} \PY{n}{np}\PY{o}{.}\PY{n}{linspace}\PY{p}{(}\PY{n}{xmin}\PY{p}{,} \PY{n}{xmax}\PY{p}{,} \PY{n}{pts}\PY{p}{)}\PY{p}{;} 
        \PY{n}{y} \PY{o}{=} \PY{n}{np}\PY{o}{.}\PY{n}{cos}\PY{p}{(}\PY{n}{x}\PY{p}{)}\PY{p}{;}
\end{Verbatim}

    \begin{Verbatim}[commandchars=\\\{\}]
{\color{incolor}In [{\color{incolor}5}]:} \PY{c}{\PYZsh{} line, markers, 2 plots, fig, title then plot}
        \PY{n}{plt}\PY{o}{.}\PY{n}{plot}\PY{p}{(}\PY{n}{x}\PY{p}{,}\PY{n}{y}\PY{p}{,}\PY{l+s}{\PYZsq{}}\PY{l+s}{ro}\PY{l+s}{\PYZsq{}}\PY{p}{)}\PY{p}{;} \PY{c}{\PYZsh{} , x, y, \PYZsq{}g\PYZhy{}\PYZsq{}\PYZsh{} \PYZsh{}}
\end{Verbatim}

    \begin{center}
    \adjustimage{max size={0.9\linewidth}{0.9\paperheight}}{L03_Matplotlib_files/L03_Matplotlib_15_0.png}
    \end{center}
    { \hspace*{\fill} \\}
    
    ~

    {[}Matplotlib{]} ~ Saving images to file

\begin{itemize}
\item
  Saving to image file is simple using savefig
\item
  File format is determined from the extension you supply
\item
  Resolution set using dpi option
\item
  Commonly supports: png, jpg, pdf, ps
\end{itemize}

    Note 1. Students to do basic plotting exercise (up to saving image).
Give them 10-15 mins to do this

    \begin{Verbatim}[commandchars=\\\{\}]
{\color{incolor}In [{\color{incolor} }]:} \PY{c}{\PYZsh{} save image to file in different formats}
        \PY{n}{plt}\PY{o}{.}\PY{n}{savefig}\PY{p}{(}\PY{l+s}{\PYZdq{}}\PY{l+s}{cos\PYZus{}plot.pdf}\PY{l+s}{\PYZdq{}}\PY{p}{)}\PY{p}{;}
        \PY{n}{plt}\PY{o}{.}\PY{n}{savefig}\PY{p}{(}\PY{l+s}{\PYZdq{}}\PY{l+s}{cos\PYZus{}plot.png}\PY{l+s}{\PYZdq{}}\PY{p}{,} \PY{n}{dpi}\PY{o}{=}\PY{l+m+mi}{300}\PY{p}{)}\PY{p}{;}  \PY{c}{\PYZsh{} higher resolution (dpi) }
\end{Verbatim}

    Time to create some plots. Please complete Basic Plotting (pages 1 - 7)
of the Matplotlib exercise.

    {[}Matplotlib{]} ~ What is a backend? (++) ~ I

Matplotlib consists of two parts, a frontend and a backend:

\begin{itemize}
\itemsep1pt\parskip0pt\parsep0pt
\item
  Frontend : the user facing code i.e the plotting code
\item
  Backend : does all the hard work behind-the-scenes to make the figure
\end{itemize}

By offering different backends, Matplotlib can support a wide range of
different use cases and output formats.

    {[}Matplotlib{]} ~ What is a backend? (++) ~ II

There are two types of backend:

\begin{itemize}
\item
  User interface, or ``interactive'', backends
\item
  Hardcopy, or ``non-interactive'', backends to make image files
\item
  e.g.~Agg (png), Cairo (svg), PDF (pdf), PS (eps, ps)
\item
  These are known as rendering engines and determine how your image is
  drawn
\end{itemize}

    {[}Matplotlib{]} ~ What is a backend? (++) ~ III

\begin{itemize}
\item
  Check which backend is being used with: matplotlib.get\_backend()
\item
  Default backend on ARCHER is Qt4Agg
\item
  Switch to a different backend with matplotlib.use(\ldots{})
\item
  Must issue command before importing matplotlib.pylot (or \%matplotlib)
\end{itemize}

    {[}Matplotlib{]} ~ What is interactive mode? (++) ~ I

Here we mean that a figure displays to screen as soon as you call either
plt.figure() or plt.plot().

\begin{itemize}
\itemsep1pt\parskip0pt\parsep0pt
\item
  Furthermore, the displayed figure does not prevent you from issuing
  commands in the IPython shell. This means you can update the figure
  and see the resulting changes immediately.
\end{itemize}

    {[}Matplotlib{]} ~ What is interactive mode? (++) ~ II

In contrast, in ``non-interactive'' mode, the figure will not display to
screen, unless you call show(). This is what happens when you create
figures in scripts.

\begin{itemize}
\itemsep1pt\parskip0pt\parsep0pt
\item
  If you show the figure, it ``blocks'' any further commands being
  issued in the shell until you have to closed the figure.
\end{itemize}

    {[}Matplotlib{]} ~ What is interactive mode? (++) ~ III

To confuse matters, an ``interactive'' backend does not guarantee your
figures will automatically display to screen. Matplotlib has a Boolean
variable in its configuration file (the matplotlibrc files, more of that
later) that sets the interactivity.

\begin{itemize}
\itemsep1pt\parskip0pt\parsep0pt
\item
  You can query this with: ~ matplotlib.is\_interactive()
\end{itemize}

    {[}Matplotlib{]} ~ What is interactive mode? (++) ~ IV

In most cases you don't need to worry about this.

The easiest way to ensure interactivity is to either:

\begin{itemize}
\item
  launch an IPython shell with the --matplotlib option or
\item
  to issue the magic command \%matplotlib within the IPython shell
  before issuing any other command.
\end{itemize}

    Notes

\begin{itemize}
\itemsep1pt\parskip0pt\parsep0pt
\item
  Actually, interactivity (plotting to screen) not guaranted by a
  particular backend. It is set by matplotlibrc variable
\end{itemize}

What is interactive mode? * Use of an interactive backend (see What is a
backend?) permits--but does not by itself require or ensure--plotting to
the screen. Whether and when plotting to the screen occurs, and whether
a script or shell session continues after a plot is drawn on the screen,
depends on the functions and methods that are called, and on a state
variable that determines whether matplotlib is in ``interactive mode''.
The default Boolean value is set by the matplotlibrc file, and may be
customized like any other configuration parameter (see Customizing
matplotlib). It may also be set via matplotlib.interactive(), and its
value may be queried via matplotlib.is\_interactive(). Turning
interactive mode on and off in the middle of a stream of plotting
commands, whether in a script or in a shell, is rarely needed and
potentially confusing, so in the following we will assume all plotting
is done with interactive mode either on or off.

\begin{itemize}
\item
  IPython: on ARCHER, Qt4Agg, macosx on laptop -\textgreater{} neither
  will plot to screen, need fig.show(), plt.show() by default. Remember,
  show is a ``blocking'' call, need to close window to continue. Also,
  fig.show() will show only what you've done to date. So need to execute
  all figure commands before fig.show() else you'll only see empty or
  part completed figure.
\item
  setting \%matplotlib on macosx, makes it interactive (still macosx
  backend).
\item
  setting \%matplotlib on ARCHER will also produce plots to screen BUT
  might have X server issues, so we will not do this\ldots{} (can do
  this if using your own laptop)
\item
  To make this more confusing, if using notebook, you have the
  interactive backend set by default so will not need to use plt.show()
  or fig.show()
\item
  A lot of documentation on the website and in the mailing lists refers
  to the ``backend'' and many new users are confused by this term.
  matplotlib targets many different use cases and output formats. Some
  people use matplotlib interactively from the python shell and have
  plotting windows pop up when they type commands. Some people embed
  matplotlib into graphical user interfaces like wxpython or pygtk to
  build rich applications. Others use matplotlib in batch scripts to
  generate postscript images from some numerical simulations, and still
  others in web application servers to dynamically serve up graphs.
\item
  To support all of these use cases, matplotlib can target different
  outputs, and each of these capabilities is called a backend; the
  ``frontend'' is the user facing code, i.e., the plotting code, whereas
  the ``backend'' does all the hard work behind-the-scenes to make the
  figure. There are two types of backends: user interface backends (for
  use in pygtk, wxpython, tkinter, qt4, or macosx; also referred to as
  ``interactive backends'') and hardcopy backends to make image files
  (PNG, SVG, PDF, PS; also referred to as ``non-interactive backends'')
\item
  To make things a little more customizable for graphical user
  interfaces, matplotlib separates the concept of the renderer (the
  thing that actually does the drawing) from the canvas (the place where
  the drawing goes). The canonical renderer for user interfaces is Agg
  which uses the Anti-Grain Geometry C++ library to make a raster
  (pixel) image of the figure. All of the user interfaces except macosx
  can be used with agg rendering, e.g., WXAgg, GTKAgg, QT4Agg, TkAgg. In
  addition, some of the user interfaces support other rendering engines.
  For example, with GTK, you can also select GDK rendering (backend GTK)
  or Cairo rendering (backend GTKCairo).
\item
  need to explain interactive mode and backend mode, e.g.~Agg
\item
  got interactive (shell/notebook) and non-interactive (script)
\item
  but also mean interactive in the sense that figure prints to screen
\item
  be default this is the case for shell, not for non-interactive
\item
  to switch it off in shell, set backend to non-interactive i.e.~save to
  file and can choose formats
\item
  to switch it on in non-interactive script, savefig FIRST then do
  fig.show() or plt.show() AFTER savefig (if you are saving figure)
\item
  cannot call non-interactive backend if: matplotlib.pyplot already been
  called, must be before this!
\item
  can call \%matplotlib at any point after you have set non-interactive
  backend to make shell interactive again
\end{itemize}

    {[}Matplotlib{]} ~ Plot customisations ~ I

There are many ways to customise a plot. Play with the following
properties.

    Notes 0. Should explain difference between certain routines in figure,
pylot, axes etc.? 1. Students to play about with plot properties. Give
them 15 - 20 mins to do this

    \begin{Verbatim}[commandchars=\\\{\}]
{\color{incolor}In [{\color{incolor}6}]:} \PY{c}{\PYZsh{} Ex: set the figure size and add a plot}
        \PY{n}{fig}\PY{o}{=}\PY{n}{plt}\PY{o}{.}\PY{n}{figure}\PY{p}{(}\PY{n}{figsize}\PY{o}{=}\PY{p}{(}\PY{l+m+mi}{4}\PY{p}{,}\PY{l+m+mi}{4}\PY{p}{)}\PY{p}{)}\PY{p}{;} 
        \PY{n}{plt}\PY{o}{.}\PY{n}{plot}\PY{p}{(}\PY{n}{x}\PY{p}{,}\PY{n}{y}\PY{p}{,}\PY{l+s}{\PYZsq{}}\PY{l+s}{c\PYZhy{}}\PY{l+s}{\PYZsq{}}\PY{p}{)}
\end{Verbatim}

            \begin{Verbatim}[commandchars=\\\{\}]
{\color{outcolor}Out[{\color{outcolor}6}]:} [<matplotlib.lines.Line2D at 0x109e666d0>]
\end{Verbatim}
        
    \begin{center}
    \adjustimage{max size={0.9\linewidth}{0.9\paperheight}}{L03_Matplotlib_files/L03_Matplotlib_31_1.png}
    \end{center}
    { \hspace*{\fill} \\}
    
    \begin{Verbatim}[commandchars=\\\{\}]
{\color{incolor}In [{\color{incolor} }]:} \PY{c}{\PYZsh{} Ex: linewidth, and }
        \PY{c}{\PYZsh{} linestyles: \PYZsq{}\PYZhy{}\PYZsq{}, \PYZsq{}.\PYZhy{}\PYZsq{}, \PYZsq{}:\PYZsq{}, \PYZsq{}\PYZhy{}\PYZhy{}\PYZsq{}}
        \PY{n}{plt}\PY{o}{.}\PY{n}{plot}\PY{p}{(}\PY{n}{x}\PY{p}{,}\PY{n}{y}\PY{p}{,}\PY{l+s}{\PYZsq{}}\PY{l+s}{k\PYZhy{}}\PY{l+s}{\PYZsq{}}\PY{p}{,}\PY{n}{linewidth}\PY{o}{=}\PY{l+m+mf}{2.0}\PY{p}{)}
\end{Verbatim}

    {[}Matplotlib{]} ~ Plot customisations ~ II

Play around with plot markers.

    \begin{Verbatim}[commandchars=\\\{\}]
{\color{incolor}In [{\color{incolor} }]:} \PY{c}{\PYZsh{} Ex: markers and their properties}
        \PY{c}{\PYZsh{} unfilled markers: \PYZsq{}.\PYZsq{},+\PYZsq{},\PYZsq{}x\PYZsq{},\PYZsq{}1\PYZsq{} to \PYZsq{}4\PYZsq{},\PYZsq{}|\PYZsq{}}
        \PY{n}{plt}\PY{o}{.}\PY{n}{plot}\PY{p}{(}\PY{n}{x}\PY{p}{,}\PY{n}{y}\PY{p}{,}\PY{l+s}{\PYZsq{}}\PY{l+s}{x}\PY{l+s}{\PYZsq{}}\PY{p}{,}\PY{n}{markersize}\PY{o}{=}\PY{l+m+mi}{10}\PY{p}{)}
\end{Verbatim}

    \begin{Verbatim}[commandchars=\\\{\}]
{\color{incolor}In [{\color{incolor} }]:} \PY{c}{\PYZsh{} filled markers: \PYZsq{}o\PYZsq{}, \PYZsq{}s\PYZsq{},\PYZsq{}*\PYZsq{},\PYZsq{}d\PYZsq{},\PYZsq{}\PYZgt{}\PYZsq{},\PYZsq{}\PYZca{}\PYZsq{},\PYZsq{}v\PYZsq{}, \PYZsq{}p\PYZsq{}, \PYZsq{}h\PYZsq{}}
        \PY{n}{plt}\PY{o}{.}\PY{n}{plot}\PY{p}{(}\PY{n}{x}\PY{p}{,}\PY{n}{y}\PY{p}{,}\PY{l+s}{\PYZsq{}}\PY{l+s}{8}\PY{l+s}{\PYZsq{}}\PY{p}{,}\PY{n}{markerfacecolor}\PY{o}{=}\PY{l+s}{\PYZsq{}}\PY{l+s}{None}\PY{l+s}{\PYZsq{}}\PY{p}{,}\PY{n}{markeredgecolor}\PY{o}{=}\PY{l+s}{\PYZsq{}}\PY{l+s}{g}\PY{l+s}{\PYZsq{}}\PY{p}{,}
                 \PY{n}{markersize}\PY{o}{=}\PY{l+m+mi}{10}\PY{p}{)}
\end{Verbatim}

    {[}Matplotlib{]} ~ Plot customisations ~ III

Set x-axis and y-axis limits

    \begin{Verbatim}[commandchars=\\\{\}]
{\color{incolor}In [{\color{incolor} }]:} \PY{c}{\PYZsh{} Ex: x,y, axis limits: }
        \PY{n}{plt}\PY{o}{.}\PY{n}{xlim}\PY{p}{(}\PY{p}{(}\PY{n}{xmax}\PY{o}{*}\PY{l+m+mf}{0.25}\PY{p}{,}\PY{n}{xmax}\PY{o}{*}\PY{l+m+mf}{0.75}\PY{p}{)}\PY{p}{)}\PY{p}{;}
        \PY{n}{plt}\PY{o}{.}\PY{n}{ylim}\PY{p}{(}\PY{p}{(}\PY{n}{np}\PY{o}{.}\PY{n}{cos}\PY{p}{(}\PY{n}{xmin}\PY{o}{*}\PY{l+m+mf}{0.25}\PY{p}{)}\PY{p}{,}\PY{n}{np}\PY{o}{.}\PY{n}{cos}\PY{p}{(}\PY{n}{xmax}\PY{o}{*}\PY{l+m+mf}{0.75}\PY{p}{)}\PY{p}{)}\PY{p}{)}\PY{p}{;}
        \PY{n}{plt}\PY{o}{.}\PY{n}{plot}\PY{p}{(}\PY{n}{x}\PY{p}{,}\PY{n}{y}\PY{p}{,}\PY{l+s}{\PYZsq{}}\PY{l+s}{mo\PYZhy{}}\PY{l+s}{\PYZsq{}}\PY{p}{)}
\end{Verbatim}

    {[}Matplotlib{]} ~ Plot customisations ~ IV

Adjust title font properties

    \begin{Verbatim}[commandchars=\\\{\}]
{\color{incolor}In [{\color{incolor} }]:} \PY{c}{\PYZsh{} Ex: title placement and font properties}
        \PY{n}{plt}\PY{o}{.}\PY{n}{plot}\PY{p}{(}\PY{n}{x}\PY{p}{,}\PY{n}{y}\PY{p}{,}\PY{l+s}{\PYZsq{}}\PY{l+s}{x}\PY{l+s}{\PYZsq{}}\PY{p}{)}
        \PY{n}{plt}\PY{o}{.}\PY{n}{suptitle}\PY{p}{(}\PY{l+s}{\PYZsq{}}\PY{l+s}{A Centered Title}\PY{l+s}{\PYZsq{}}\PY{p}{,} \PY{n}{fontsize}\PY{o}{=}\PY{l+m+mi}{20}\PY{p}{)}
        \PY{c}{\PYZsh{} loc: center, left, right}
        \PY{c}{\PYZsh{} verticalalignment: center, top, bottom, baseline}
        \PY{n}{plt}\PY{o}{.}\PY{n}{title}\PY{p}{(}\PY{l+s}{\PYZsq{}}\PY{l+s}{A Placed Title}\PY{l+s}{\PYZsq{}}\PY{p}{,} \PY{n}{loc}\PY{o}{=}\PY{l+s}{\PYZsq{}}\PY{l+s}{left}\PY{l+s}{\PYZsq{}}\PY{p}{,} \PY{n}{verticalalignment}\PY{o}{=}\PY{l+s}{\PYZsq{}}\PY{l+s}{top}\PY{l+s}{\PYZsq{}} \PY{p}{)}
\end{Verbatim}

    {[}Matplotlib{]} ~ Plot customisations ~ V

Add tickmarks

    \begin{Verbatim}[commandchars=\\\{\}]
{\color{incolor}In [{\color{incolor} }]:} \PY{c}{\PYZsh{} Ex: tick marks }
        \PY{n}{fig}\PY{o}{=}\PY{n}{plt}\PY{o}{.}\PY{n}{figure}\PY{p}{(}\PY{n}{figsize}\PY{o}{=}\PY{p}{(}\PY{l+m+mi}{4}\PY{p}{,}\PY{l+m+mf}{3.5}\PY{p}{)}\PY{p}{)}\PY{p}{;} \PY{n}{plt}\PY{o}{.}\PY{n}{plot}\PY{p}{(}\PY{n}{x}\PY{p}{,}\PY{n}{y}\PY{p}{,}\PY{l+s}{\PYZsq{}}\PY{l+s}{x}\PY{l+s}{\PYZsq{}}\PY{p}{)}\PY{p}{;} 
        \PY{n}{nticks} \PY{o}{=} \PY{l+m+mi}{5}\PY{p}{;}
        \PY{n}{tickpos} \PY{o}{=} \PY{n}{np}\PY{o}{.}\PY{n}{linspace}\PY{p}{(}\PY{n}{xmin}\PY{p}{,}\PY{n}{xmax}\PY{p}{,}\PY{n}{nticks}\PY{p}{)}\PY{p}{;}
        \PY{n}{labels} \PY{o}{=} \PY{n}{np}\PY{o}{.}\PY{n}{repeat}\PY{p}{(}\PY{p}{[}\PY{l+s}{\PYZsq{}}\PY{l+s}{tick}\PY{l+s}{\PYZsq{}}\PY{p}{]}\PY{p}{,}\PY{n}{nticks}\PY{p}{)}\PY{p}{;}
        \PY{n}{plt}\PY{o}{.}\PY{n}{xticks}\PY{p}{(}\PY{n}{tickpos}\PY{p}{,} \PY{n}{labels}\PY{p}{,} \PY{n}{rotation}\PY{o}{=}\PY{l+s}{\PYZsq{}}\PY{l+s}{vertical}\PY{l+s}{\PYZsq{}}\PY{p}{)}\PY{p}{;}
\end{Verbatim}

    {[}Matplotlib{]} ~ Plot customisations ~ VI

Add annotations

    \begin{Verbatim}[commandchars=\\\{\}]
{\color{incolor}In [{\color{incolor} }]:} \PY{c}{\PYZsh{} Ex++: arrows and annotations}
        \PY{n}{plt}\PY{o}{.}\PY{n}{plot}\PY{p}{(}\PY{n}{x}\PY{p}{,}\PY{n}{y}\PY{p}{,}\PY{l+s}{\PYZsq{}}\PY{l+s}{x}\PY{l+s}{\PYZsq{}}\PY{p}{)}\PY{p}{;}
        \PY{n}{atext}\PY{o}{=}\PY{l+s}{\PYZsq{}}\PY{l+s}{annotate this}\PY{l+s}{\PYZsq{}}\PY{p}{;} \PY{n}{arrowtip}\PY{o}{=}\PY{p}{(}\PY{l+m+mf}{1.5}\PY{p}{,}\PY{l+m+mf}{0.5}\PY{p}{)}\PY{p}{;} \PY{n}{textloc}\PY{o}{=}\PY{p}{(}\PY{l+m+mi}{3}\PY{p}{,} \PY{l+m+mf}{0.75}\PY{p}{)}\PY{p}{;}
        \PY{n}{plt}\PY{o}{.}\PY{n}{annotate}\PY{p}{(}\PY{n}{atext}\PY{p}{,} \PY{n}{xy}\PY{o}{=}\PY{n}{arrowtip}\PY{p}{,} \PY{n}{xytext}\PY{o}{=}\PY{n}{textloc}\PY{p}{,}
                    \PY{n}{arrowprops}\PY{o}{=}\PY{n+nb}{dict}\PY{p}{(}\PY{n}{facecolor}\PY{o}{=}\PY{l+s}{\PYZsq{}}\PY{l+s}{black}\PY{l+s}{\PYZsq{}}\PY{p}{,} \PY{n}{shrink}\PY{o}{=}\PY{l+m+mf}{0.01}\PY{p}{)}\PY{p}{,}\PY{p}{)}
\end{Verbatim}

    {[}Matplotlib{]} ~ Subplots ~

\begin{itemize}
\item
  There can be multiple plots, or subplots, within a figure
\item
  Use subplot(nrows, ncols, plot number) to place plots on a regular
  grid
\item
  The most recently created subplot is the current plot
\end{itemize}

    {[}Matplotlib{]} ~ Subplots ~ II

\begin{itemize}
\itemsep1pt\parskip0pt\parsep0pt
\item
  Can move between subplots by creating each subplot with a ``handle''
  for each axes
\end{itemize}

    \begin{Verbatim}[commandchars=\\\{\}]
{\color{incolor}In [{\color{incolor} }]:} \PY{p}{(}\PY{n}{fig}\PY{p}{,} \PY{n}{axes}\PY{p}{)} \PY{o}{=} \PY{n}{plt}\PY{o}{.}\PY{n}{subplots}\PY{p}{(}\PY{n}{nrows}\PY{o}{=}\PY{l+m+mi}{2}\PY{p}{,} \PY{n}{ncols}\PY{o}{=}\PY{l+m+mi}{2}\PY{p}{)}\PY{p}{;}
        \PY{n}{axes}\PY{o}{.}\PY{n}{size}
        
        \PY{n}{axes}\PY{p}{[}\PY{l+m+mi}{0}\PY{p}{,}\PY{l+m+mi}{0}\PY{p}{]}\PY{o}{.}\PY{n}{plot}\PY{p}{(}\PY{n}{x}\PY{p}{,}\PY{n}{y}\PY{p}{,}\PY{l+s}{\PYZsq{}}\PY{l+s}{g\PYZhy{}}\PY{l+s}{\PYZsq{}}\PY{p}{)}\PY{p}{;}
        \PY{n}{axes}\PY{p}{[}\PY{l+m+mi}{1}\PY{p}{,}\PY{l+m+mi}{1}\PY{p}{]}\PY{o}{.}\PY{n}{plot}\PY{p}{(}\PY{n}{x}\PY{p}{,}\PY{n}{y}\PY{p}{,}\PY{l+s}{\PYZsq{}}\PY{l+s}{r\PYZhy{}}\PY{l+s}{\PYZsq{}}\PY{p}{)}\PY{p}{;}
\end{Verbatim}

    {[}Matplotlib{]} ~ subplot2grid (++)

\begin{itemize}
\item
  For more control over subplot layout, use subplot2grid
\item
  Subplots can span more than one row or column
\end{itemize}

 

    \begin{Verbatim}[commandchars=\\\{\}]
{\color{incolor}In [{\color{incolor} }]:} \PY{c}{\PYZsh{} Ex++: subplot2grid(shape, loc, rowspan=1, colspan=1)}
        \PY{n}{fig} \PY{o}{=} \PY{n}{plt}\PY{o}{.}\PY{n}{figure}\PY{p}{(}\PY{p}{)}
        \PY{n}{ax1} \PY{o}{=} \PY{n}{plt}\PY{o}{.}\PY{n}{subplot2grid}\PY{p}{(}\PY{p}{(}\PY{l+m+mi}{3}\PY{p}{,} \PY{l+m+mi}{3}\PY{p}{)}\PY{p}{,} \PY{p}{(}\PY{l+m+mi}{0}\PY{p}{,} \PY{l+m+mi}{0}\PY{p}{)}\PY{p}{)}\PY{p}{;} \PY{n}{ax1}\PY{o}{.}\PY{n}{plot}\PY{p}{(}\PY{n}{x}\PY{p}{,}\PY{n}{y}\PY{p}{,}\PY{l+s}{\PYZsq{}}\PY{l+s}{r\PYZhy{}}\PY{l+s}{\PYZsq{}}\PY{p}{)}\PY{p}{;}
        \PY{n}{ax2} \PY{o}{=} \PY{n}{plt}\PY{o}{.}\PY{n}{subplot2grid}\PY{p}{(}\PY{p}{(}\PY{l+m+mi}{3}\PY{p}{,} \PY{l+m+mi}{3}\PY{p}{)}\PY{p}{,} \PY{p}{(}\PY{l+m+mi}{0}\PY{p}{,} \PY{l+m+mi}{1}\PY{p}{)}\PY{p}{,} \PY{n}{colspan}\PY{o}{=}\PY{l+m+mi}{2}\PY{p}{)}\PY{p}{;} \PY{n}{ax2}\PY{o}{.}\PY{n}{plot}\PY{p}{(}\PY{n}{x}\PY{p}{,}\PY{n}{y}\PY{p}{,}\PY{l+s}{\PYZsq{}}\PY{l+s}{g\PYZhy{}}\PY{l+s}{\PYZsq{}}\PY{p}{)}\PY{p}{;}
        \PY{n}{ax3} \PY{o}{=} \PY{n}{plt}\PY{o}{.}\PY{n}{subplot2grid}\PY{p}{(}\PY{p}{(}\PY{l+m+mi}{3}\PY{p}{,} \PY{l+m+mi}{3}\PY{p}{)}\PY{p}{,} \PY{p}{(}\PY{l+m+mi}{1}\PY{p}{,} \PY{l+m+mi}{0}\PY{p}{)}\PY{p}{,} \PY{n}{colspan}\PY{o}{=}\PY{l+m+mi}{2}\PY{p}{,} \PY{n}{rowspan}\PY{o}{=}\PY{l+m+mi}{2}\PY{p}{)}\PY{p}{;} \PY{n}{ax3}\PY{o}{.}\PY{n}{plot}\PY{p}{(}\PY{n}{x}\PY{p}{,}\PY{n}{y}\PY{p}{,}\PY{l+s}{\PYZsq{}}\PY{l+s}{b\PYZhy{}}\PY{l+s}{\PYZsq{}}\PY{p}{)}\PY{p}{;}
        \PY{n}{ax4} \PY{o}{=} \PY{n}{plt}\PY{o}{.}\PY{n}{subplot2grid}\PY{p}{(}\PY{p}{(}\PY{l+m+mi}{3}\PY{p}{,} \PY{l+m+mi}{3}\PY{p}{)}\PY{p}{,} \PY{p}{(}\PY{l+m+mi}{1}\PY{p}{,} \PY{l+m+mi}{2}\PY{p}{)}\PY{p}{,} \PY{n}{rowspan}\PY{o}{=}\PY{l+m+mi}{2}\PY{p}{)}\PY{p}{;} \PY{n}{ax4}\PY{o}{.}\PY{n}{plot}\PY{p}{(}\PY{n}{x}\PY{p}{,}\PY{n}{y}\PY{p}{,}\PY{l+s}{\PYZsq{}}\PY{l+s}{c\PYZhy{}}\PY{l+s}{\PYZsq{}}\PY{p}{)}\PY{p}{;}
\end{Verbatim}

    {[}Matplotlib{]} ~ Customise some subplots

\begin{itemize}
\itemsep1pt\parskip0pt\parsep0pt
\item
  Go back to the Matplotlib exercise and create multiple customised
  plots (pages 8 - 11)
\end{itemize}

    Note 1. Students to create a customised plot with multiple plots, give
them 15 - 20 mins to do this

    {[}Matplotlib{]} ~ Other type of plots

http://matplotlib.org/gallery.html

 

    Notes * Only just skimmed the surface of what is possible * the above
slide looks terrible but it looks reasonable in a slideshow\ldots{}
promise\ldots{} * xkcd figure in the bottom right corner\ldots{}

    {[}Matplotlib{]} ~ Advanced : animation

Can even create animations (from Nicolas P. Rougier,
https://github.com/rougier)

    \begin{Verbatim}[commandchars=\\\{\}]
{\color{incolor}In [{\color{incolor} }]:} \PY{o}{\PYZpc{}}\PY{k}{reset}
        \PY{k+kn}{import} \PY{n+nn}{warnings}
        \PY{n}{warnings}\PY{o}{.}\PY{n}{filterwarnings}\PY{p}{(}\PY{l+s}{\PYZsq{}}\PY{l+s}{ignore}\PY{l+s}{\PYZsq{}}\PY{p}{)}
\end{Verbatim}

    \begin{Verbatim}[commandchars=\\\{\}]
{\color{incolor}In [{\color{incolor} }]:} \PY{c}{\PYZsh{}from matplotlib import use}
        \PY{c}{\PYZsh{}\PYZsh{} animation doesn\PYZsq{}t work with macosx backend!}
        \PY{c}{\PYZsh{}use(\PYZdq{}nbagg\PYZdq{})}
        \PY{c}{\PYZsh{}import earthquakes;}
\end{Verbatim}

    ~

    ~

    {[}Matplotlib{]} ~ Images for publication ~ I

\begin{itemize}
\item
  Matplotlib uses matplotlibrc configuration files to customize and set
  defaults for all kinds of properties (rc settings, rc parameters)
\item
  Creating a custom matplotlibrc file in your local directory will
  override the default matplotlibrc file, and limit changes to that
  directory.
\end{itemize}

 From Damon McDougall: http://bit.ly/1jIuuU0

    {[}Matplotlib{]} ~ Images for publication ~ II

\begin{itemize}
\item
  You will most likely want different settings for each journal
\item
  useful to keep a different matplotlibrc file for each journal
\end{itemize}

    {[}Matplotlib{]} ~ matplotlibrc : import rc file

\begin{itemize}
\itemsep1pt\parskip0pt\parsep0pt
\item
  import a particular settings file with:
\end{itemize}

~ ~ from matplotlib import rc\_file ~ ~
rc\_file(`/path/to/my/matplotlibrc')

    {[}Matplotlib{]} ~ matplotlibrc : settings ~ I

axes.labelsize : 9.0 \# fontsize of the x any y labels xtick.labelsize :
9.0 \# fontsize of the tick labels ytick.labelsize : 9.0 \# fontsize of
the tick labels legend.fontsize : 9.0 \# fontsize in legend font.family
: serif font.serif : Computer Modern Roman Marker size :
lines.markersize : 3 text.usetex : True

\begin{itemize}
\itemsep1pt\parskip0pt\parsep0pt
\item
  Last line means use TeX to format all text (only available with Agg,
  PS, PDF backends)
\end{itemize}

    Notes * set font properties etc.

    {[}Matplotlib{]} ~ matplotlibrc : settings ~ II

Here are some settings you can use to create a nice figure ratio

WIDTH = 500.0 \# Figure width in pt (usually from LaTeX) FACTOR = 0.45
\# Fraction of the width you'd like the figure to use widthpt = WIDTH *
FACTOR inperpt = 1.0 / 72.27 \# use the Golden ratio because it looks
good golden\_ratio = (np.sqrt(5) - 1.0) / 2.0 widthin = widthpt *
inperpt heightin = widthin * golden\_ratio figdims = {[}widthin,
heightin{]} \# Dimensions as list fig = plt.figure(figsize=figdims)

    Notes * students will see what `nice' means when they do the exercise

    {[}Matplotlib{]} ~ Include images in $\LaTeX$

When you include the figure in the LaTeX source you should specify the
scale factor as the width:

$\begin{figure}   \includegraphics[width=0.45\textwidth]{figure.pdf} \end{figure}$

Complete the Matplotlib exercise and create a publication standard image
(pages 12 - 13)

    Note 1. Students to create publication image. Give students 15 - 20 mins
to do this

    Notes * some notes

    {[}Matplotlib{]} ~ Summary ~ I

\begin{itemize}
\item
  Simple, interactive plotting
\item
  integration with NumPy allows you to easily read data
\item
  Plotting syntax is simple and concise
\item
  Complex plotting types also available
\item
  Can start from code for simple plots
\item
  Many examples available online
\end{itemize}

    {[}Matplotlib{]} ~ Summary ~ II

\begin{itemize}
\item
  Producing publication-ready images is relatively simple
\item
  Easily customised for different scenarios
\item
  The more you use matplotlib, the more you get out of it!
\item
  Other packages
\item
  Bokeh : interactive visualisation library.
\end{itemize}


    % Add a bibliography block to the postdoc
    
    
    
    \end{document}
