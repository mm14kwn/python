
% Default to the notebook output style

    


% Inherit from the specified cell style.




    
\documentclass{article}

    
    
    \usepackage{graphicx} % Used to insert images
    \usepackage{adjustbox} % Used to constrain images to a maximum size 
    \usepackage{color} % Allow colors to be defined
    \usepackage{enumerate} % Needed for markdown enumerations to work
    \usepackage{geometry} % Used to adjust the document margins
    \usepackage{amsmath} % Equations
    \usepackage{amssymb} % Equations
    \usepackage{eurosym} % defines \euro
    \usepackage[mathletters]{ucs} % Extended unicode (utf-8) support
    \usepackage[utf8x]{inputenc} % Allow utf-8 characters in the tex document
    \usepackage{fancyvrb} % verbatim replacement that allows latex
    \usepackage{grffile} % extends the file name processing of package graphics 
                         % to support a larger range 
    % The hyperref package gives us a pdf with properly built
    % internal navigation ('pdf bookmarks' for the table of contents,
    % internal cross-reference links, web links for URLs, etc.)
    \usepackage{hyperref}
    \usepackage{longtable} % longtable support required by pandoc >1.10
    \usepackage{booktabs}  % table support for pandoc > 1.12.2
    

    
    
    \definecolor{orange}{cmyk}{0,0.4,0.8,0.2}
    \definecolor{darkorange}{rgb}{.71,0.21,0.01}
    \definecolor{darkgreen}{rgb}{.12,.54,.11}
    \definecolor{myteal}{rgb}{.26, .44, .56}
    \definecolor{gray}{gray}{0.45}
    \definecolor{lightgray}{gray}{.95}
    \definecolor{mediumgray}{gray}{.8}
    \definecolor{inputbackground}{rgb}{.95, .95, .85}
    \definecolor{outputbackground}{rgb}{.95, .95, .95}
    \definecolor{traceback}{rgb}{1, .95, .95}
    % ansi colors
    \definecolor{red}{rgb}{.6,0,0}
    \definecolor{green}{rgb}{0,.65,0}
    \definecolor{brown}{rgb}{0.6,0.6,0}
    \definecolor{blue}{rgb}{0,.145,.698}
    \definecolor{purple}{rgb}{.698,.145,.698}
    \definecolor{cyan}{rgb}{0,.698,.698}
    \definecolor{lightgray}{gray}{0.5}
    
    % bright ansi colors
    \definecolor{darkgray}{gray}{0.25}
    \definecolor{lightred}{rgb}{1.0,0.39,0.28}
    \definecolor{lightgreen}{rgb}{0.48,0.99,0.0}
    \definecolor{lightblue}{rgb}{0.53,0.81,0.92}
    \definecolor{lightpurple}{rgb}{0.87,0.63,0.87}
    \definecolor{lightcyan}{rgb}{0.5,1.0,0.83}
    
    % commands and environments needed by pandoc snippets
    % extracted from the output of `pandoc -s`
    \providecommand{\tightlist}{%
      \setlength{\itemsep}{0pt}\setlength{\parskip}{0pt}}
    \DefineVerbatimEnvironment{Highlighting}{Verbatim}{commandchars=\\\{\}}
    % Add ',fontsize=\small' for more characters per line
    \newenvironment{Shaded}{}{}
    \newcommand{\KeywordTok}[1]{\textcolor[rgb]{0.00,0.44,0.13}{\textbf{{#1}}}}
    \newcommand{\DataTypeTok}[1]{\textcolor[rgb]{0.56,0.13,0.00}{{#1}}}
    \newcommand{\DecValTok}[1]{\textcolor[rgb]{0.25,0.63,0.44}{{#1}}}
    \newcommand{\BaseNTok}[1]{\textcolor[rgb]{0.25,0.63,0.44}{{#1}}}
    \newcommand{\FloatTok}[1]{\textcolor[rgb]{0.25,0.63,0.44}{{#1}}}
    \newcommand{\CharTok}[1]{\textcolor[rgb]{0.25,0.44,0.63}{{#1}}}
    \newcommand{\StringTok}[1]{\textcolor[rgb]{0.25,0.44,0.63}{{#1}}}
    \newcommand{\CommentTok}[1]{\textcolor[rgb]{0.38,0.63,0.69}{\textit{{#1}}}}
    \newcommand{\OtherTok}[1]{\textcolor[rgb]{0.00,0.44,0.13}{{#1}}}
    \newcommand{\AlertTok}[1]{\textcolor[rgb]{1.00,0.00,0.00}{\textbf{{#1}}}}
    \newcommand{\FunctionTok}[1]{\textcolor[rgb]{0.02,0.16,0.49}{{#1}}}
    \newcommand{\RegionMarkerTok}[1]{{#1}}
    \newcommand{\ErrorTok}[1]{\textcolor[rgb]{1.00,0.00,0.00}{\textbf{{#1}}}}
    \newcommand{\NormalTok}[1]{{#1}}
    
    % Define a nice break command that doesn't care if a line doesn't already
    % exist.
    \def\br{\hspace*{\fill} \\* }
    % Math Jax compatability definitions
    \def\gt{>}
    \def\lt{<}
    % Document parameters
    \title{L02\_NumPy}
    
    
    

    % Pygments definitions
    
\makeatletter
\def\PY@reset{\let\PY@it=\relax \let\PY@bf=\relax%
    \let\PY@ul=\relax \let\PY@tc=\relax%
    \let\PY@bc=\relax \let\PY@ff=\relax}
\def\PY@tok#1{\csname PY@tok@#1\endcsname}
\def\PY@toks#1+{\ifx\relax#1\empty\else%
    \PY@tok{#1}\expandafter\PY@toks\fi}
\def\PY@do#1{\PY@bc{\PY@tc{\PY@ul{%
    \PY@it{\PY@bf{\PY@ff{#1}}}}}}}
\def\PY#1#2{\PY@reset\PY@toks#1+\relax+\PY@do{#2}}

\expandafter\def\csname PY@tok@gd\endcsname{\def\PY@tc##1{\textcolor[rgb]{0.63,0.00,0.00}{##1}}}
\expandafter\def\csname PY@tok@gu\endcsname{\let\PY@bf=\textbf\def\PY@tc##1{\textcolor[rgb]{0.50,0.00,0.50}{##1}}}
\expandafter\def\csname PY@tok@gt\endcsname{\def\PY@tc##1{\textcolor[rgb]{0.00,0.27,0.87}{##1}}}
\expandafter\def\csname PY@tok@gs\endcsname{\let\PY@bf=\textbf}
\expandafter\def\csname PY@tok@gr\endcsname{\def\PY@tc##1{\textcolor[rgb]{1.00,0.00,0.00}{##1}}}
\expandafter\def\csname PY@tok@cm\endcsname{\let\PY@it=\textit\def\PY@tc##1{\textcolor[rgb]{0.25,0.50,0.50}{##1}}}
\expandafter\def\csname PY@tok@vg\endcsname{\def\PY@tc##1{\textcolor[rgb]{0.10,0.09,0.49}{##1}}}
\expandafter\def\csname PY@tok@m\endcsname{\def\PY@tc##1{\textcolor[rgb]{0.40,0.40,0.40}{##1}}}
\expandafter\def\csname PY@tok@mh\endcsname{\def\PY@tc##1{\textcolor[rgb]{0.40,0.40,0.40}{##1}}}
\expandafter\def\csname PY@tok@go\endcsname{\def\PY@tc##1{\textcolor[rgb]{0.53,0.53,0.53}{##1}}}
\expandafter\def\csname PY@tok@ge\endcsname{\let\PY@it=\textit}
\expandafter\def\csname PY@tok@vc\endcsname{\def\PY@tc##1{\textcolor[rgb]{0.10,0.09,0.49}{##1}}}
\expandafter\def\csname PY@tok@il\endcsname{\def\PY@tc##1{\textcolor[rgb]{0.40,0.40,0.40}{##1}}}
\expandafter\def\csname PY@tok@cs\endcsname{\let\PY@it=\textit\def\PY@tc##1{\textcolor[rgb]{0.25,0.50,0.50}{##1}}}
\expandafter\def\csname PY@tok@cp\endcsname{\def\PY@tc##1{\textcolor[rgb]{0.74,0.48,0.00}{##1}}}
\expandafter\def\csname PY@tok@gi\endcsname{\def\PY@tc##1{\textcolor[rgb]{0.00,0.63,0.00}{##1}}}
\expandafter\def\csname PY@tok@gh\endcsname{\let\PY@bf=\textbf\def\PY@tc##1{\textcolor[rgb]{0.00,0.00,0.50}{##1}}}
\expandafter\def\csname PY@tok@ni\endcsname{\let\PY@bf=\textbf\def\PY@tc##1{\textcolor[rgb]{0.60,0.60,0.60}{##1}}}
\expandafter\def\csname PY@tok@nl\endcsname{\def\PY@tc##1{\textcolor[rgb]{0.63,0.63,0.00}{##1}}}
\expandafter\def\csname PY@tok@nn\endcsname{\let\PY@bf=\textbf\def\PY@tc##1{\textcolor[rgb]{0.00,0.00,1.00}{##1}}}
\expandafter\def\csname PY@tok@no\endcsname{\def\PY@tc##1{\textcolor[rgb]{0.53,0.00,0.00}{##1}}}
\expandafter\def\csname PY@tok@na\endcsname{\def\PY@tc##1{\textcolor[rgb]{0.49,0.56,0.16}{##1}}}
\expandafter\def\csname PY@tok@nb\endcsname{\def\PY@tc##1{\textcolor[rgb]{0.00,0.50,0.00}{##1}}}
\expandafter\def\csname PY@tok@nc\endcsname{\let\PY@bf=\textbf\def\PY@tc##1{\textcolor[rgb]{0.00,0.00,1.00}{##1}}}
\expandafter\def\csname PY@tok@nd\endcsname{\def\PY@tc##1{\textcolor[rgb]{0.67,0.13,1.00}{##1}}}
\expandafter\def\csname PY@tok@ne\endcsname{\let\PY@bf=\textbf\def\PY@tc##1{\textcolor[rgb]{0.82,0.25,0.23}{##1}}}
\expandafter\def\csname PY@tok@nf\endcsname{\def\PY@tc##1{\textcolor[rgb]{0.00,0.00,1.00}{##1}}}
\expandafter\def\csname PY@tok@si\endcsname{\let\PY@bf=\textbf\def\PY@tc##1{\textcolor[rgb]{0.73,0.40,0.53}{##1}}}
\expandafter\def\csname PY@tok@s2\endcsname{\def\PY@tc##1{\textcolor[rgb]{0.73,0.13,0.13}{##1}}}
\expandafter\def\csname PY@tok@vi\endcsname{\def\PY@tc##1{\textcolor[rgb]{0.10,0.09,0.49}{##1}}}
\expandafter\def\csname PY@tok@nt\endcsname{\let\PY@bf=\textbf\def\PY@tc##1{\textcolor[rgb]{0.00,0.50,0.00}{##1}}}
\expandafter\def\csname PY@tok@nv\endcsname{\def\PY@tc##1{\textcolor[rgb]{0.10,0.09,0.49}{##1}}}
\expandafter\def\csname PY@tok@s1\endcsname{\def\PY@tc##1{\textcolor[rgb]{0.73,0.13,0.13}{##1}}}
\expandafter\def\csname PY@tok@kd\endcsname{\let\PY@bf=\textbf\def\PY@tc##1{\textcolor[rgb]{0.00,0.50,0.00}{##1}}}
\expandafter\def\csname PY@tok@sh\endcsname{\def\PY@tc##1{\textcolor[rgb]{0.73,0.13,0.13}{##1}}}
\expandafter\def\csname PY@tok@sc\endcsname{\def\PY@tc##1{\textcolor[rgb]{0.73,0.13,0.13}{##1}}}
\expandafter\def\csname PY@tok@sx\endcsname{\def\PY@tc##1{\textcolor[rgb]{0.00,0.50,0.00}{##1}}}
\expandafter\def\csname PY@tok@bp\endcsname{\def\PY@tc##1{\textcolor[rgb]{0.00,0.50,0.00}{##1}}}
\expandafter\def\csname PY@tok@c1\endcsname{\let\PY@it=\textit\def\PY@tc##1{\textcolor[rgb]{0.25,0.50,0.50}{##1}}}
\expandafter\def\csname PY@tok@kc\endcsname{\let\PY@bf=\textbf\def\PY@tc##1{\textcolor[rgb]{0.00,0.50,0.00}{##1}}}
\expandafter\def\csname PY@tok@c\endcsname{\let\PY@it=\textit\def\PY@tc##1{\textcolor[rgb]{0.25,0.50,0.50}{##1}}}
\expandafter\def\csname PY@tok@mf\endcsname{\def\PY@tc##1{\textcolor[rgb]{0.40,0.40,0.40}{##1}}}
\expandafter\def\csname PY@tok@err\endcsname{\def\PY@bc##1{\setlength{\fboxsep}{0pt}\fcolorbox[rgb]{1.00,0.00,0.00}{1,1,1}{\strut ##1}}}
\expandafter\def\csname PY@tok@mb\endcsname{\def\PY@tc##1{\textcolor[rgb]{0.40,0.40,0.40}{##1}}}
\expandafter\def\csname PY@tok@ss\endcsname{\def\PY@tc##1{\textcolor[rgb]{0.10,0.09,0.49}{##1}}}
\expandafter\def\csname PY@tok@sr\endcsname{\def\PY@tc##1{\textcolor[rgb]{0.73,0.40,0.53}{##1}}}
\expandafter\def\csname PY@tok@mo\endcsname{\def\PY@tc##1{\textcolor[rgb]{0.40,0.40,0.40}{##1}}}
\expandafter\def\csname PY@tok@kn\endcsname{\let\PY@bf=\textbf\def\PY@tc##1{\textcolor[rgb]{0.00,0.50,0.00}{##1}}}
\expandafter\def\csname PY@tok@mi\endcsname{\def\PY@tc##1{\textcolor[rgb]{0.40,0.40,0.40}{##1}}}
\expandafter\def\csname PY@tok@gp\endcsname{\let\PY@bf=\textbf\def\PY@tc##1{\textcolor[rgb]{0.00,0.00,0.50}{##1}}}
\expandafter\def\csname PY@tok@o\endcsname{\def\PY@tc##1{\textcolor[rgb]{0.40,0.40,0.40}{##1}}}
\expandafter\def\csname PY@tok@kr\endcsname{\let\PY@bf=\textbf\def\PY@tc##1{\textcolor[rgb]{0.00,0.50,0.00}{##1}}}
\expandafter\def\csname PY@tok@s\endcsname{\def\PY@tc##1{\textcolor[rgb]{0.73,0.13,0.13}{##1}}}
\expandafter\def\csname PY@tok@kp\endcsname{\def\PY@tc##1{\textcolor[rgb]{0.00,0.50,0.00}{##1}}}
\expandafter\def\csname PY@tok@w\endcsname{\def\PY@tc##1{\textcolor[rgb]{0.73,0.73,0.73}{##1}}}
\expandafter\def\csname PY@tok@kt\endcsname{\def\PY@tc##1{\textcolor[rgb]{0.69,0.00,0.25}{##1}}}
\expandafter\def\csname PY@tok@ow\endcsname{\let\PY@bf=\textbf\def\PY@tc##1{\textcolor[rgb]{0.67,0.13,1.00}{##1}}}
\expandafter\def\csname PY@tok@sb\endcsname{\def\PY@tc##1{\textcolor[rgb]{0.73,0.13,0.13}{##1}}}
\expandafter\def\csname PY@tok@k\endcsname{\let\PY@bf=\textbf\def\PY@tc##1{\textcolor[rgb]{0.00,0.50,0.00}{##1}}}
\expandafter\def\csname PY@tok@se\endcsname{\let\PY@bf=\textbf\def\PY@tc##1{\textcolor[rgb]{0.73,0.40,0.13}{##1}}}
\expandafter\def\csname PY@tok@sd\endcsname{\let\PY@it=\textit\def\PY@tc##1{\textcolor[rgb]{0.73,0.13,0.13}{##1}}}

\def\PYZbs{\char`\\}
\def\PYZus{\char`\_}
\def\PYZob{\char`\{}
\def\PYZcb{\char`\}}
\def\PYZca{\char`\^}
\def\PYZam{\char`\&}
\def\PYZlt{\char`\<}
\def\PYZgt{\char`\>}
\def\PYZsh{\char`\#}
\def\PYZpc{\char`\%}
\def\PYZdl{\char`\$}
\def\PYZhy{\char`\-}
\def\PYZsq{\char`\'}
\def\PYZdq{\char`\"}
\def\PYZti{\char`\~}
% for compatibility with earlier versions
\def\PYZat{@}
\def\PYZlb{[}
\def\PYZrb{]}
\makeatother


    % Exact colors from NB
    \definecolor{incolor}{rgb}{0.0, 0.0, 0.5}
    \definecolor{outcolor}{rgb}{0.545, 0.0, 0.0}



    
    % Prevent overflowing lines due to hard-to-break entities
    \sloppy 
    % Setup hyperref package
    \hypersetup{
      breaklinks=true,  % so long urls are correctly broken across lines
      colorlinks=true,
      urlcolor=blue,
      linkcolor=darkorange,
      citecolor=darkgreen,
      }
    % Slightly bigger margins than the latex defaults
    
    \geometry{verbose,tmargin=1in,bmargin=1in,lmargin=1in,rmargin=1in}
    
    

    \begin{document}
    
    
    \maketitle
    
    

    
    \begin{Verbatim}[commandchars=\\\{\}]
{\color{incolor}In [{\color{incolor} }]:} \PY{c}{\PYZsh{} suppress deprecated warnings}
        \PY{k+kn}{import} \PY{n+nn}{warnings}
        \PY{n}{warnings}\PY{o}{.}\PY{n}{filterwarnings}\PY{p}{(}\PY{l+s}{\PYZsq{}}\PY{l+s}{ignore}\PY{l+s}{\PYZsq{}}\PY{p}{)}
\end{Verbatim}

    \section{NumPy }\label{numpy}

Kevin Stratford ~ ~ ~ ~ ~ ~ ~ ~ ~ ~kevin@epcc.ed.ac.uk Emmanouil
Farsarakis ~ ~ farsarakis@epcc.ed.ac.uk

Other course authors: Neelofer Banglawala Andy Turner Arno Proeme 

    ~

    ~

www.archer.ac.uk support@archer.ac.uk

 

    {[}NumPy{]} ~ Introducing NumPy

\begin{itemize}
\item
  Core Python provides lists
\item
  Lists are slow for many numerical algorithms
\item
  NumPy provides fast precompiled functions for numerical routines:
\item
  multidimensional arrays : faster than lists
\item
  matrices and linear algebra operations
\item
  random number generation
\item
  Fourier transforms and much more\ldots{}
\item
  https://www.numpy.org/
\end{itemize}

    {[}NumPy{]} ~ Calculating $\pi$

If we know the area $A$ of square length $R$, and the area $Q$ of the
quarter circle with radius $R$, we can calculate $\pi$ : \$ Q/A =
\pi R\^{}2/ 4 R\^{}2 \$, so

\$ \pi = 4,Q/A \$

    {[}NumPy{]} ~ Calculating $\pi$ : monte carlo method

We can use the monte carlo method to determine areas $A$ and $Q$ and
approximate $\pi$. For $N$ iterations

\begin{enumerate}
\def\labelenumi{\arabic{enumi}.}
\itemsep1pt\parskip0pt\parsep0pt
\item
  randomly generate the coordinates $(x,\,y)$, where
  $0 \leq \,x,\, y <R$ 
\item
  Calculate distance \$ r = x\^{}2 + y\^{}2 $. Check if $(x,,y)\$ lies
  within radius of circle
\item
  Check if \$ r \$ lies within radius $R$ of circle i.e.~if $r \leq R $ 
\item
  if yes, add to count for approximating area of circle
\end{enumerate}

The numerical approximation of $\pi$ is then : 4 * (count/$N$)

    {[}NumPy{]} ~ Calculating $\pi$ : a solution 

    \begin{Verbatim}[commandchars=\\\{\}]
{\color{incolor}In [{\color{incolor} }]:} \PY{c}{\PYZsh{} calculate pi}
        \PY{k+kn}{import} \PY{n+nn}{numpy} \PY{k+kn}{as} \PY{n+nn}{np}
        
        \PY{c}{\PYZsh{} N : number of iterations}
        \PY{k}{def} \PY{n+nf}{calc\PYZus{}pi}\PY{p}{(}\PY{n}{N}\PY{p}{)}\PY{p}{:}
            \PY{n}{x} \PY{o}{=} \PY{n}{np}\PY{o}{.}\PY{n}{random}\PY{o}{.}\PY{n}{ranf}\PY{p}{(}\PY{n}{N}\PY{p}{)}\PY{p}{;}
            \PY{n}{y} \PY{o}{=} \PY{n}{np}\PY{o}{.}\PY{n}{random}\PY{o}{.}\PY{n}{ranf}\PY{p}{(}\PY{n}{N}\PY{p}{)}\PY{p}{;}
            \PY{n}{r} \PY{o}{=} \PY{n}{np}\PY{o}{.}\PY{n}{sqrt}\PY{p}{(}\PY{n}{x}\PY{o}{*}\PY{n}{x} \PY{o}{+} \PY{n}{y}\PY{o}{*}\PY{n}{y}\PY{p}{)}\PY{p}{;}
            \PY{n}{c}\PY{o}{=}\PY{n}{r}\PY{p}{[} \PY{n}{r} \PY{o}{\PYZlt{}}\PY{o}{=} \PY{l+m+mf}{1.0} \PY{p}{]}
            \PY{k}{return} \PY{l+m+mi}{4}\PY{o}{*}\PY{n+nb}{float}\PY{p}{(}\PY{p}{(}\PY{n}{c}\PY{o}{.}\PY{n}{size}\PY{p}{)}\PY{p}{)}\PY{o}{/}\PY{n+nb}{float}\PY{p}{(}\PY{n}{N}\PY{p}{)}
        
        \PY{c}{\PYZsh{} time the results}
        \PY{n}{pts} \PY{o}{=} \PY{l+m+mi}{6}\PY{p}{;} \PY{n}{N} \PY{o}{=} \PY{n}{np}\PY{o}{.}\PY{n}{logspace}\PY{p}{(}\PY{l+m+mi}{1}\PY{p}{,}\PY{l+m+mi}{8}\PY{p}{,}\PY{n}{num}\PY{o}{=}\PY{n}{pts}\PY{p}{)}\PY{p}{;}
        \PY{n}{result} \PY{o}{=} \PY{n}{np}\PY{o}{.}\PY{n}{zeros}\PY{p}{(}\PY{n}{pts}\PY{p}{)}\PY{p}{;} \PY{n}{count} \PY{o}{=} \PY{l+m+mi}{0}\PY{p}{;}
        \PY{k}{for} \PY{n}{n} \PY{o+ow}{in} \PY{n}{N}\PY{p}{:}
            \PY{n}{result} \PY{o}{=} \PY{o}{\PYZpc{}}\PY{k}{timeit} \PYZhy{}o \PYZhy{}n1 calc\PYZus{}pi(n) 
            \PY{n}{result}\PY{p}{[}\PY{n}{count}\PY{p}{]} \PY{o}{=} \PY{n}{result}\PY{o}{.}\PY{n}{best}
            \PY{n}{count} \PY{o}{+}\PY{o}{=} \PY{l+m+mi}{1}
            
        \PY{c}{\PYZsh{} and save results to file   }
        \PY{n}{np}\PY{o}{.}\PY{n}{savetxt}\PY{p}{(}\PY{l+s}{\PYZsq{}}\PY{l+s}{calcpi\PYZus{}timings.txt}\PY{l+s}{\PYZsq{}}\PY{p}{,} \PY{n}{np}\PY{o}{.}\PY{n}{c\PYZus{}}\PY{p}{[}\PY{n}{N}\PY{p}{,}\PY{n}{results}\PY{p}{]}\PY{p}{,}
                   \PY{n}{fmt}\PY{o}{=}\PY{l+s}{\PYZsq{}}\PY{l+s+si}{\PYZpc{}1.4e}\PY{l+s}{ }\PY{l+s+si}{\PYZpc{}1.6e}\PY{l+s}{\PYZsq{}}\PY{p}{)}\PY{p}{;}
\end{Verbatim}

    {[}NumPy{]} ~ Creating arrays ~ I

~

    \begin{Verbatim}[commandchars=\\\{\}]
{\color{incolor}In [{\color{incolor} }]:} \PY{c}{\PYZsh{} import numpy as alias np}
        \PY{k+kn}{import} \PY{n+nn}{numpy} \PY{k+kn}{as} \PY{n+nn}{np}
\end{Verbatim}

    \begin{Verbatim}[commandchars=\\\{\}]
{\color{incolor}In [{\color{incolor} }]:} \PY{c}{\PYZsh{} create a 1d array with a list}
        \PY{n}{a} \PY{o}{=} \PY{n}{np}\PY{o}{.}\PY{n}{array}\PY{p}{(} \PY{p}{[}\PY{o}{\PYZhy{}}\PY{l+m+mi}{1}\PY{p}{,}\PY{l+m+mi}{0}\PY{p}{,}\PY{l+m+mi}{1}\PY{p}{]} \PY{p}{)}\PY{p}{;} \PY{n}{a}
\end{Verbatim}

    ~

    {[}NumPy{]} ~ Creating arrays ~ II

~

    \begin{Verbatim}[commandchars=\\\{\}]
{\color{incolor}In [{\color{incolor} }]:} \PY{c}{\PYZsh{} use arrays to create arrays}
        \PY{n}{b} \PY{o}{=} \PY{n}{np}\PY{o}{.}\PY{n}{array}\PY{p}{(} \PY{n}{a} \PY{p}{)}\PY{p}{;} \PY{n}{b}
\end{Verbatim}

    \begin{Verbatim}[commandchars=\\\{\}]
{\color{incolor}In [{\color{incolor} }]:} \PY{c}{\PYZsh{} use numpy functions to create arrays }
        \PY{c}{\PYZsh{} arange for arrays, range for lists!}
        \PY{n}{a} \PY{o}{=} \PY{n}{np}\PY{o}{.}\PY{n}{arange}\PY{p}{(} \PY{o}{\PYZhy{}}\PY{l+m+mi}{2}\PY{p}{,} \PY{l+m+mi}{6}\PY{p}{,} \PY{l+m+mi}{2} \PY{p}{)}\PY{p}{;} \PY{n}{a}
\end{Verbatim}

    ~

    {[}NumPy{]} ~ Creating arrays ~ III

~

    \begin{Verbatim}[commandchars=\\\{\}]
{\color{incolor}In [{\color{incolor} }]:} \PY{c}{\PYZsh{} between start, stop, sample step points}
        \PY{n}{a} \PY{o}{=} \PY{n}{np}\PY{o}{.}\PY{n}{linspace}\PY{p}{(}\PY{o}{\PYZhy{}}\PY{l+m+mi}{10}\PY{p}{,} \PY{l+m+mi}{10}\PY{p}{,} \PY{l+m+mi}{5}\PY{p}{)}\PY{p}{;} 
        \PY{n}{a}\PY{p}{;}
\end{Verbatim}

    \begin{Verbatim}[commandchars=\\\{\}]
{\color{incolor}In [{\color{incolor} }]:} \PY{c}{\PYZsh{} Ex: can you guess these functions do?}
        \PY{n}{b} \PY{o}{=} \PY{n}{np}\PY{o}{.}\PY{n}{zeros}\PY{p}{(}\PY{l+m+mi}{3}\PY{p}{)}\PY{p}{;} \PY{k}{print} \PY{n}{b}
        \PY{n}{c} \PY{o}{=} \PY{n}{np}\PY{o}{.}\PY{n}{ones}\PY{p}{(}\PY{l+m+mi}{3}\PY{p}{)}\PY{p}{;} \PY{k}{print} \PY{n}{c}
\end{Verbatim}

    \begin{Verbatim}[commandchars=\\\{\}]
{\color{incolor}In [{\color{incolor} }]:} \PY{c}{\PYZsh{} Ex++: what does this do? Check documentation!}
        \PY{n}{h} \PY{o}{=} \PY{n}{np}\PY{o}{.}\PY{n}{hstack}\PY{p}{(} \PY{p}{(}\PY{n}{a}\PY{p}{,} \PY{n}{a}\PY{p}{,} \PY{n}{a}\PY{p}{)}\PY{p}{,} \PY{l+m+mi}{0} \PY{p}{)}\PY{p}{;} \PY{k}{print} \PY{n}{h}
\end{Verbatim}

    ~

    {[}NumPy{]} ~ Array characteristics

~

    \begin{Verbatim}[commandchars=\\\{\}]
{\color{incolor}In [{\color{incolor} }]:} \PY{c}{\PYZsh{} array characteristics such as:}
        \PY{k}{print} \PY{n}{a}
        \PY{k}{print} \PY{n}{a}\PY{o}{.}\PY{n}{ndim}  \PY{c}{\PYZsh{} dimensions}
        \PY{k}{print} \PY{n}{a}\PY{o}{.}\PY{n}{shape} \PY{c}{\PYZsh{} shape}
        \PY{k}{print} \PY{n}{a}\PY{o}{.}\PY{n}{size}  \PY{c}{\PYZsh{} size}
        \PY{k}{print} \PY{n}{a}\PY{o}{.}\PY{n}{dtype} \PY{c}{\PYZsh{} data type}
\end{Verbatim}

    \begin{Verbatim}[commandchars=\\\{\}]
{\color{incolor}In [{\color{incolor} }]:} \PY{c}{\PYZsh{} can choose data type}
        \PY{n}{a} \PY{o}{=} \PY{n}{np}\PY{o}{.}\PY{n}{array}\PY{p}{(} \PY{p}{[}\PY{l+m+mi}{1}\PY{p}{,}\PY{l+m+mi}{2}\PY{p}{,}\PY{l+m+mi}{3}\PY{p}{]}\PY{p}{,} \PY{n}{np}\PY{o}{.}\PY{n}{int16} \PY{p}{)}\PY{p}{;} \PY{n}{a}\PY{o}{.}\PY{n}{dtype}
\end{Verbatim}

    ~

    {[}NumPy{]} ~ Multi-dimensional arrays ~ I

~

    \begin{Verbatim}[commandchars=\\\{\}]
{\color{incolor}In [{\color{incolor} }]:} \PY{c}{\PYZsh{} multi\PYZhy{}dimensional arrays e.g. 2d array or matrix }
        \PY{c}{\PYZsh{} e.g. list of lists}
        \PY{n}{mat} \PY{o}{=} \PY{n}{np}\PY{o}{.}\PY{n}{array}\PY{p}{(} \PY{p}{[}\PY{p}{[}\PY{l+m+mi}{1}\PY{p}{,}\PY{l+m+mi}{2}\PY{p}{,}\PY{l+m+mi}{3}\PY{p}{]}\PY{p}{,} \PY{p}{[}\PY{l+m+mi}{4}\PY{p}{,}\PY{l+m+mi}{5}\PY{p}{,}\PY{l+m+mi}{6}\PY{p}{]}\PY{p}{]}\PY{p}{)}\PY{p}{;}
        \PY{k}{print} \PY{n}{mat}\PY{p}{;} \PY{k}{print} \PY{n}{mat}\PY{o}{.}\PY{n}{size}\PY{p}{;} \PY{n}{mat}\PY{o}{.}\PY{n}{shape}
\end{Verbatim}

    \begin{Verbatim}[commandchars=\\\{\}]
{\color{incolor}In [{\color{incolor} }]:} \PY{c}{\PYZsh{} join arrays along first axis (0)}
        \PY{n}{d} \PY{o}{=} \PY{n}{np}\PY{o}{.}\PY{n}{r\PYZus{}}\PY{p}{[}\PY{n}{np}\PY{o}{.}\PY{n}{array}\PY{p}{(}\PY{p}{[}\PY{l+m+mi}{1}\PY{p}{,}\PY{l+m+mi}{2}\PY{p}{,}\PY{l+m+mi}{3}\PY{p}{]}\PY{p}{)}\PY{p}{,} \PY{l+m+mi}{0}\PY{p}{,} \PY{l+m+mi}{0}\PY{p}{,} \PY{p}{[}\PY{l+m+mi}{4}\PY{p}{,}\PY{l+m+mi}{5}\PY{p}{,}\PY{l+m+mi}{6}\PY{p}{]}\PY{p}{]}\PY{p}{;}
        \PY{k}{print} \PY{n}{d}\PY{p}{;} \PY{n}{d}\PY{o}{.}\PY{n}{shape}
\end{Verbatim}

    ~

    {[}NumPy{]} ~ Multi-dimensional arrays ~ II

~

    \begin{Verbatim}[commandchars=\\\{\}]
{\color{incolor}In [{\color{incolor} }]:} \PY{c}{\PYZsh{} join arrays along second axis (1)}
        \PY{n}{d} \PY{o}{=} \PY{n}{np}\PY{o}{.}\PY{n}{c\PYZus{}}\PY{p}{[}\PY{n}{np}\PY{o}{.}\PY{n}{array}\PY{p}{(}\PY{p}{[}\PY{l+m+mi}{1}\PY{p}{,}\PY{l+m+mi}{2}\PY{p}{,}\PY{l+m+mi}{3}\PY{p}{]}\PY{p}{)}\PY{p}{,} \PY{p}{[}\PY{l+m+mi}{4}\PY{p}{,}\PY{l+m+mi}{5}\PY{p}{,}\PY{l+m+mi}{6}\PY{p}{]}\PY{p}{]}\PY{p}{;}
        \PY{k}{print} \PY{n}{d}\PY{p}{;} \PY{n}{d}\PY{o}{.}\PY{n}{shape}
\end{Verbatim}

    \begin{Verbatim}[commandchars=\\\{\}]
{\color{incolor}In [{\color{incolor} }]:} \PY{c}{\PYZsh{} Ex: use r\PYZus{}, c\PYZus{} with nd (n\PYZgt{}1) arrays}
\end{Verbatim}

    \begin{Verbatim}[commandchars=\\\{\}]
{\color{incolor}In [{\color{incolor} }]:} \PY{c}{\PYZsh{} Ex: can you guess the shape of these arrays?}
        \PY{n}{h} \PY{o}{=} \PY{n}{np}\PY{o}{.}\PY{n}{array}\PY{p}{(} \PY{p}{[}\PY{l+m+mi}{1}\PY{p}{,}\PY{l+m+mi}{2}\PY{p}{,}\PY{l+m+mi}{3}\PY{p}{,}\PY{l+m+mi}{4}\PY{p}{,}\PY{l+m+mi}{5}\PY{p}{,}\PY{l+m+mi}{6}\PY{p}{]} \PY{p}{)}\PY{p}{;} 
        \PY{n}{i} \PY{o}{=} \PY{n}{np}\PY{o}{.}\PY{n}{array}\PY{p}{(} \PY{p}{[}\PY{p}{[}\PY{l+m+mi}{1}\PY{p}{,}\PY{l+m+mi}{1}\PY{p}{]}\PY{p}{,}\PY{p}{[}\PY{l+m+mi}{2}\PY{p}{,}\PY{l+m+mi}{2}\PY{p}{]}\PY{p}{,}\PY{p}{[}\PY{l+m+mi}{3}\PY{p}{,}\PY{l+m+mi}{3}\PY{p}{]}\PY{p}{,}\PY{p}{[}\PY{l+m+mi}{4}\PY{p}{,}\PY{l+m+mi}{4}\PY{p}{]}\PY{p}{,}\PY{p}{[}\PY{l+m+mi}{5}\PY{p}{,}\PY{l+m+mi}{5}\PY{p}{]}\PY{p}{,}\PY{p}{[}\PY{l+m+mi}{6}\PY{p}{,}\PY{l+m+mi}{6}\PY{p}{]}\PY{p}{]} \PY{p}{)}\PY{p}{;} 
        \PY{n}{j} \PY{o}{=} \PY{n}{np}\PY{o}{.}\PY{n}{array}\PY{p}{(} \PY{p}{[}\PY{p}{[}\PY{p}{[}\PY{l+m+mi}{1}\PY{p}{]}\PY{p}{,}\PY{p}{[}\PY{l+m+mi}{2}\PY{p}{]}\PY{p}{,}\PY{p}{[}\PY{l+m+mi}{3}\PY{p}{]}\PY{p}{,}\PY{p}{[}\PY{l+m+mi}{4}\PY{p}{]}\PY{p}{,}\PY{p}{[}\PY{l+m+mi}{5}\PY{p}{]}\PY{p}{,}\PY{p}{[}\PY{l+m+mi}{6}\PY{p}{]}\PY{p}{]}\PY{p}{]} \PY{p}{)}\PY{p}{;} 
        \PY{n}{k} \PY{o}{=} \PY{n}{np}\PY{o}{.}\PY{n}{array}\PY{p}{(} \PY{p}{[}\PY{p}{[}\PY{p}{[}\PY{p}{[}\PY{l+m+mi}{1}\PY{p}{]}\PY{p}{,}\PY{p}{[}\PY{l+m+mi}{2}\PY{p}{]}\PY{p}{,}\PY{p}{[}\PY{l+m+mi}{3}\PY{p}{]}\PY{p}{,}\PY{p}{[}\PY{l+m+mi}{4}\PY{p}{]}\PY{p}{,}\PY{p}{[}\PY{l+m+mi}{5}\PY{p}{]}\PY{p}{,}\PY{p}{[}\PY{l+m+mi}{6}\PY{p}{]}\PY{p}{]}\PY{p}{]}\PY{p}{]} \PY{p}{)}\PY{p}{;}
\end{Verbatim}

    {[}NumPy{]} ~ Reshaping arrays ~ I

~

    \begin{Verbatim}[commandchars=\\\{\}]
{\color{incolor}In [{\color{incolor} }]:} \PY{c}{\PYZsh{} reshape 1d arrays into nd arrays original matrix unaffected}
        \PY{n}{mat} \PY{o}{=} \PY{n}{np}\PY{o}{.}\PY{n}{arange}\PY{p}{(}\PY{l+m+mi}{6}\PY{p}{)}\PY{p}{;} \PY{k}{print} \PY{n}{mat}
        \PY{k}{print} \PY{n}{mat}\PY{o}{.}\PY{n}{reshape}\PY{p}{(} \PY{p}{(}\PY{l+m+mi}{3}\PY{p}{,} \PY{l+m+mi}{2}\PY{p}{)} \PY{p}{)}
        \PY{k}{print} \PY{n}{mat}\PY{p}{;} \PY{k}{print} \PY{n}{mat}\PY{o}{.}\PY{n}{size}\PY{p}{;}
        \PY{k}{print} \PY{n}{mat}\PY{o}{.}\PY{n}{shape}
\end{Verbatim}

    \begin{Verbatim}[commandchars=\\\{\}]
{\color{incolor}In [{\color{incolor} }]:} \PY{c}{\PYZsh{} can also use the shape, this modifies the original array}
        \PY{n}{a} \PY{o}{=} \PY{n}{np}\PY{o}{.}\PY{n}{zeros}\PY{p}{(}\PY{l+m+mi}{10}\PY{p}{)}\PY{p}{;} \PY{k}{print} \PY{n}{a}
        \PY{n}{a}\PY{o}{.}\PY{n}{shape} \PY{o}{=} \PY{p}{(}\PY{l+m+mi}{2}\PY{p}{,}\PY{l+m+mi}{5}\PY{p}{)}
        \PY{k}{print} \PY{n}{a}\PY{p}{;} \PY{k}{print} \PY{n}{a}\PY{o}{.}\PY{n}{shape}\PY{p}{;}
\end{Verbatim}

    ~

    {[}NumPy{]} ~ Reshaping arrays ~ II

~

    \begin{Verbatim}[commandchars=\\\{\}]
{\color{incolor}In [{\color{incolor} }]:} \PY{c}{\PYZsh{} Ex: what do flatten() and ravel()? }
        \PY{c}{\PYZsh{} use online documentation, or \PYZsq{}?\PYZsq{}}
        \PY{n}{mat2} \PY{o}{=} \PY{n}{mat}\PY{o}{.}\PY{n}{flatten}\PY{p}{(}\PY{p}{)}
        \PY{n}{mat2} \PY{o}{=} \PY{n}{mat}\PY{o}{.}\PY{n}{ravel}\PY{p}{(}\PY{p}{)}
\end{Verbatim}

    \begin{Verbatim}[commandchars=\\\{\}]
{\color{incolor}In [{\color{incolor} }]:} \PY{c}{\PYZsh{} Ex: split a martix? Change the cuts and axis values}
        \PY{c}{\PYZsh{} need help?: np.split?}
        \PY{n}{cuts}\PY{o}{=}\PY{l+m+mi}{2}\PY{p}{;}
        \PY{n}{np}\PY{o}{.}\PY{n}{split}\PY{p}{(}\PY{n}{mat}\PY{p}{,} \PY{n}{cuts}\PY{p}{,} \PY{n}{axis}\PY{o}{=}\PY{l+m+mi}{0}\PY{p}{)}
\end{Verbatim}

    ~

    {[}NumPy{]} ~ Functions for you to explore

~

    \begin{Verbatim}[commandchars=\\\{\}]
{\color{incolor}In [{\color{incolor} }]:} \PY{c}{\PYZsh{} Ex: can you guess what these functions do?}
        \PY{c}{\PYZsh{} np.copyto(b, a);}
        \PY{c}{\PYZsh{} v = np.vstack( (arr2d, arr2d) ); print v; v.ndim;}
        \PY{c}{\PYZsh{} c0 = np.concatenate( (arr2d, arr2d), axis=0); c0;}
        \PY{c}{\PYZsh{} c1 = np.concatenate(( mat, mat ), axis=1); print \PYZdq{}c1:\PYZdq{}, c1;}
\end{Verbatim}

    \begin{Verbatim}[commandchars=\\\{\}]
{\color{incolor}In [{\color{incolor} }]:} \PY{c}{\PYZsh{} Ex++: other functions to explore}
        \PY{c}{\PYZsh{}}
        \PY{c}{\PYZsh{} stack(arrays[, axis])}
        \PY{c}{\PYZsh{} tile(A, reps)}
        \PY{c}{\PYZsh{} repeat(a, repeats[, axis])}
        \PY{c}{\PYZsh{} unique(ar[, return\PYZus{}index, return\PYZus{}inverse, ...])}
        \PY{c}{\PYZsh{} trim\PYZus{}zeros(filt[, trim]), fill(scalar)}
        \PY{c}{\PYZsh{} xv, yv = meshgrid(x,y)}
\end{Verbatim}

    {[}NumPy{]} ~ Accessing arrays ~ I

~

    \begin{Verbatim}[commandchars=\\\{\}]
{\color{incolor}In [{\color{incolor} }]:} \PY{c}{\PYZsh{} basic indexing and slicing we know from lists}
        \PY{n}{a} \PY{o}{=} \PY{n}{np}\PY{o}{.}\PY{n}{arange}\PY{p}{(}\PY{l+m+mi}{8}\PY{p}{)}\PY{p}{;} \PY{k}{print} \PY{n}{a}
        \PY{n}{a}\PY{p}{[}\PY{l+m+mi}{3}\PY{p}{]}
\end{Verbatim}

    \begin{Verbatim}[commandchars=\\\{\}]
{\color{incolor}In [{\color{incolor} }]:} \PY{c}{\PYZsh{} a[start:stop:step] \PYZhy{}\PYZhy{}\PYZgt{} [start, stop every step)}
        \PY{k}{print} \PY{n}{a}\PY{p}{[}\PY{l+m+mi}{0}\PY{p}{:}\PY{l+m+mi}{7}\PY{p}{:}\PY{l+m+mi}{2}\PY{p}{]}
        \PY{k}{print} \PY{n}{a}\PY{p}{[}\PY{l+m+mi}{0}\PY{p}{:}\PY{p}{:}\PY{l+m+mi}{2}\PY{p}{]}
\end{Verbatim}

    \begin{Verbatim}[commandchars=\\\{\}]
{\color{incolor}In [{\color{incolor} }]:} \PY{c}{\PYZsh{} negative indices are valid! }
        \PY{c}{\PYZsh{} last element index is \PYZhy{}1}
        \PY{k}{print} \PY{n}{a}\PY{p}{[}\PY{l+m+mi}{2}\PY{p}{:}\PY{o}{\PYZhy{}}\PY{l+m+mi}{3}\PY{p}{:}\PY{l+m+mi}{2}\PY{p}{]}
\end{Verbatim}

    ~

    {[}NumPy{]} ~ Accessing arrays ~ II

~

    \begin{Verbatim}[commandchars=\\\{\}]
{\color{incolor}In [{\color{incolor} }]:} \PY{c}{\PYZsh{} basic indexing of a 2d array : take care of each dimension}
        \PY{n}{nd} \PY{o}{=} \PY{n}{np}\PY{o}{.}\PY{n}{arange}\PY{p}{(}\PY{l+m+mi}{12}\PY{p}{)}\PY{o}{.}\PY{n}{reshape}\PY{p}{(}\PY{p}{(}\PY{l+m+mi}{4}\PY{p}{,}\PY{l+m+mi}{3}\PY{p}{)}\PY{p}{)}\PY{p}{;} \PY{k}{print} \PY{n}{nd}\PY{p}{;}
        \PY{k}{print} \PY{n}{nd}\PY{p}{[}\PY{l+m+mi}{2}\PY{p}{,}\PY{l+m+mi}{2}\PY{p}{]}\PY{p}{;}
        \PY{k}{print} \PY{n}{nd}\PY{p}{[}\PY{l+m+mi}{2}\PY{p}{]}\PY{p}{[}\PY{l+m+mi}{2}\PY{p}{]}\PY{p}{;}
\end{Verbatim}

    \begin{Verbatim}[commandchars=\\\{\}]
{\color{incolor}In [{\color{incolor} }]:} \PY{c}{\PYZsh{} get corner elements 0,2,9,11}
        \PY{k}{print} \PY{n}{nd}\PY{p}{[}\PY{l+m+mi}{0}\PY{p}{:}\PY{l+m+mi}{4}\PY{p}{:}\PY{l+m+mi}{3}\PY{p}{,} \PY{l+m+mi}{0}\PY{p}{:}\PY{l+m+mi}{3}\PY{p}{:}\PY{l+m+mi}{2}\PY{p}{]}
\end{Verbatim}

    \begin{Verbatim}[commandchars=\\\{\}]
{\color{incolor}In [{\color{incolor} }]:} \PY{c}{\PYZsh{} Ex: get elements 7,8,10,11 that make up the bottom right corner}
        \PY{n}{nd} \PY{o}{=} \PY{n}{np}\PY{o}{.}\PY{n}{arange}\PY{p}{(}\PY{l+m+mi}{12}\PY{p}{)}\PY{o}{.}\PY{n}{reshape}\PY{p}{(}\PY{p}{(}\PY{l+m+mi}{4}\PY{p}{,}\PY{l+m+mi}{3}\PY{p}{)}\PY{p}{)}\PY{p}{;} 
        \PY{k}{print} \PY{n}{nd}\PY{p}{;} \PY{n}{nd}\PY{p}{[}\PY{l+m+mi}{2}\PY{p}{:}\PY{l+m+mi}{4}\PY{p}{,} \PY{l+m+mi}{1}\PY{p}{:}\PY{l+m+mi}{3}\PY{p}{]}
\end{Verbatim}

    ~

    {[}NumPy{]} ~ Slices and copies ~ I

~

    \begin{Verbatim}[commandchars=\\\{\}]
{\color{incolor}In [{\color{incolor} }]:} \PY{c}{\PYZsh{} slices are views (like references) }
        \PY{c}{\PYZsh{} on an array, can change elements}
        \PY{n}{nd}\PY{p}{[}\PY{l+m+mi}{2}\PY{p}{:}\PY{l+m+mi}{4}\PY{p}{,} \PY{l+m+mi}{1}\PY{p}{:}\PY{l+m+mi}{3}\PY{p}{]} \PY{o}{=} \PY{o}{\PYZhy{}}\PY{l+m+mi}{1}\PY{p}{;} \PY{n}{nd}
\end{Verbatim}

    \begin{Verbatim}[commandchars=\\\{\}]
{\color{incolor}In [{\color{incolor} }]:} \PY{c}{\PYZsh{} assign slice to a variable to prevent this}
        \PY{n}{s} \PY{o}{=} \PY{n}{nd}\PY{p}{[}\PY{l+m+mi}{2}\PY{p}{:}\PY{l+m+mi}{4}\PY{p}{,} \PY{l+m+mi}{1}\PY{p}{:}\PY{l+m+mi}{3}\PY{p}{]}\PY{p}{;} \PY{k}{print} \PY{n}{nd}\PY{p}{;} 
        \PY{n}{s} \PY{o}{=} \PY{o}{\PYZhy{}}\PY{l+m+mi}{1}\PY{p}{;} \PY{n}{nd}
\end{Verbatim}

    ~

    {[}NumPy{]} ~ Slices and copies ~ II

~

    \begin{Verbatim}[commandchars=\\\{\}]
{\color{incolor}In [{\color{incolor} }]:} \PY{c}{\PYZsh{} Care \PYZhy{} simple assignment between arrays}
        \PY{c}{\PYZsh{} creates references!}
        \PY{n}{nd} \PY{o}{=} \PY{n}{np}\PY{o}{.}\PY{n}{arange}\PY{p}{(}\PY{l+m+mi}{12}\PY{p}{)}\PY{o}{.}\PY{n}{reshape}\PY{p}{(}\PY{p}{(}\PY{l+m+mi}{4}\PY{p}{,}\PY{l+m+mi}{3}\PY{p}{)}\PY{p}{)}
        \PY{n}{md} \PY{o}{=} \PY{n}{nd}
        \PY{n}{md}\PY{p}{[}\PY{l+m+mi}{3}\PY{p}{]} \PY{o}{=} \PY{l+m+mi}{1000}
        \PY{k}{print} \PY{n}{nd}
\end{Verbatim}

    \begin{Verbatim}[commandchars=\\\{\}]
{\color{incolor}In [{\color{incolor} }]:} \PY{c}{\PYZsh{} avoid this by creating distinct copies}
        \PY{c}{\PYZsh{} using copy()}
        \PY{n}{nd} \PY{o}{=} \PY{n}{np}\PY{o}{.}\PY{n}{arange}\PY{p}{(}\PY{l+m+mi}{12}\PY{p}{)}\PY{o}{.}\PY{n}{reshape}\PY{p}{(}\PY{p}{(}\PY{l+m+mi}{4}\PY{p}{,}\PY{l+m+mi}{3}\PY{p}{)}\PY{p}{)}
        \PY{n}{md} \PY{o}{=} \PY{n}{nd}\PY{o}{.}\PY{n}{copy}\PY{p}{(}\PY{p}{)}
        \PY{n}{md}\PY{p}{[}\PY{l+m+mi}{3}\PY{p}{]} \PY{o}{=} \PY{l+m+mi}{999}
        \PY{k}{print} \PY{n}{nd}
\end{Verbatim}

    ~

    {[}NumPy{]} ~ Fancy indexing ~ I

~

    \begin{Verbatim}[commandchars=\\\{\}]
{\color{incolor}In [{\color{incolor} }]:} \PY{c}{\PYZsh{} advanced or fancy indexing lets you do more}
        \PY{n}{p} \PY{o}{=} \PY{n}{np}\PY{o}{.}\PY{n}{array}\PY{p}{(} \PY{p}{[}\PY{p}{[}\PY{l+m+mi}{0}\PY{p}{,}\PY{l+m+mi}{1}\PY{p}{,}\PY{l+m+mi}{2}\PY{p}{]}\PY{p}{,} \PY{p}{[}\PY{l+m+mi}{3}\PY{p}{,}\PY{l+m+mi}{4}\PY{p}{,}\PY{l+m+mi}{5}\PY{p}{]}\PY{p}{,} \PY{p}{[}\PY{l+m+mi}{6}\PY{p}{,}\PY{l+m+mi}{7}\PY{p}{,}\PY{l+m+mi}{8}\PY{p}{]}\PY{p}{,} \PY{p}{[}\PY{l+m+mi}{9}\PY{p}{,}\PY{l+m+mi}{10}\PY{p}{,}\PY{l+m+mi}{11}\PY{p}{]}\PY{p}{]} \PY{p}{)}\PY{p}{;} 
        \PY{k}{print} \PY{n}{p}
\end{Verbatim}

    \begin{Verbatim}[commandchars=\\\{\}]
{\color{incolor}In [{\color{incolor} }]:} \PY{n}{rows} \PY{o}{=} \PY{p}{[}\PY{l+m+mi}{0}\PY{p}{,}\PY{l+m+mi}{0}\PY{p}{,}\PY{l+m+mi}{3}\PY{p}{,}\PY{l+m+mi}{3}\PY{p}{]}\PY{p}{;} \PY{n}{cols} \PY{o}{=} \PY{p}{[}\PY{l+m+mi}{0}\PY{p}{,}\PY{l+m+mi}{2}\PY{p}{,}\PY{l+m+mi}{0}\PY{p}{,}\PY{l+m+mi}{2}\PY{p}{]}\PY{p}{;}
        \PY{k}{print} \PY{n}{p}\PY{p}{[}\PY{n}{rows}\PY{p}{,} \PY{n}{cols}\PY{p}{]}
\end{Verbatim}

    \begin{Verbatim}[commandchars=\\\{\}]
{\color{incolor}In [{\color{incolor} }]:} \PY{c}{\PYZsh{} Ex: what will this slice look like?}
        \PY{n}{m} \PY{o}{=} \PY{n}{np}\PY{o}{.}\PY{n}{array}\PY{p}{(} \PY{p}{[}\PY{p}{[}\PY{l+m+mi}{0}\PY{p}{,}\PY{o}{\PYZhy{}}\PY{l+m+mi}{1}\PY{p}{,}\PY{l+m+mi}{4}\PY{p}{,}\PY{l+m+mi}{20}\PY{p}{,}\PY{l+m+mi}{99}\PY{p}{]}\PY{p}{,} \PY{p}{[}\PY{o}{\PYZhy{}}\PY{l+m+mi}{3}\PY{p}{,}\PY{o}{\PYZhy{}}\PY{l+m+mi}{5}\PY{p}{,}\PY{l+m+mi}{6}\PY{p}{,}\PY{l+m+mi}{7}\PY{p}{,}\PY{o}{\PYZhy{}}\PY{l+m+mi}{10}\PY{p}{]}\PY{p}{]} \PY{p}{)}\PY{p}{;}
        \PY{k}{print} \PY{n}{m}\PY{p}{[}\PY{p}{[}\PY{l+m+mi}{0}\PY{p}{,}\PY{l+m+mi}{1}\PY{p}{,}\PY{l+m+mi}{1}\PY{p}{,}\PY{l+m+mi}{1}\PY{p}{]}\PY{p}{,} \PY{p}{[}\PY{l+m+mi}{1}\PY{p}{,}\PY{l+m+mi}{0}\PY{p}{,}\PY{l+m+mi}{1}\PY{p}{,}\PY{l+m+mi}{4}\PY{p}{]}\PY{p}{]}\PY{p}{;}
\end{Verbatim}

    ~

    {[}NumPy{]} ~ Fancy indexing ~ II

~

    \begin{Verbatim}[commandchars=\\\{\}]
{\color{incolor}In [{\color{incolor} }]:} \PY{c}{\PYZsh{} can use conditionals in indexing}
        \PY{c}{\PYZsh{} m = np.array([[0,\PYZhy{}1,4,20,99],[\PYZhy{}3,\PYZhy{}5,6,7,\PYZhy{}10]]);}
        \PY{n}{m}\PY{p}{[} \PY{n}{m} \PY{o}{\PYZlt{}} \PY{l+m+mi}{0} \PY{p}{]}
\end{Verbatim}

    \begin{Verbatim}[commandchars=\\\{\}]
{\color{incolor}In [{\color{incolor} }]:} \PY{c}{\PYZsh{} Ex: can you guess what this does? query: np.sum?}
        \PY{n}{y} \PY{o}{=} \PY{n}{np}\PY{o}{.}\PY{n}{array}\PY{p}{(}\PY{p}{[}\PY{p}{[}\PY{l+m+mi}{0}\PY{p}{,} \PY{l+m+mi}{1}\PY{p}{]}\PY{p}{,} \PY{p}{[}\PY{l+m+mi}{1}\PY{p}{,} \PY{l+m+mi}{1}\PY{p}{]}\PY{p}{,} \PY{p}{[}\PY{l+m+mi}{2}\PY{p}{,} \PY{l+m+mi}{2}\PY{p}{]}\PY{p}{]}\PY{p}{)}\PY{p}{;}
        \PY{n}{rowsum} \PY{o}{=} \PY{n}{y}\PY{o}{.}\PY{n}{sum}\PY{p}{(}\PY{l+m+mi}{1}\PY{p}{)}\PY{p}{;}
        \PY{n}{y}\PY{p}{[}\PY{n}{rowsum} \PY{o}{\PYZlt{}}\PY{o}{=} \PY{l+m+mi}{2}\PY{p}{,} \PY{p}{:}\PY{p}{]}
\end{Verbatim}

    \begin{Verbatim}[commandchars=\\\{\}]
{\color{incolor}In [{\color{incolor} }]:} \PY{c}{\PYZsh{} Ex: and this? }
        \PY{n}{a} \PY{o}{=} \PY{n}{np}\PY{o}{.}\PY{n}{arange}\PY{p}{(}\PY{l+m+mi}{10}\PY{p}{)}\PY{p}{;}
        \PY{n}{mask} \PY{o}{=} \PY{n}{np}\PY{o}{.}\PY{n}{ones}\PY{p}{(}\PY{n+nb}{len}\PY{p}{(}\PY{n}{a}\PY{p}{)}\PY{p}{,} \PY{n}{dtype} \PY{o}{=} \PY{n+nb}{bool}\PY{p}{)}\PY{p}{;}
        \PY{n}{mask}\PY{p}{[}\PY{p}{[}\PY{l+m+mi}{0}\PY{p}{,}\PY{l+m+mi}{2}\PY{p}{,}\PY{l+m+mi}{4}\PY{p}{]}\PY{p}{]} \PY{o}{=} \PY{n+nb+bp}{False}\PY{p}{;} \PY{k}{print} \PY{n}{mask}
        \PY{n}{result} \PY{o}{=} \PY{n}{a}\PY{p}{[}\PY{n}{mask}\PY{p}{]}\PY{p}{;} \PY{n}{result}
\end{Verbatim}

    \begin{Verbatim}[commandchars=\\\{\}]
{\color{incolor}In [{\color{incolor} }]:} \PY{c}{\PYZsh{} Ex: r=np.array([[0,1,2],[3,4,5]]); }
        \PY{n}{xp} \PY{o}{=} \PY{n}{np}\PY{o}{.}\PY{n}{array}\PY{p}{(} \PY{p}{[}\PY{p}{[}\PY{p}{[}\PY{l+m+mi}{1}\PY{p}{,}\PY{l+m+mi}{11}\PY{p}{]}\PY{p}{,}\PY{p}{[}\PY{l+m+mi}{2}\PY{p}{,}\PY{l+m+mi}{22}\PY{p}{]}\PY{p}{,}\PY{p}{[}\PY{l+m+mi}{3}\PY{p}{,}\PY{l+m+mi}{33}\PY{p}{]}\PY{p}{]}\PY{p}{,} \PY{p}{[}\PY{p}{[}\PY{l+m+mi}{4}\PY{p}{,}\PY{l+m+mi}{44}\PY{p}{]}\PY{p}{,}\PY{p}{[}\PY{l+m+mi}{5}\PY{p}{,}\PY{l+m+mi}{55}\PY{p}{]}\PY{p}{,}\PY{p}{[}\PY{l+m+mi}{6}\PY{p}{,}\PY{l+m+mi}{66}\PY{p}{]}\PY{p}{]}\PY{p}{]} \PY{p}{)}\PY{p}{;}
        \PY{n}{xp}\PY{p}{[}\PY{n+nb}{slice}\PY{p}{(}\PY{l+m+mi}{1}\PY{p}{)}\PY{p}{,} \PY{n+nb}{slice}\PY{p}{(}\PY{l+m+mi}{1}\PY{p}{,}\PY{l+m+mi}{3}\PY{p}{,}\PY{n+nb+bp}{None}\PY{p}{)}\PY{p}{,} \PY{n+nb}{slice}\PY{p}{(}\PY{l+m+mi}{1}\PY{p}{)}\PY{p}{]}\PY{p}{;} \PY{n}{xp}\PY{p}{[}\PY{p}{:}\PY{l+m+mi}{1}\PY{p}{,} \PY{l+m+mi}{1}\PY{p}{:}\PY{l+m+mi}{3}\PY{p}{:}\PY{p}{,} \PY{p}{:}\PY{l+m+mi}{1}\PY{p}{]}\PY{p}{;}
        \PY{k}{print} \PY{n}{xp}\PY{p}{[}\PY{p}{[}\PY{l+m+mi}{1}\PY{p}{,}\PY{l+m+mi}{1}\PY{p}{,}\PY{l+m+mi}{1}\PY{p}{]}\PY{p}{,}\PY{p}{[}\PY{l+m+mi}{1}\PY{p}{,}\PY{l+m+mi}{2}\PY{p}{,}\PY{l+m+mi}{1}\PY{p}{]}\PY{p}{,}\PY{p}{[}\PY{l+m+mi}{0}\PY{p}{,}\PY{l+m+mi}{1}\PY{p}{,}\PY{l+m+mi}{0}\PY{p}{]}\PY{p}{]}
\end{Verbatim}

    ~

    {[}NumPy{]} ~ Manipulating arrays

~

    \begin{Verbatim}[commandchars=\\\{\}]
{\color{incolor}In [{\color{incolor} }]:} \PY{c}{\PYZsh{} add an element with insert}
        \PY{n}{a} \PY{o}{=} \PY{n}{np}\PY{o}{.}\PY{n}{arange}\PY{p}{(}\PY{l+m+mi}{6}\PY{p}{)}\PY{o}{.}\PY{n}{reshape}\PY{p}{(}\PY{p}{[}\PY{l+m+mi}{2}\PY{p}{,}\PY{l+m+mi}{3}\PY{p}{]}\PY{p}{)}\PY{p}{;} \PY{k}{print} \PY{n}{a}
        \PY{n}{np}\PY{o}{.}\PY{n}{append}\PY{p}{(}\PY{n}{a}\PY{p}{,} \PY{n}{np}\PY{o}{.}\PY{n}{ones}\PY{p}{(}\PY{p}{[}\PY{l+m+mi}{2}\PY{p}{,}\PY{l+m+mi}{3}\PY{p}{]}\PY{p}{)}\PY{p}{,} \PY{n}{axis}\PY{o}{=}\PY{l+m+mi}{0}\PY{p}{)}
\end{Verbatim}

    \begin{Verbatim}[commandchars=\\\{\}]
{\color{incolor}In [{\color{incolor} }]:} \PY{c}{\PYZsh{} inserting an array of elements}
        \PY{n}{np}\PY{o}{.}\PY{n}{insert}\PY{p}{(}\PY{n}{a}\PY{p}{,} \PY{l+m+mi}{1}\PY{p}{,} \PY{o}{\PYZhy{}}\PY{l+m+mi}{10}\PY{p}{,} \PY{n}{axis}\PY{o}{=}\PY{l+m+mi}{0}\PY{p}{)}
\end{Verbatim}

    \begin{Verbatim}[commandchars=\\\{\}]
{\color{incolor}In [{\color{incolor} }]:} \PY{c}{\PYZsh{} can use delete, or a boolean mask, to delete array elements}
        \PY{n}{a} \PY{o}{=} \PY{n}{np}\PY{o}{.}\PY{n}{arange}\PY{p}{(}\PY{l+m+mi}{10}\PY{p}{)}
        \PY{n}{np}\PY{o}{.}\PY{n}{delete}\PY{p}{(}\PY{n}{a}\PY{p}{,} \PY{p}{[}\PY{l+m+mi}{0}\PY{p}{,}\PY{l+m+mi}{2}\PY{p}{,}\PY{l+m+mi}{4}\PY{p}{]}\PY{p}{,} \PY{n}{axis}\PY{o}{=}\PY{l+m+mi}{0}\PY{p}{)}
\end{Verbatim}

    {[}NumPy{]} ~ Vectorization ~ I

~

    \begin{Verbatim}[commandchars=\\\{\}]
{\color{incolor}In [{\color{incolor} }]:} \PY{c}{\PYZsh{} vectorization allows element\PYZhy{}wise operations (no for loop!)}
        \PY{n}{a} \PY{o}{=} \PY{n}{np}\PY{o}{.}\PY{n}{arange}\PY{p}{(}\PY{l+m+mi}{10}\PY{p}{)}\PY{o}{.}\PY{n}{reshape}\PY{p}{(}\PY{p}{[}\PY{l+m+mi}{2}\PY{p}{,}\PY{l+m+mi}{5}\PY{p}{]}\PY{p}{)}\PY{p}{;} \PY{n}{b} \PY{o}{=} \PY{n}{np}\PY{o}{.}\PY{n}{arange}\PY{p}{(}\PY{l+m+mi}{10}\PY{p}{)}\PY{o}{.}\PY{n}{reshape}\PY{p}{(}\PY{p}{[}\PY{l+m+mi}{2}\PY{p}{,}\PY{l+m+mi}{5}\PY{p}{]}\PY{p}{)}\PY{p}{;}
\end{Verbatim}

    \begin{Verbatim}[commandchars=\\\{\}]
{\color{incolor}In [{\color{incolor} }]:} \PY{o}{\PYZhy{}}\PY{l+m+mf}{0.1}\PY{o}{*}\PY{n}{a}
\end{Verbatim}

    \begin{Verbatim}[commandchars=\\\{\}]
{\color{incolor}In [{\color{incolor} }]:} \PY{n}{a}\PY{o}{*}\PY{n}{b}
\end{Verbatim}

    \begin{Verbatim}[commandchars=\\\{\}]
{\color{incolor}In [{\color{incolor} }]:} \PY{n}{a}\PY{o}{/}\PY{p}{(}\PY{n}{b}\PY{o}{+}\PY{l+m+mi}{1}\PY{p}{)}  \PY{c}{\PYZsh{}.astype(float)}
\end{Verbatim}

    {[}NumPy{]} ~ Random number generation

~

    \begin{Verbatim}[commandchars=\\\{\}]
{\color{incolor}In [{\color{incolor} }]:} \PY{c}{\PYZsh{} random floats}
        \PY{n}{a} \PY{o}{=} \PY{n}{np}\PY{o}{.}\PY{n}{random}\PY{o}{.}\PY{n}{ranf}\PY{p}{(}\PY{l+m+mi}{10}\PY{p}{)}\PY{p}{;} \PY{n}{a} 
\end{Verbatim}

    \begin{Verbatim}[commandchars=\\\{\}]
{\color{incolor}In [{\color{incolor} }]:} \PY{c}{\PYZsh{} create random 2d int array }
        \PY{n}{a} \PY{o}{=} \PY{n}{np}\PY{o}{.}\PY{n}{random}\PY{o}{.}\PY{n}{randint}\PY{p}{(}\PY{l+m+mi}{0}\PY{p}{,} \PY{n}{high}\PY{o}{=}\PY{l+m+mi}{5}\PY{p}{,} \PY{n}{size}\PY{o}{=}\PY{l+m+mi}{25}\PY{p}{)}\PY{o}{.}\PY{n}{reshape}\PY{p}{(}\PY{l+m+mi}{5}\PY{p}{,}\PY{l+m+mi}{5}\PY{p}{)}\PY{p}{;} 
        \PY{k}{print} \PY{n}{a}\PY{p}{;}
\end{Verbatim}

    \begin{Verbatim}[commandchars=\\\{\}]
{\color{incolor}In [{\color{incolor} }]:} \PY{c}{\PYZsh{} generate sample from normal distribution }
        \PY{c}{\PYZsh{} (mean=0, standard deviation=1)}
        \PY{n}{s} \PY{o}{=} \PY{n}{np}\PY{o}{.}\PY{n}{random}\PY{o}{.}\PY{n}{standard\PYZus{}normal}\PY{p}{(}\PY{p}{(}\PY{l+m+mi}{5}\PY{p}{,}\PY{l+m+mi}{5}\PY{p}{)}\PY{p}{)}\PY{p}{;} \PY{n}{s}\PY{p}{;}
\end{Verbatim}

    \begin{Verbatim}[commandchars=\\\{\}]
{\color{incolor}In [{\color{incolor} }]:} \PY{c}{\PYZsh{} Ex: what other ways are there to generate random numbers?}
        \PY{c}{\PYZsh{} What other distributions can you sample? }
\end{Verbatim}

    {[}NumPy{]} ~ File IO

~

    \begin{Verbatim}[commandchars=\\\{\}]
{\color{incolor}In [{\color{incolor} }]:} \PY{c}{\PYZsh{} easy way to save data to text file}
        \PY{n}{pts} \PY{o}{=} \PY{l+m+mi}{5}\PY{p}{;} \PY{n}{x} \PY{o}{=} \PY{n}{np}\PY{o}{.}\PY{n}{arange}\PY{p}{(}\PY{n}{pts}\PY{p}{)}\PY{p}{;} \PY{n}{y} \PY{o}{=} \PY{n}{np}\PY{o}{.}\PY{n}{random}\PY{o}{.}\PY{n}{random}\PY{p}{(}\PY{n}{pts}\PY{p}{)}\PY{p}{;}
\end{Verbatim}

    \begin{Verbatim}[commandchars=\\\{\}]
{\color{incolor}In [{\color{incolor} }]:} \PY{c}{\PYZsh{} format specifiers: d = int, f = float, e = scientific}
        \PY{n}{np}\PY{o}{.}\PY{n}{savetxt}\PY{p}{(}\PY{l+s}{\PYZsq{}}\PY{l+s}{savedata.txt}\PY{l+s}{\PYZsq{}}\PY{p}{,} \PY{n}{np}\PY{o}{.}\PY{n}{c\PYZus{}}\PY{p}{[}\PY{n}{x}\PY{p}{,}\PY{n}{y}\PY{p}{]}\PY{p}{,} \PY{n}{header} \PY{o}{=} \PY{l+s}{\PYZsq{}}\PY{l+s}{DATA}\PY{l+s}{\PYZsq{}}\PY{p}{,} \PY{n}{footer} \PY{o}{=} \PY{l+s}{\PYZsq{}}\PY{l+s}{END}\PY{l+s}{\PYZsq{}}\PY{p}{,}
                   \PY{n}{fmt} \PY{o}{=} \PY{l+s}{\PYZsq{}}\PY{l+s+si}{\PYZpc{}d}\PY{l+s}{ }\PY{l+s+si}{\PYZpc{}1.4f}\PY{l+s}{\PYZsq{}}\PY{p}{)}
\end{Verbatim}

    \begin{Verbatim}[commandchars=\\\{\}]
{\color{incolor}In [{\color{incolor} }]:} \PY{o}{!}cat savedata.txt
        \PY{c}{\PYZsh{} One could do ...}
        \PY{c}{\PYZsh{} p = np.loadtxt(\PYZsq{}savedata.txt\PYZsq{})}
\end{Verbatim}

    \begin{Verbatim}[commandchars=\\\{\}]
{\color{incolor}In [{\color{incolor} }]:} \PY{c}{\PYZsh{} ...but much more flexibility with genfromtext}
        \PY{n}{p} \PY{o}{=} \PY{n}{np}\PY{o}{.}\PY{n}{genfromtxt}\PY{p}{(}\PY{l+s}{\PYZsq{}}\PY{l+s}{savedata.txt}\PY{l+s}{\PYZsq{}}\PY{p}{,} \PY{n}{skip\PYZus{}header}\PY{o}{=}\PY{l+m+mi}{2}\PY{p}{,} \PY{n}{skip\PYZus{}footer}\PY{o}{=}\PY{l+m+mi}{1}\PY{p}{)}\PY{p}{;} \PY{n}{p}
\end{Verbatim}

    \begin{Verbatim}[commandchars=\\\{\}]
{\color{incolor}In [{\color{incolor} }]:} \PY{c}{\PYZsh{} Ex++: what do numpy.save, numpy.load do ?}
\end{Verbatim}

    {[}NumPy{]} ~ Polynomials ~ I

Can represent polynomials with the numpy class Polynomial from
numpy.polynomial.polynomial.

Polynomial({[}a, b, c, d, e{]}) is equivalent to
$p(x) = a\,+\,b\,x \,+\,c\,x^2\,+\,d\,x^3\,+\,e\,x^4$. For example:

\begin{itemize}
\item
  Polynomial({[}1,2,3{]}) is equivalent to
  $p(x) = 1\,+\,2\,x \,+\,3\,x^2$
\item
  Polynomial({[}0,1,0,2,0,3{]}) is equivalent to
  $p(x) = x \,+\,2\,x^3\,+\,3\,x^5 $
\end{itemize}

    {[}NumPy{]} ~ Polynomials ~ II

Can carry out arithmetic operations on polynomials, as well integrate
and differentiate them.

Can also use the polynomial package to find a least-squares fit to data.

    {[}NumPy{]} ~ Polynomials : calculating $\pi$ ~ I

The Taylor series expansion for the trigonometric function $\arctan(y)$
is :

~ ~ ~
$\arctan ( y) \, = \,y - \frac{y^3}{3} + \frac{y^5}{5}  - \frac{y^7}{7}  + \dots $

Now, $\arctan(1) = \frac{\pi}{4} $, so \ldots{}

~ ~ ~\$ \pi  = 4 , \big( - \frac{1}{3} + \frac{1}{5} - \frac{1}{7} +
\ldots{} \big) \$

We can represent the series expansion using a numpy Polynomial, with
coefficients: $p(x)$ = {[}0, ~ 1, ~ 0, ~ -1/3, ~ 0, ~ 1/5, ~ 0, ~
-1/7,\ldots{}{]}, and use it to approximate $\pi$.

    {[}NumPy{]} ~ Polynomials : calculating $\pi$ ~ II

~

    \begin{Verbatim}[commandchars=\\\{\}]
{\color{incolor}In [{\color{incolor} }]:} \PY{c}{\PYZsh{} calculate pi using polynomials}
        \PY{c}{\PYZsh{} import Polynomial class }
        \PY{k+kn}{from} \PY{n+nn}{numpy.polynomial} \PY{k+kn}{import} \PY{n}{Polynomial} \PY{k}{as} \PY{n}{poly}\PY{p}{;}
        \PY{n}{num} \PY{o}{=} \PY{l+m+mi}{100000}\PY{p}{;}
        \PY{n}{denominator} \PY{o}{=} \PY{n}{np}\PY{o}{.}\PY{n}{arange}\PY{p}{(}\PY{n}{num}\PY{p}{)}\PY{p}{;} 
        
        \PY{n}{denominator}\PY{p}{[}\PY{l+m+mi}{3}\PY{p}{:}\PY{p}{:}\PY{l+m+mi}{4}\PY{p}{]} \PY{o}{*}\PY{o}{=} \PY{o}{\PYZhy{}}\PY{l+m+mi}{1} \PY{c}{\PYZsh{} every other odd coefficient is \PYZhy{}ve}
        \PY{n}{numerator} \PY{o}{=} \PY{n}{np}\PY{o}{.}\PY{n}{ones}\PY{p}{(}\PY{n}{denominator}\PY{o}{.}\PY{n}{size}\PY{p}{)}\PY{p}{;} 
        
        \PY{c}{\PYZsh{} avoid dividing by zero, drop first element denominator}
        \PY{n}{almost} \PY{o}{=} \PY{n}{numerator}\PY{p}{[}\PY{l+m+mi}{1}\PY{p}{:}\PY{p}{]}\PY{o}{/}\PY{n}{denominator}\PY{p}{[}\PY{l+m+mi}{1}\PY{p}{:}\PY{p}{]}\PY{p}{;} 
        
        \PY{c}{\PYZsh{} make even coefficients zero}
        \PY{n}{almost}\PY{p}{[}\PY{l+m+mi}{1}\PY{p}{:}\PY{p}{:}\PY{l+m+mi}{2}\PY{p}{]} \PY{o}{=} \PY{l+m+mi}{0}
        
        \PY{c}{\PYZsh{} add back zero coefficient}
        \PY{n}{coeffs} \PY{o}{=} \PY{n}{np}\PY{o}{.}\PY{n}{r\PYZus{}}\PY{p}{[}\PY{l+m+mi}{0}\PY{p}{,}\PY{n}{almost}\PY{p}{]}\PY{p}{;} 
        
        \PY{n}{p} \PY{o}{=} \PY{n}{poly}\PY{p}{(}\PY{n}{coeffs}\PY{p}{)}\PY{p}{;}
        \PY{l+m+mi}{4}\PY{o}{*}\PY{n}{p}\PY{p}{(}\PY{l+m+mi}{1}\PY{p}{)} \PY{c}{\PYZsh{} pi approximation}
\end{Verbatim}

    {[}NumPy{]} ~ Performance ~ I

Python has a convenient timing function called timeit.

\begin{itemize}
\item
  Can use this to measure the execution time of small code snippets.
\item
  To use timeit function
\item
  import module timeit and use timeit.timeit or
\item
  use magic command \%timeit in an IPython shell
\end{itemize}

    {[}NumPy{]} ~ Performance ~ II

By default, timeit:

\begin{itemize}
\itemsep1pt\parskip0pt\parsep0pt
\item
  Takes the best time out of 3 repeat tests (-r)
\item
  takes the average time for a number of iterations (-n) per repeat
\end{itemize}

In an IPython shell:

\begin{itemize}
\item
  \%timeit ~ -n\textless{}iterations\textgreater{} ~
  ~-r\textless{}repeats\textgreater{} ~ ~\textless{}code\textgreater{}
\item
  query \%timeit? for more information
\item
  https://docs.python.org/2/library/timeit.html
\end{itemize}

    {[}NumPy{]} ~ Performance : experiments ~ I

Here are some timeit experiments for you to run.

    \begin{Verbatim}[commandchars=\\\{\}]
{\color{incolor}In [{\color{incolor} }]:} \PY{c}{\PYZsh{} accessing a 2d array}
        \PY{n}{nd} \PY{o}{=} \PY{n}{np}\PY{o}{.}\PY{n}{arange}\PY{p}{(}\PY{l+m+mi}{100}\PY{p}{)}\PY{o}{.}\PY{n}{reshape}\PY{p}{(}\PY{p}{(}\PY{l+m+mi}{10}\PY{p}{,}\PY{l+m+mi}{10}\PY{p}{)}\PY{p}{)}
        
        \PY{c}{\PYZsh{} accessing element of 2d array}
        \PY{o}{\PYZpc{}}\PY{k}{timeit} \PYZhy{}n10000000 \PYZhy{}r3 nd[5][5]
        \PY{o}{\PYZpc{}}\PY{k}{timeit} \PYZhy{}n10000000 \PYZhy{}r3 nd[(5,5)]
\end{Verbatim}

    \begin{Verbatim}[commandchars=\\\{\}]
{\color{incolor}In [{\color{incolor} }]:} \PY{c}{\PYZsh{} Ex: multiplying two vectors}
        \PY{n}{x}\PY{o}{=}\PY{n}{np}\PY{o}{.}\PY{n}{arange}\PY{p}{(}\PY{l+m+mf}{10E7}\PY{p}{)}
        \PY{o}{\PYZpc{}}\PY{k}{timeit} \PYZhy{}n1 \PYZhy{}r10 x*x
        \PY{o}{\PYZpc{}}\PY{k}{timeit} \PYZhy{}n1 \PYZhy{}r10 x**2
        
        \PY{c}{\PYZsh{} Ex++: from the linear algebra package}
        \PY{o}{\PYZpc{}}\PY{k}{timeit} \PYZhy{}n1 \PYZhy{}r10 np.dot(x,x) 
\end{Verbatim}

    {[}NumPy{]} ~ Performance : experiments ~ II

~

    \begin{Verbatim}[commandchars=\\\{\}]
{\color{incolor}In [{\color{incolor} }]:} \PY{k+kn}{import} \PY{n+nn}{numpy} \PY{k+kn}{as} \PY{n+nn}{np}
        \PY{c}{\PYZsh{} Ex: range functions and iterating in  for loops}
        \PY{n}{size} \PY{o}{=} \PY{n+nb}{int}\PY{p}{(}\PY{l+m+mf}{1E6}\PY{p}{)}\PY{p}{;}
        
        \PY{o}{\PYZpc{}}\PY{k}{timeit} for x in range(size): x ** 2
        
        \PY{c}{\PYZsh{} faster than range for very large arrays?}
        \PY{o}{\PYZpc{}}\PY{k}{timeit} for x in xrange(size): x ** 2
        
        \PY{o}{\PYZpc{}}\PY{k}{timeit} for x in np.arange(size): x ** 2
        
        \PY{o}{\PYZpc{}}\PY{k}{timeit} np.arange(size) ** 2
\end{Verbatim}

    \begin{Verbatim}[commandchars=\\\{\}]
{\color{incolor}In [{\color{incolor}3}]:} \PY{c}{\PYZsh{} Ex: look at the calculating pi code }
        \PY{c}{\PYZsh{} Make sure you understand it. Time the code.}
\end{Verbatim}

    {[}NumPy{]} ~ Summary

\begin{itemize}
\item
  NumPy introduces multi-dimensional arrays to Python, which is crucial
  for efficient scientific computing
\item
  It also provides fast numerical routines for scientific computation
\item
  Next up: Matplotlib
\end{itemize}


    % Add a bibliography block to the postdoc
    
    
    
    \end{document}
