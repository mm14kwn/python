
% Default to the notebook output style

    


% Inherit from the specified cell style.




    
\documentclass{article}

    
    
    \usepackage{graphicx} % Used to insert images
    \usepackage{adjustbox} % Used to constrain images to a maximum size 
    \usepackage{color} % Allow colors to be defined
    \usepackage{enumerate} % Needed for markdown enumerations to work
    \usepackage{geometry} % Used to adjust the document margins
    \usepackage{amsmath} % Equations
    \usepackage{amssymb} % Equations
    \usepackage{eurosym} % defines \euro
    \usepackage[mathletters]{ucs} % Extended unicode (utf-8) support
    \usepackage[utf8x]{inputenc} % Allow utf-8 characters in the tex document
    \usepackage{fancyvrb} % verbatim replacement that allows latex
    \usepackage{grffile} % extends the file name processing of package graphics 
                         % to support a larger range 
    % The hyperref package gives us a pdf with properly built
    % internal navigation ('pdf bookmarks' for the table of contents,
    % internal cross-reference links, web links for URLs, etc.)
    \usepackage{hyperref}
    \usepackage{longtable} % longtable support required by pandoc >1.10
    \usepackage{booktabs}  % table support for pandoc > 1.12.2
    

    
    
    \definecolor{orange}{cmyk}{0,0.4,0.8,0.2}
    \definecolor{darkorange}{rgb}{.71,0.21,0.01}
    \definecolor{darkgreen}{rgb}{.12,.54,.11}
    \definecolor{myteal}{rgb}{.26, .44, .56}
    \definecolor{gray}{gray}{0.45}
    \definecolor{lightgray}{gray}{.95}
    \definecolor{mediumgray}{gray}{.8}
    \definecolor{inputbackground}{rgb}{.95, .95, .85}
    \definecolor{outputbackground}{rgb}{.95, .95, .95}
    \definecolor{traceback}{rgb}{1, .95, .95}
    % ansi colors
    \definecolor{red}{rgb}{.6,0,0}
    \definecolor{green}{rgb}{0,.65,0}
    \definecolor{brown}{rgb}{0.6,0.6,0}
    \definecolor{blue}{rgb}{0,.145,.698}
    \definecolor{purple}{rgb}{.698,.145,.698}
    \definecolor{cyan}{rgb}{0,.698,.698}
    \definecolor{lightgray}{gray}{0.5}
    
    % bright ansi colors
    \definecolor{darkgray}{gray}{0.25}
    \definecolor{lightred}{rgb}{1.0,0.39,0.28}
    \definecolor{lightgreen}{rgb}{0.48,0.99,0.0}
    \definecolor{lightblue}{rgb}{0.53,0.81,0.92}
    \definecolor{lightpurple}{rgb}{0.87,0.63,0.87}
    \definecolor{lightcyan}{rgb}{0.5,1.0,0.83}
    
    % commands and environments needed by pandoc snippets
    % extracted from the output of `pandoc -s`
    \providecommand{\tightlist}{%
      \setlength{\itemsep}{0pt}\setlength{\parskip}{0pt}}
    \DefineVerbatimEnvironment{Highlighting}{Verbatim}{commandchars=\\\{\}}
    % Add ',fontsize=\small' for more characters per line
    \newenvironment{Shaded}{}{}
    \newcommand{\KeywordTok}[1]{\textcolor[rgb]{0.00,0.44,0.13}{\textbf{{#1}}}}
    \newcommand{\DataTypeTok}[1]{\textcolor[rgb]{0.56,0.13,0.00}{{#1}}}
    \newcommand{\DecValTok}[1]{\textcolor[rgb]{0.25,0.63,0.44}{{#1}}}
    \newcommand{\BaseNTok}[1]{\textcolor[rgb]{0.25,0.63,0.44}{{#1}}}
    \newcommand{\FloatTok}[1]{\textcolor[rgb]{0.25,0.63,0.44}{{#1}}}
    \newcommand{\CharTok}[1]{\textcolor[rgb]{0.25,0.44,0.63}{{#1}}}
    \newcommand{\StringTok}[1]{\textcolor[rgb]{0.25,0.44,0.63}{{#1}}}
    \newcommand{\CommentTok}[1]{\textcolor[rgb]{0.38,0.63,0.69}{\textit{{#1}}}}
    \newcommand{\OtherTok}[1]{\textcolor[rgb]{0.00,0.44,0.13}{{#1}}}
    \newcommand{\AlertTok}[1]{\textcolor[rgb]{1.00,0.00,0.00}{\textbf{{#1}}}}
    \newcommand{\FunctionTok}[1]{\textcolor[rgb]{0.02,0.16,0.49}{{#1}}}
    \newcommand{\RegionMarkerTok}[1]{{#1}}
    \newcommand{\ErrorTok}[1]{\textcolor[rgb]{1.00,0.00,0.00}{\textbf{{#1}}}}
    \newcommand{\NormalTok}[1]{{#1}}
    
    % Define a nice break command that doesn't care if a line doesn't already
    % exist.
    \def\br{\hspace*{\fill} \\* }
    % Math Jax compatability definitions
    \def\gt{>}
    \def\lt{<}
    % Document parameters
    \title{L05\_FortranCInterface}
    
    
    

    % Pygments definitions
    
\makeatletter
\def\PY@reset{\let\PY@it=\relax \let\PY@bf=\relax%
    \let\PY@ul=\relax \let\PY@tc=\relax%
    \let\PY@bc=\relax \let\PY@ff=\relax}
\def\PY@tok#1{\csname PY@tok@#1\endcsname}
\def\PY@toks#1+{\ifx\relax#1\empty\else%
    \PY@tok{#1}\expandafter\PY@toks\fi}
\def\PY@do#1{\PY@bc{\PY@tc{\PY@ul{%
    \PY@it{\PY@bf{\PY@ff{#1}}}}}}}
\def\PY#1#2{\PY@reset\PY@toks#1+\relax+\PY@do{#2}}

\expandafter\def\csname PY@tok@gd\endcsname{\def\PY@tc##1{\textcolor[rgb]{0.63,0.00,0.00}{##1}}}
\expandafter\def\csname PY@tok@gu\endcsname{\let\PY@bf=\textbf\def\PY@tc##1{\textcolor[rgb]{0.50,0.00,0.50}{##1}}}
\expandafter\def\csname PY@tok@gt\endcsname{\def\PY@tc##1{\textcolor[rgb]{0.00,0.27,0.87}{##1}}}
\expandafter\def\csname PY@tok@gs\endcsname{\let\PY@bf=\textbf}
\expandafter\def\csname PY@tok@gr\endcsname{\def\PY@tc##1{\textcolor[rgb]{1.00,0.00,0.00}{##1}}}
\expandafter\def\csname PY@tok@cm\endcsname{\let\PY@it=\textit\def\PY@tc##1{\textcolor[rgb]{0.25,0.50,0.50}{##1}}}
\expandafter\def\csname PY@tok@vg\endcsname{\def\PY@tc##1{\textcolor[rgb]{0.10,0.09,0.49}{##1}}}
\expandafter\def\csname PY@tok@m\endcsname{\def\PY@tc##1{\textcolor[rgb]{0.40,0.40,0.40}{##1}}}
\expandafter\def\csname PY@tok@mh\endcsname{\def\PY@tc##1{\textcolor[rgb]{0.40,0.40,0.40}{##1}}}
\expandafter\def\csname PY@tok@go\endcsname{\def\PY@tc##1{\textcolor[rgb]{0.53,0.53,0.53}{##1}}}
\expandafter\def\csname PY@tok@ge\endcsname{\let\PY@it=\textit}
\expandafter\def\csname PY@tok@vc\endcsname{\def\PY@tc##1{\textcolor[rgb]{0.10,0.09,0.49}{##1}}}
\expandafter\def\csname PY@tok@il\endcsname{\def\PY@tc##1{\textcolor[rgb]{0.40,0.40,0.40}{##1}}}
\expandafter\def\csname PY@tok@cs\endcsname{\let\PY@it=\textit\def\PY@tc##1{\textcolor[rgb]{0.25,0.50,0.50}{##1}}}
\expandafter\def\csname PY@tok@cp\endcsname{\def\PY@tc##1{\textcolor[rgb]{0.74,0.48,0.00}{##1}}}
\expandafter\def\csname PY@tok@gi\endcsname{\def\PY@tc##1{\textcolor[rgb]{0.00,0.63,0.00}{##1}}}
\expandafter\def\csname PY@tok@gh\endcsname{\let\PY@bf=\textbf\def\PY@tc##1{\textcolor[rgb]{0.00,0.00,0.50}{##1}}}
\expandafter\def\csname PY@tok@ni\endcsname{\let\PY@bf=\textbf\def\PY@tc##1{\textcolor[rgb]{0.60,0.60,0.60}{##1}}}
\expandafter\def\csname PY@tok@nl\endcsname{\def\PY@tc##1{\textcolor[rgb]{0.63,0.63,0.00}{##1}}}
\expandafter\def\csname PY@tok@nn\endcsname{\let\PY@bf=\textbf\def\PY@tc##1{\textcolor[rgb]{0.00,0.00,1.00}{##1}}}
\expandafter\def\csname PY@tok@no\endcsname{\def\PY@tc##1{\textcolor[rgb]{0.53,0.00,0.00}{##1}}}
\expandafter\def\csname PY@tok@na\endcsname{\def\PY@tc##1{\textcolor[rgb]{0.49,0.56,0.16}{##1}}}
\expandafter\def\csname PY@tok@nb\endcsname{\def\PY@tc##1{\textcolor[rgb]{0.00,0.50,0.00}{##1}}}
\expandafter\def\csname PY@tok@nc\endcsname{\let\PY@bf=\textbf\def\PY@tc##1{\textcolor[rgb]{0.00,0.00,1.00}{##1}}}
\expandafter\def\csname PY@tok@nd\endcsname{\def\PY@tc##1{\textcolor[rgb]{0.67,0.13,1.00}{##1}}}
\expandafter\def\csname PY@tok@ne\endcsname{\let\PY@bf=\textbf\def\PY@tc##1{\textcolor[rgb]{0.82,0.25,0.23}{##1}}}
\expandafter\def\csname PY@tok@nf\endcsname{\def\PY@tc##1{\textcolor[rgb]{0.00,0.00,1.00}{##1}}}
\expandafter\def\csname PY@tok@si\endcsname{\let\PY@bf=\textbf\def\PY@tc##1{\textcolor[rgb]{0.73,0.40,0.53}{##1}}}
\expandafter\def\csname PY@tok@s2\endcsname{\def\PY@tc##1{\textcolor[rgb]{0.73,0.13,0.13}{##1}}}
\expandafter\def\csname PY@tok@vi\endcsname{\def\PY@tc##1{\textcolor[rgb]{0.10,0.09,0.49}{##1}}}
\expandafter\def\csname PY@tok@nt\endcsname{\let\PY@bf=\textbf\def\PY@tc##1{\textcolor[rgb]{0.00,0.50,0.00}{##1}}}
\expandafter\def\csname PY@tok@nv\endcsname{\def\PY@tc##1{\textcolor[rgb]{0.10,0.09,0.49}{##1}}}
\expandafter\def\csname PY@tok@s1\endcsname{\def\PY@tc##1{\textcolor[rgb]{0.73,0.13,0.13}{##1}}}
\expandafter\def\csname PY@tok@kd\endcsname{\let\PY@bf=\textbf\def\PY@tc##1{\textcolor[rgb]{0.00,0.50,0.00}{##1}}}
\expandafter\def\csname PY@tok@sh\endcsname{\def\PY@tc##1{\textcolor[rgb]{0.73,0.13,0.13}{##1}}}
\expandafter\def\csname PY@tok@sc\endcsname{\def\PY@tc##1{\textcolor[rgb]{0.73,0.13,0.13}{##1}}}
\expandafter\def\csname PY@tok@sx\endcsname{\def\PY@tc##1{\textcolor[rgb]{0.00,0.50,0.00}{##1}}}
\expandafter\def\csname PY@tok@bp\endcsname{\def\PY@tc##1{\textcolor[rgb]{0.00,0.50,0.00}{##1}}}
\expandafter\def\csname PY@tok@c1\endcsname{\let\PY@it=\textit\def\PY@tc##1{\textcolor[rgb]{0.25,0.50,0.50}{##1}}}
\expandafter\def\csname PY@tok@kc\endcsname{\let\PY@bf=\textbf\def\PY@tc##1{\textcolor[rgb]{0.00,0.50,0.00}{##1}}}
\expandafter\def\csname PY@tok@c\endcsname{\let\PY@it=\textit\def\PY@tc##1{\textcolor[rgb]{0.25,0.50,0.50}{##1}}}
\expandafter\def\csname PY@tok@mf\endcsname{\def\PY@tc##1{\textcolor[rgb]{0.40,0.40,0.40}{##1}}}
\expandafter\def\csname PY@tok@err\endcsname{\def\PY@bc##1{\setlength{\fboxsep}{0pt}\fcolorbox[rgb]{1.00,0.00,0.00}{1,1,1}{\strut ##1}}}
\expandafter\def\csname PY@tok@mb\endcsname{\def\PY@tc##1{\textcolor[rgb]{0.40,0.40,0.40}{##1}}}
\expandafter\def\csname PY@tok@ss\endcsname{\def\PY@tc##1{\textcolor[rgb]{0.10,0.09,0.49}{##1}}}
\expandafter\def\csname PY@tok@sr\endcsname{\def\PY@tc##1{\textcolor[rgb]{0.73,0.40,0.53}{##1}}}
\expandafter\def\csname PY@tok@mo\endcsname{\def\PY@tc##1{\textcolor[rgb]{0.40,0.40,0.40}{##1}}}
\expandafter\def\csname PY@tok@kn\endcsname{\let\PY@bf=\textbf\def\PY@tc##1{\textcolor[rgb]{0.00,0.50,0.00}{##1}}}
\expandafter\def\csname PY@tok@mi\endcsname{\def\PY@tc##1{\textcolor[rgb]{0.40,0.40,0.40}{##1}}}
\expandafter\def\csname PY@tok@gp\endcsname{\let\PY@bf=\textbf\def\PY@tc##1{\textcolor[rgb]{0.00,0.00,0.50}{##1}}}
\expandafter\def\csname PY@tok@o\endcsname{\def\PY@tc##1{\textcolor[rgb]{0.40,0.40,0.40}{##1}}}
\expandafter\def\csname PY@tok@kr\endcsname{\let\PY@bf=\textbf\def\PY@tc##1{\textcolor[rgb]{0.00,0.50,0.00}{##1}}}
\expandafter\def\csname PY@tok@s\endcsname{\def\PY@tc##1{\textcolor[rgb]{0.73,0.13,0.13}{##1}}}
\expandafter\def\csname PY@tok@kp\endcsname{\def\PY@tc##1{\textcolor[rgb]{0.00,0.50,0.00}{##1}}}
\expandafter\def\csname PY@tok@w\endcsname{\def\PY@tc##1{\textcolor[rgb]{0.73,0.73,0.73}{##1}}}
\expandafter\def\csname PY@tok@kt\endcsname{\def\PY@tc##1{\textcolor[rgb]{0.69,0.00,0.25}{##1}}}
\expandafter\def\csname PY@tok@ow\endcsname{\let\PY@bf=\textbf\def\PY@tc##1{\textcolor[rgb]{0.67,0.13,1.00}{##1}}}
\expandafter\def\csname PY@tok@sb\endcsname{\def\PY@tc##1{\textcolor[rgb]{0.73,0.13,0.13}{##1}}}
\expandafter\def\csname PY@tok@k\endcsname{\let\PY@bf=\textbf\def\PY@tc##1{\textcolor[rgb]{0.00,0.50,0.00}{##1}}}
\expandafter\def\csname PY@tok@se\endcsname{\let\PY@bf=\textbf\def\PY@tc##1{\textcolor[rgb]{0.73,0.40,0.13}{##1}}}
\expandafter\def\csname PY@tok@sd\endcsname{\let\PY@it=\textit\def\PY@tc##1{\textcolor[rgb]{0.73,0.13,0.13}{##1}}}

\def\PYZbs{\char`\\}
\def\PYZus{\char`\_}
\def\PYZob{\char`\{}
\def\PYZcb{\char`\}}
\def\PYZca{\char`\^}
\def\PYZam{\char`\&}
\def\PYZlt{\char`\<}
\def\PYZgt{\char`\>}
\def\PYZsh{\char`\#}
\def\PYZpc{\char`\%}
\def\PYZdl{\char`\$}
\def\PYZhy{\char`\-}
\def\PYZsq{\char`\'}
\def\PYZdq{\char`\"}
\def\PYZti{\char`\~}
% for compatibility with earlier versions
\def\PYZat{@}
\def\PYZlb{[}
\def\PYZrb{]}
\makeatother


    % Exact colors from NB
    \definecolor{incolor}{rgb}{0.0, 0.0, 0.5}
    \definecolor{outcolor}{rgb}{0.545, 0.0, 0.0}



    
    % Prevent overflowing lines due to hard-to-break entities
    \sloppy 
    % Setup hyperref package
    \hypersetup{
      breaklinks=true,  % so long urls are correctly broken across lines
      colorlinks=true,
      urlcolor=blue,
      linkcolor=darkorange,
      citecolor=darkgreen,
      }
    % Slightly bigger margins than the latex defaults
    
    \geometry{verbose,tmargin=1in,bmargin=1in,lmargin=1in,rmargin=1in}
    
    

    \begin{document}
    
    
    \maketitle
    
    

    
    \section{Python as Glue }\label{python-as-glue}

Kevin Stratford ~ ~ ~ ~ ~ ~ ~ ~ ~ ~kevin@epcc.ed.ac.uk Emmanouil
Farsarakis ~ ~ farsarakis@epcc.ed.ac.uk

Contributing authors: Neelofer Banglawala Andy Turner Arno Proeme 

    ~

    ~

www.archer.ac.uk support@archer.ac.uk

 

    {[}Python as Glue{]} ~ Scientific Workflows ~

\begin{itemize}
\itemsep1pt\parskip0pt\parsep0pt
\item
  Data analysis / Simulation / Numerical computation

  \begin{itemize}
  \itemsep1pt\parskip0pt\parsep0pt
  \item
    becoming larger amd more complex
  \item
    can be time critical
  \item
    consuming ever-increasing amounts of resource
  \end{itemize}
\item
  There is a need to

  \begin{itemize}
  \itemsep1pt\parskip0pt\parsep0pt
  \item
    manage data, executable programs; and dispose of output
  \item
    develop flexible applications rapidly
  \item
    concert (administratively or geographically) disparate resources
  \end{itemize}
\end{itemize}

    {[}Python as Glue{]} ~ Horses for Courses ~

\begin{itemize}
\itemsep1pt\parskip0pt\parsep0pt
\item
  ``Tradiational'' languages (C/C++/Fortran)

  \begin{itemize}
  \itemsep1pt\parskip0pt\parsep0pt
  \item
    provide large existing pool of standard libraries / standalone
    executables
  \item
    are good as expressing numerical algorithms
  \item
    use compilation, admitting fast code
  \end{itemize}
\item
  Python

  \begin{itemize}
  \itemsep1pt\parskip0pt\parsep0pt
  \item
    provides a fast development cycle
  \item
    Good at dealing with ``messy'' unstructured data
  \item
    Allows easy interaction with data / OS / Internet
  \end{itemize}
\end{itemize}

    {[}Python as Glue{]} ~ The Best of Both Worlds ~

\begin{itemize}
\item
  Broadly two approaches:

  \begin{itemize}
  \itemsep1pt\parskip0pt\parsep0pt
  \item
    Loosely coupled: interact via operating system / external service
  \item
    Strongly coupled: interact at the level of code API
  \end{itemize}
\end{itemize}

    {[}Python as Glue{]} ~ Loose Coupling I

\begin{itemize}
\itemsep1pt\parskip0pt\parsep0pt
\item
  Standard library includes

  \begin{itemize}
  \itemsep1pt\parskip0pt\parsep0pt
  \item
    Data handling: strings, data types
  \item
    Data persistence, compression archiving
  \item
    Data base functionality
  \end{itemize}
\item
  Interact with ``outside world'' including

  \begin{itemize}
  \itemsep1pt\parskip0pt\parsep0pt
  \item
    Operating system
  \item
    Internet protocols
  \item
    HTML/XML support
  \end{itemize}
\end{itemize}

https://docs.python.org/2/library/

    {[}Python as Glue{]} ~ Loose Coupling II

\begin{itemize}
\itemsep1pt\parskip0pt\parsep0pt
\item
  General workflow:

  \begin{itemize}
  \itemsep1pt\parskip0pt\parsep0pt
  \item
    Marshal input data / control parameters
  \item
    Launch executable program via OS
  \item
    Review output
  \item
    {[}Refine and repeat?{]}
  \end{itemize}
\item
  Care with

  \begin{itemize}
  \itemsep1pt\parskip0pt\parsep0pt
  \item
    portability issues
  \item
    interaction with queue systems
  \item
    parallelism
  \end{itemize}
\end{itemize}

    {[}Python as Glue{]} ~ Strong Coupling ~ I

\begin{itemize}
\itemsep1pt\parskip0pt\parsep0pt
\item
  Call one language from another

  \begin{enumerate}
  \def\labelenumi{\arabic{enumi}.}
  \itemsep1pt\parskip0pt\parsep0pt
  \item
    python calls target language
  \item
    target language calls python
  \item
    ``two-way coupling'' ~
  \end{enumerate}
\item
  We will consider case (1)

  \begin{itemize}
  \item
    \begin{enumerate}
    \def\labelenumi{(\arabic{enumi})}
    \setcounter{enumi}{1}
    \itemsep1pt\parskip0pt\parsep0pt
    \item
      is possible via C native API, but probably not the model we want
    \end{enumerate}
  \item
    \begin{enumerate}
    \def\labelenumi{(\arabic{enumi})}
    \setcounter{enumi}{2}
    \itemsep1pt\parskip0pt\parsep0pt
    \item
      likewise, probably undesirable here
    \end{enumerate}
  \end{itemize}
\end{itemize}

    {[}Python as Glue{]} ~ Strong Coupling ~ II

\begin{enumerate}
\def\labelenumi{\arabic{enumi}.}
\itemsep1pt\parskip0pt\parsep0pt
\item
  Existing code in target language

  \begin{itemize}
  \itemsep1pt\parskip0pt\parsep0pt
  \item
    should have well-defined API (i.e., is a library)
  \item
    may need to be ``re-entrant''
  \item
    may be parallel
  \end{itemize}
\end{enumerate}

\begin{itemize}
\itemsep1pt\parskip0pt\parsep0pt
\item
  Exact approach depends on

  \begin{itemize}
  \itemsep1pt\parskip0pt\parsep0pt
  \item
    target language
  \item
    whether a clean separation of python/target language is required
  \item
    what the python interface should look like
  \end{itemize}
\end{itemize}

    {[}Python as Glue{]} ~ Fortran ~ I

\begin{itemize}
\itemsep1pt\parskip0pt\parsep0pt
\item
  Python style is typically to have

  \begin{itemize}
  \itemsep1pt\parskip0pt\parsep0pt
  \item
    functions with dummy arguments that remain unchanged
  \item
    result object(s) returned via return list
  \end{itemize}
\item
  Fortran

  \begin{itemize}
  \itemsep1pt\parskip0pt\parsep0pt
  \item
    functions with intent(in) arguments ok
  \item
    subroutines with intent(inout) arguments?
  \end{itemize}
\end{itemize}

    {[}Python as Glue{]} ~ Fortran ~ II

\begin{itemize}
\itemsep1pt\parskip0pt\parsep0pt
\item
  How to proceed

  \begin{itemize}
  \itemsep1pt\parskip0pt\parsep0pt
  \item
    numpy supplies, as standard, f2py
  \item
    tool to create python interface directly from Fortran
  \item
    Servicable for external subroutines (a la Fortran 77)
  \end{itemize}
\item
  In practice, for ``modern'' Fortran

  \begin{itemize}
  \itemsep1pt\parskip0pt\parsep0pt
  \item
    Number of other tools have been developed
  \item
    Most (e.g., pyfort) have no active development
  \item
    Best option appears to be f90wrap
  \end{itemize}
\end{itemize}

https://github.com/jameskermode/f90wrap

    {[}Python as Glue{]} ~ Fortran ~ III

\begin{itemize}
\itemsep1pt\parskip0pt\parsep0pt
\item
  What f90wrap does

  \begin{itemize}
  \itemsep1pt\parskip0pt\parsep0pt
  \item
    command line tool (python)
  \item
    operates on Fortran source (module.f90)
  \item
    generates a simplified Fortran interface module
  \end{itemize}
\item
  Uses native compiler to

  \begin{itemize}
  \itemsep1pt\parskip0pt\parsep0pt
  \item
    compile module.f90 and the simplified interface
  \end{itemize}
\end{itemize}

    {[}Python as Glue{]} ~ Fortran ~ IV

\begin{itemize}
\itemsep1pt\parskip0pt\parsep0pt
\item
  Uses f2py to generate

  \begin{itemize}
  \itemsep1pt\parskip0pt\parsep0pt
  \item
    python extension module describing python interface
  \item
    module.py
  \item
    a shared object module.so
  \end{itemize}
\item
  From python (script or shell)

  \begin{itemize}
  \itemsep1pt\parskip0pt\parsep0pt
  \item
    interface available via import module
  \end{itemize}
\end{itemize}

    {[}Python as Glue{]} ~ C/C++ I

\begin{itemize}
\itemsep1pt\parskip0pt\parsep0pt
\item
  Python to C/C++

  \begin{itemize}
  \itemsep1pt\parskip0pt\parsep0pt
  \item
    is a more natural ``match''
  \item
    more interfacing alternatives available
  \end{itemize}
\item
  Some provide a clean separation

  \begin{itemize}
  \itemsep1pt\parskip0pt\parsep0pt
  \item
    C foreign function interface (CFFI)
  \end{itemize}
\end{itemize}

https://cffi.readthedocs.org/en/latest/

    {[}Python as Glue{]} ~ C/C++ II

\begin{itemize}
\itemsep1pt\parskip0pt\parsep0pt
\item
  Others allow patching code into python

  \begin{itemize}
  \itemsep1pt\parskip0pt\parsep0pt
  \item
    Weave (part of scipy)
  \end{itemize}
\item
  Other still require ``intermediate'' language

  \begin{itemize}
  \itemsep1pt\parskip0pt\parsep0pt
  \item
    Cython - C externsions to python
  \item
    A superset of python
  \end{itemize}
\end{itemize}

http://docs.scipy.org/doc/scipy-0.14.0/reference/tutorial/weave.html
http://cython.org/

    {[}Python as Glue{]} ~ Summary

\begin{itemize}
\itemsep1pt\parskip0pt\parsep0pt
\item
  Many possibilities

  \begin{itemize}
  \itemsep1pt\parskip0pt\parsep0pt
  \item
    many are ``work in progress''
  \item
    some will fall by the wayside
  \end{itemize}
\item
  Care required

  \begin{itemize}
  \itemsep1pt\parskip0pt\parsep0pt
  \item
    choosing what to do in the first place
  \item
    identifyying sustainable solutions
  \end{itemize}
\end{itemize}


    % Add a bibliography block to the postdoc
    
    
    
    \end{document}
